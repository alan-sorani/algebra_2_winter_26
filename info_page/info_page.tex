\documentclass{article}
\usepackage{graphicx} % Required for inserting images

%-------------------------------
%           HYPERLINKS
%-------------------------------

\usepackage{hyperref}

%-------------------------------
%           MARGINS
%--------------------------------

\usepackage[top=2cm]{geometry}

%-------------------------------
%            HEBREW
%-------------------------------

\usepackage{hebrewdoc}

%-------------------------------
%            LISTS
%-------------------------------

\usepackage{enumitem}

%-------------------------------
%            TITLE
%-------------------------------

\title{דף מידע \\ אלגברה ב' - 01040168 \\ חורף 2026}
\date{}

\begin{document}

\maketitle

\section*{סגל הקורס}

\begin{description}
\item[מרצה:]
מקס גורביץ'
-
\textenglish{\href{mailto:maxg@technion.ac.il}{maxg@technion.ac.il}}
\\
\emph{משרד:}
אמאדו 911.
\\
\emph{שעת קבלה:}
יום א' במשרד בשעה 13:30-14:20, ובשעות נוספות בתיאום מראש.

\item[מתרגל:] 
אלן סורני
-
\textenglish{\href{mailto:elad.tzorani@technion.ac.il}{elad.tzorani@technion.ac.il}}
\\
\emph{שעת קבלה:}
יום ג' באולמן 100 בשעה 14:30-15:20 (אחרי התרגול), ובשעות נוספות בתיאום מראש.

\end{description}

\section*{מבנה הציון}

הציון הסופי בקורס יקבע באופן הבא.

\begin{itemize}
\item[-] \emph{10\%}
מגן עבור גיליונות להגשה.
\\
12 גיליונות תרגילים יפורסמו בערך אחת לשבוע, וציונם יחושב לטובת מגן. בשביל זכאות למגן, יש להגיש לפחות שלושה מכל ארבעה גיליונות עוקבים; כלומר לפחות שלושה מבין גיליונות 1-4, לפחות שלושה מבין גיליונות 5-8 וכו‘. ציון המגן יחושב כממוצע 9 הציונים הטובים ביותר עבור גיליונות ההגשה.
\\
המגן ילקח בחשבון בתנאי שהוא גבוה מציון הבחינה הסופית, ובתנאי שהציון בבחינה הסופית הינו עובר.
\item[-] \emph{15\%}
מגן עבור בוחן אמצע.
\\
המגן ילקח בחשבון בתנאי שהוא גבוה מציון הבחינה הסופית, ובתנאי שהציון בבחינה הסופית הינו עובר.
\item[-] \emph{75-100\%}
בחינה סופית.
\end{itemize}

\section*{מידע כללי}

\begin{itemize}
\item[-]
פניות בנושאי ניהול הקורס וגיליונות התרגילים יש להפנות לאלן המתרגל.
\item[-]
מותר להגיש תרגילי בית עד 5 ימי איחור מצטברים בלי אישור מיוחד. הארכות נוספות ינתנו עקב מילואים או מקרים רפואיים בלבד, באישור בכתב ובהתאם לשיקול הדעת של סגל הקורס.
\item[-]
יש להגיש פתרון להגשה בקובץ \textenglish{.pdf} יחיד בלבד, עם עמודים בגודל סטנדרטי, בכתב קריא, ועם דפים מיושרים כלפי מעלה.
\\
מותר ואף רצוי להגיש פתרונות מוקלדים.
\end{itemize}

\pagebreak

\section*{סילבוס}
\begin{enumerate}
\item סכומים ישרים, הטלות, תת־מרחבים שמורים וצמצום של אופרטור לינארי.
\item המרחב הדואלי.
\item מרחבי מכפלה פנימית, המשלים הניצב ותהליך גרם־שמידט. הטלות אורתוגונליות.
\item האופרטור הצמוד, אופרטורים נורמליים, אופרטורים צמודים לעצמם, אופרטורים אוניטריים ואיזומטריות.
\item משפט הפירוק הספקטרלי ולכסון אורתוגונלי. אופרטורים חיוביים ומשפט הפירוק הפולרי. \textenglish{SVD}.
\item אופרטורים נילפוטנטיים ואינדקס הנילפוטנטיות. צורת ז'ורדן.
\item הדטרמיננטה: הגדרה ותכונות. מטריצה מצורפת וכלל קרמר.
\item תבניות בילינאריות ותבניות ריבועיות. חפיפת מטריצות. משפט סילבסטר.
\item אלגברה מולטילינארית ומכפלות טנזוריות.
\end{enumerate}

\section*{ספרות}
\begin{itemize}
\item[-] \textenglish{Linear Algebra Done Right - S.~Axler}
\item[-] \textenglish{Algebra (2nd Edition) - M.~Artin}
\item[-] \textenglish{Linear Algebra - K.~Hoffman and R.F.~Kunze}
\end{itemize}

\end{document}