\documentclass[a4paper,10pt,twoside,openany]{article}

\usepackage[lang=hebrew]{maths}
\usepackage{hebrewdoc}
\usepackage{stylish}
\usepackage{lipsum}
\let\bs\blacksquare

\setlength{\parindent}{0pt}

%%%%%%%%%%%%
% Styling %
%%%%%%%%%%%%

\usepackage{enumitem}

%%%%%%%%%%%%%
% Counters  %
%%%%%%%%%%%%%

\setcounter{section}{1}     
            
%BIBLIOGRAPHY
\usepackage[
backend=biber,
style=alphabetic,
]{biblatex}
\addbibresource{bibliography.bib} %Imports bibliography file

%%%%%%%%%%
% Title  %
%%%%%%%%%%
\title{
אלגברה ב' - גיליון תרגילי בית 2 \\
מטריצות מייצגות
\\
\vspace{1cm}
\large{13.11.2025}
}
\date{}

\begin{document}
\maketitle

\begin{exercise}
יהי
$V = \mbb{C}_3\brs{x}$,
תהי
\begin{align*}
T \colon V &\to V \\
\text{,} \hphantom{lalala} p\prs{x} &\mapsto p\prs{x+1}
\end{align*}
יהי
$\St = \pmat{1,x,x^2,x^3}$,
\emph{הבסיס הסטנדרטי של $V$}
ותהי
$A = \pmat{0 & 1 & 0 & 0 \\ 1 & 0 & 0 & 0 \\ 0 & 0 & 0 & 1 \\ 0 & 0 & 1 & 0}$.
כיתבו מפורשות בסיס
$C$
של
$V$
עבורו
$A = \brs{T}^{\St}_C$.
\end{exercise}

\begin{exercise}
יהי
$V = \hom_{\RR}\prs{\RR^2, \RR^2}$
עם הבסיס הסטנדרטי
$\St = \prs{e_1, e_2}$,
ויהי
$D \coloneqq \prs{T_{1,1}, T_{1,2}, T_{2,1}, T_{2,2}}$
כאשר
$T_{i,j}$
מוגדרים על ידי
\begin{align*}
\text{.} T_{i,j}\prs{e_k} = \fcases{e_j & k = i \\ 0 & \text{otherwise}}
\end{align*}

יהי
\begin{align*}
R \colon \RR^2 &\to \RR^2 \\
\pmat{x \\ y} &\mapsto \pmat{x \\ -y}
\end{align*}
שיקוף דרך ציר ה־%
$x$,
ויהי
\begin{align*}
\phi \colon V &\to V \\
\text{.} \hphantom{lala} T &\mapsto R^{-1} \circ T \circ R
\end{align*}

מיצאו את המטריצה המייצגת
$\brs{\phi}_D$.
\end{exercise}

\begin{exercise}
יהי
$V$
מרחב וקטורי מעל שדה
$\mbb{F}$.
נזכיר כי שני אופרטורים
$T,S \in \End_{\mbb{F}}\prs{V}$
הינם
\emph{דומים}
אם קיים אופרטור הפיך
$P \in \End_{\mbb{F}}\prs{V}$
עבורו
$T = P^{-1} \circ S \circ P$.

\begin{enumerate}
\item הראו כי דמיון הינו יחס שקילות.

\item הראו כי אם
$\sim$
הינו יחס שקילות על קבוצה
$X$,
ו־%
$Y \subset X$,
אז
$\sim_Y$
המוגדר על ידי
\[y_1 \sim_Y y_2 \iff y_1 \sim y_2\]
הינו יחס שקילות על
$Y$.

\item יהי
$V = \Mat_2\prs{\CC}$
ותהי
$S \subseteq V$
קבוצת כל המטריצות ב־%
$V$
ש־%
$1$
הינו הערך העצמי היחיד שלהן.

מחלקת דמיון של מטריצה בקבוצה $X$, היא קבוצת כל המטריצות ב־%
$X$ הדומות לה,
וידוע כי ב־%
$S$
יש שתי מחלקות דמיון. כיתבו את שתיהן מפורשות.
\end{enumerate}
\end{exercise}

\end{document}