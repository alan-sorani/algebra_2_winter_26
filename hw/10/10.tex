\documentclass[a4paper,10pt,twoside]{article}

\usepackage[lang=hebrew]{maths}

\usepackage{siunitx,array}
\sisetup{table-format=-1.5, group-digits=false}
\newcolumntype{L}{>{\phantom{$-$}}l}       % prefix some whitespace
\newcommand\mcL[1]{\multicolumn{1}{L}{#1}} % handy shortcut macro

\usepackage{hebrewdoc}
\usepackage{stylish}
\usepackage{lipsum}
\let\bs\blacksquare

\setlength{\parindent}{0pt}

%%%%%%%%%%%%
% Styling %
%%%%%%%%%%%%

\usepackage{enumitem}

%%%%%%%%%%%%%
% Counters  %
%%%%%%%%%%%%%

\setcounter{section}{1}     
            
%BIBLIOGRAPHY
\usepackage[
backend=biber,
style=alphabetic,
]{biblatex}
\addbibresource{bibliography.bib} %Imports bibliography file

%%%%%%%%%%
% Title  %
%%%%%%%%%%
\title{
אלגברה ב' - גיליון תרגילי בית 10 \\
פולינומים אופייני ומינימלי, והדטרמיננטה
\\
\vspace{1cm}
\large{תאריך הגשה: 16.1.2026}
}
\date{}

\begin{document}
\maketitle

\begin{exercise}[חישוב פולינום מינימלי]
תהי
$T \in \End_\CC\prs{V}$
ויהי
$B$
בסיס של
$V$
עבורו
\[\text{.} \brs{T}_B = \pmat{a & b \\ c & d} \in \Mat_2\prs{\CC}\]
\begin{enumerate}
\item הראו כי
\[\text{,} p_T(x) = x^2 - \prs{a+d}x + \prs{ad - bc}\]
והסיקו כי
\[\text{.} T^2 - \prs{a+d} T + \prs{ad - bc} \id_V = 0\]

\item הראו כי
\begin{align*}
\text{.} m_T\prs{x} = \fcases{
x - a & b=c=0 \text{ and } a=d \\
x^2 - \prs{a+d}x + \prs{ad - bc} & \text{otherwise}
}
\end{align*}
\end{enumerate}
\end{exercise}

\begin{exercise}[פולינום מינימלי של אופרטור הופכי]
יהי
$V$
מרחב וקטורי סוף־מימדי מעל
$\CC$
ויהי
$T \in \End_\CC\prs{V}$
עם פולינום מינימלי
\[\text{.} m_T\prs{x} = x^5  + 2x^4 − 7x^3 - 6x^2 + 5x + 4\]
הראו כי
$T$
הפיך
ומיצאו את הפולינום המינימלי
$m_{T^{-1}}\prs{x}$
של
$T^{-1}$.
\end{exercise}

\begin{exercise}[חסם עליון לדרגה של הפולינום המינימלי]
יהי
$V$
מרחב וקטורי סוף־מימדי מעל
$\CC$
ויהי
$T \in \End_\CC\prs{V}$.

\begin{enumerate}
\item הראו כי
\[\text{.} \deg m_T\prs{x} \leq \dim\prs{\im\prs{T}} + 1\]

\item מיצאו דוגמה לאופרטור
$T$
עבורו מתקיים שוויון.
\end{enumerate}

\textbf{רמז:}
חישבו על צורת ז'ורדן של
$T$.
\end{exercise}

\begin{exercise}[דטרמיננטה לפי שורה או עמודה]
הגדרנו בהרצאה את הדטרמיננטה של
$A \in \Mat_n\prs{\FF}$
בתור
\begin{align*}
\text{.} \det\prs{A} = \sum_{\sigma \in S_n} \epsilon\prs{\sigma} \prod_{i \in \brs{k}} \prs{A}_{\sigma\prs{i}, i}
\end{align*}

הראו שמתקיים
\begin{align*}
\text{,} \det\prs{A} = \sum_{\sigma \in S_n} \epsilon\prs{\sigma} \prod_{i \in \brs{k}} \prs{A}_{i, \sigma\prs{i}}
\end{align*}
והסיקו כי
$\det\prs{A^t} = \det\prs{A}$.
\end{exercise}

\newpage

\begin{exercise}[כלל קרמר]
תהי
$Ax = b$
מערכת משוואות כאשר
$A \in M_n\prs{\mbb{F}}$
הפיכה.

לכל
$i \in [n]$
תהי
\begin{align*}
K_{A,i} \colon \mbb{F}^n &\to \mbb{F} \\
y &\mapsto \frac{\det\pmat{\vert & & \vert & \vert & \vert & & \vert \\ A_1 & \cdots & A_{i-1} & y & A_{i+1} & \cdots & A_n \\ \vert & & \vert & \vert & \vert & & \vert}}{\det\prs{A}}
\end{align*}
ותהי
\begin{align*}
K_A \colon \mbb{F}^n &\to \mbb{F}^n \\
\text{.} \hphantom{lalala} y &\mapsto \pmat{K_{A,1}\prs{y} \\ K_{A,2}\prs{y} \\ \vdots \\ K_{A, n-1}\prs{y} \\ K_{A,n}\prs{y}}
\end{align*}

נראה שהפתרון היחיד למערכת נתון על ידי
$x = K_{A}\prs{b}$.

\begin{enumerate}
\item הראו שלכל
$i \in [n]$
ההעתקה
$K_{A,i}$
לינארית.
\item
הסיקו ש־%
$K_A$
העתקה לינארית.
\item
תהי
\begin{align*}
L_A \colon \mbb{F}^n &\to \mbb{F}^n \\
\text{.} \hphantom{lalala} v &\mapsto Av
\end{align*}

הראו ש־%
$K_A \circ L_A = \id_{\mbb{F}^n}$
על ידי בדיקה על הבסיס הסטנדרטי, והסיקו שמתקיים
$K_A = \prs{L_A}^{-1}$.
\item הסיקו שמתקיים
$\brs{K_A}_E = A^{-1}$
כאשר
$E$
הבסיס הסטנדרטי של
$\mbb{F}^n$.
\item הסיקו שהפתרון היחיד למערכת
$Ax = b$
הוא
$x = K_{A}\prs{b}$.
\end{enumerate}
\end{exercise}

\begin{exercise}[חישוב בעזרת כלל קרמר]
פיתרו את מערכת המשוואות הבאה, בעזרת כלל קרמר.

\begin{center}
\begin{english}
\begin{tabular}{llllllllS}
4x & + & y & + & z & + & w & = & 6  \\
3x & + & 7y & - & z & + & w & = & 1 \\
7x & + & 3y & - & 5z & + & 8w & = & -3 \\
x & + & y & + & z & + & 2w & = & 3
\end{tabular}
\end{english}
\end{center}
\end{exercise}

\end{document}