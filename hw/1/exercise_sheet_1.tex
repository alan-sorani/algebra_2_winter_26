\documentclass[a4paper,10pt,twoside,openany]{article}

\usepackage[lang=hebrew]{maths}
\usepackage{hebrewdoc}
\usepackage{stylish}
\usepackage{lipsum}
\let\bs\blacksquare

\setlength{\parindent}{0pt}

%%%%%%%%%%%%
% Styling %
%%%%%%%%%%%%

\usepackage{enumitem}

%%%%%%%%%%%%%
% Counters  %
%%%%%%%%%%%%%

\setcounter{section}{1}     
            
%BIBLIOGRAPHY
\usepackage[
backend=biber,
style=alphabetic,
]{biblatex}
\addbibresource{bibliography.bib} %Imports bibliography file

%%%%%%%%%%
% Title  %
%%%%%%%%%%
\title{
אלגברה ב' - גיליון תרגילי בית 1 \\
מטריצות מייצגות
\\
\vspace{1cm}
\large{תאריך הגשה: 5.11.2025}
}
\date{}

\begin{document}
\maketitle

\begin{exercise}
יהי
$\mathrm{St} = \prs{e_1, \ldots, e_n}$
הבסיס הסטנדרטי של
$\mathbb{F}^n$,
ונזכיר כי עבור בסיס
$B$
של מרחב וקטורי סוף־מימדי
$V$
מעל
$\mbb{F}$
קיים איזומורפיזם
\begin{align*}
\rho_B : V &\to \mbb{F}^n \\
\text{.} \hphantom{lalala} v &\mapsto \brs{v}_B
\end{align*}
הראו כי
$\rho_{\mrm{St}} : \mathbb{F}^n \to \mathbb{F}^n$
הינה העתקת הזהות.
\end{exercise}

\begin{exercise}
יהי
$V$
מרחב וקטורי סוף־מימדי מעל שדה
$\mbb{F}$,
ויהיו
$B,C$
שני בסיסים של
$V$.
יהי
$T \in \End_{\mbb{F}}\prs{V}$
אופרטור לינארי על
$V$.
נסמן
$B = \prs{v_1, \ldots, v_n}$.

נראה בהרצאה כי
\begin{align*}
\text{.} \forall i \in \brs{n} : \brs{T}^B_C \brs{v_i}_B = \brs{T\prs{v_i}}_C
\end{align*}
הסיקו מכך כי
\begin{align*}
\text{.} \forall v \in V: \brs{T}^B_C \brs{v}_B = \brs{T\prs{v}}_C
\end{align*}
\end{exercise}

\begin{exercise}
יהי
$B = \prs{v_i}_{i \in [n]}$
בסיס למרחב וקטורי
$V$.
נתונה
$T \colon V \to V$
הפיכה המקיימת
\[\text{.} T\prs{v_1 + 2 v_2} = \sum_{i \in [n]} v_i\]
מצאו את סכום כל מקדמי המטריצה
$\brs{T^{-1}}_B$.

למשל, אם
$\brs{T^{-1}}_B = \pmat{1 & 2 \\ 3 & 4}$,
הסכום הוא
$1+2+3+4 = 10$.
\end{exercise}

\textbf{שימו לב לתרגילים הנוספים שבעמוד הבא.}

\newpage

עבור שני התרגילים הבאים, נזכיר את ההגדרה עבור מטריצות מעבר בסיס, ושתי תכונות שלהן.

\begin{definition}
יהי
$V$
מרחב וקטורי סוף־מימדי מעל שדה
$\mbb{F}$,
ויהיו
$B, C$
שני בסיסים של
$V$.
תהי
\begin{align*}
\id_V \colon V &\to V \\
v &\mapsto v
\end{align*}
העתקת הזהות על
$V$,
כלומר ההעתקה המקיימת
$\id_V\prs{v} = v$
לכל
$v \in V$.

נגדיר את מטריצת המעבר
$M^B_C$
בתור
\begin{align*}
\text{.} M^B_C = \brs{\id_V}^B_C
\end{align*}
\end{definition}

\textbf{שימו לב:}
למטריצה
$M^B_C$
לא נקרא מטריצת מעבר מהבסיס
$B$
לבסיס
$C$
או להיפך. ראו הערה
\ref{remark:change-of-basis}
שבעמוד הבא עבור הסבר בנושא.

\begin{proposition}
יהי
$V$
מרחב וקטורי סוף־מימדי מעל שדה
$\mbb{F}$,
ויהיו
$B,C,D$
שלושה בסיסים של
$V$.

אז:
\begin{enumerate}
\item $M^B_C$
מטריצה הפיכה, ומתקיים
$\prs{M^B_C}^{-1} = M^C_B$.
\item $M^B_C = M^D_C M^B_D$.
\end{enumerate}
\end{proposition}

\begin{exercise}
יהיו
$V,W$
מרחבים וקטוריים סוף־מימדיים מעל שדה
$\mbb{F}$
ותהי
$T \in \Hom_{\mbb{F}}\prs{V,W}$
העתקה לינארית חד־חד ערכית.

יהיו
\begin{align*}
B &= \prs{v_1, \ldots, v_n} \\
C &= \prs{u_1, \ldots, u_n}
\end{align*}
בסיסים של
$V$
ויהיו
\begin{align*}
B' &= \prs{T\prs{v_1}, \ldots, T\prs{v_n}} \\
\text{.} C' &= \prs{T\prs{u_1}, \ldots, T\prs{u_n}}
\end{align*}

\begin{enumerate}
\item
הראו כי
$\dim V = \dim \im T$.

\item הראו כי
$B', C'$
קבוצות בלתי תלויות לינאריות, והסיקו כי הן בסיסים של
$\im\prs{T} = \set{T\prs{v}}{v \in V}$.

\item
הראו כי
$M^B_C = M^{B'}_{C'}$.

\textbf{רמז:}
כיתבו את
$M^B_C e_i$
ואת
$M^{B'}_{C'} e_i$
כוקטורי קואורדינטות לפי הבסיסים
$C$
ו־%
$C'$
בהתאמה, והראו שמתקיים שוויון.
\end{enumerate}
\end{exercise}

\begin{exercise}
תהי
$A \in \Mat_{n}\prs{\mbb{F}}$
הפיכה.
\begin{enumerate}
\item יהי
$\mrm{St}$
הבסיס הסטנדרטי של
$\mbb{F}^n$.
מיצאו בסיס
$B$
של
$\mbb{F}^n$
עבורו
$A = M^B_{\mrm{St}}$.

\item
מיצאו בסיס
$C$
של
$\mbb{F}^n$
עבורו
$A = M^{\mrm{St}}_C$.

\item
יהי
$B$
בסיס של
$\mbb{F}^n$.
מיצאו בסיס
$C$
של
$\mbb{F}^n$
עבורו
$A = M^B_C$.

\textbf{רמז:}
היעזרו בבסיס הסטנדרטי של
$\mbb{F}^n$
ובאחת התכונות של מטריצות מעבר בסיס כדי לקבל שוויון הדומה לזה שבאחד הסעיפים הקודמים.

\item
יהי
$V$
מרחב וקטורי ממימד
$n \in \mbb{N}_+$
מעל
$\mbb{F}$,
יהי
$T \in \End_{\mbb{F}}\prs{V}$
איזומורפיזם
ויהי
$B = \prs{v_1, \ldots, v_n}$
בסיס של
$V$.
מיצאו בסיס
$C$
של
$V$
עבורו
$\brs{T}^B_C = A$.

\textbf{רמז:}
היעזרו בכך ש־%
$\brs{T}^B_C = M^B_C \brs{T}_B$
ובתרגיל הקודם יחד עם האופרטור
\begin{align*}
\rho_B \colon V &\to \mbb{F}^n \\
v &\mapsto \brs{v}_B
\end{align*}
ששולח וקטור
$v \in V$
לוקטור
$\brs{v}_B \in \mbb{F}^n$,
כדי להפוך את הבעיה לזאת של מציאת בסיס
$\hat{C}$
של
$\mbb{F}^n$
עבורו
$M^{\mrm{St}}_{\hat{C}} = A \prs{\brs{T}_B}^{-1}$.
כיתבו את
$C$
בעזרת הבסיס
$\hat{C}$.
\end{enumerate}
\end{exercise}

\newpage

\begin{remark} \label{remark:change-of-basis-matrices}
למטריצה
$M^B_C$
\emph{לא נקרא}
מטריצת מעבר מהבסיס
$B$
לבסיס
$C$
או להיפך.

מצד אחד, היא מקיימת
\[\text{,} \forall v \in V : M^B_C \brs{v}_B = \brs{v}_C\]
ולכן יש הקוראים לה מטריצת מעבר מהבסיס
$B$
לבסיס
$C$.

מצד שני, אם
$V = \mbb{F}^n$,
\[{\mrm{St}} = \prs{e_1, \ldots, e_n}\]
הבסיס הסטנדרטי של
$V$,
ונכתוב
\begin{align*}
B &= \St \\
\text{,} C &= \prs{c_1, \ldots, c_n}
\end{align*}
נקבל כי
\begin{align*}
M^B_C c_i &= M^{\St}_C \brs{c_i}_{\St}
\\&= \brs{c_i}_{C}
\\&= e_i
\end{align*}
כלומר
\begin{align*}
\text{.} \prs{M^B_C c_1, \ldots, M^B_C c_n} = B
\end{align*}
לכן, יש הקוראים למטריצה
$M^B_C$
מטריצת מעבר מהבסיס
$C$
לבסיס
$B$.

כדי למנוע בלבול, במקום להתייחס למטריצה כזאת כמטריצת מעבר מבסיס זה או אחר, נכתוב מפורשות $M^B_C$.
\end{remark}

\end{document}