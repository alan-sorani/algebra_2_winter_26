\documentclass[a4paper,10pt,twoside,openany]{article}

\usepackage[lang=hebrew]{maths}
\usepackage{hebrewdoc}
\usepackage{stylish}
\usepackage{lipsum}
\let\bs\blacksquare

\setlength{\parindent}{0pt}

%%%%%%%%%%%%
% Styling %
%%%%%%%%%%%%

\usepackage{enumitem}

%%%%%%%%%%%%%
% Counters  %
%%%%%%%%%%%%%

\setcounter{section}{1}     
            
%BIBLIOGRAPHY
\usepackage[
backend=biber,
style=alphabetic,
]{biblatex}
\addbibresource{bibliography.bib} %Imports bibliography file

%%%%%%%%%%
% Title  %
%%%%%%%%%%
\title{
אלגברה ב' - גיליון תרגילי בית 8 \\
אופרטורים נילפוטנטיים, ומרחבים עצמיים מוכללים
\\
\vspace{1cm}
\large{1.1.2026}
}
\date{}

\begin{document}
\maketitle

\begin{exercise}
יהי
$T$
אופרטור נילפונטי מאינדקס
$k$.

\begin{enumerate}
\item
הראו כי מתקיים
\begin{align*}
\set{0} = \im\prs{T^k} \subset \im\prs{T^{k-1}} \subset \ldots \subset \im\prs{T}
\end{align*}
כאשר כל ההכלות הינן הכלות ממש.

\item
הסיקו כי
$k \leq \dim \im\prs{T} + 1$.
וכי
$T^{\dim \im\prs{T} + 1} = 0$.
\end{enumerate}
\end{exercise}

\begin{exercise}
יהי
$T \in \End_\CC\prs{\CC^5}$
עבורו
$\dim \ker\prs{T^4} \neq \dim \ker\prs{T^5}$.
הראו כי
$T$
נילפוטנטי מאינדקס
$5$
ושקיים בסיס
$B$
עבורו
\[\text{.} \brs{T}_B = J_5 \coloneqq \pmat{0 & 1 & 0 & 0 & 0 \\ 0 & 0 & 1 & 0 & 0 \\ 0 & 0 & 0 & 1 & 0 \\ 0 & 0 & 0 & 0 & 1 \\ 0 & 0 & 0 & 0 & 0}\]
\end{exercise}

\begin{exercise}
יהי
$V$
מרחב וקטורי סוף־מימדי מעל
$\CC$
ויהי
$T \in \End_\CC\prs{V}$.
הראו שלכל
$i \geq r_a\prs{0}$
מתקיים
\[\text{,} \dim\ker\prs{T^i} = \dim\ker\prs{T^{r_a\prs{0}}}\]
כאשר
$r_a\prs{0}$
הוא הריבוי האלגברי של
$0$
כערך עצמי של
$T$.
\end{exercise}

\begin{exercise}
יהי
$V$
מרחב וקטורי ממימד
$n \in \NN_+$
מעל
$\CC$
ויהי
$T \in \End_\CC\prs{V}$
עם
$k$
ערכים עצמיים שונים.

\begin{enumerate}
\item הראו כי
\begin{align*}
\text{.} V = \ker\prs{T^{n-\prs{k-1}}} \oplus \im\prs{T^{{n-\prs{k-1}}}}
\end{align*}

\item הראו בעזרת דוגמה נגדית שלא מתקיים תמיד
\begin{align*}
\text{.} V = \ker\prs{T^{n-k}} \oplus \im\prs{T^{n-k}}
\end{align*}
\end{enumerate}
\end{exercise}

\end{document}