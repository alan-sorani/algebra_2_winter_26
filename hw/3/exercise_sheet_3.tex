\documentclass[a4paper,10pt,twoside,openany]{article}

\usepackage[lang=hebrew]{maths}
\usepackage{hebrewdoc}
\usepackage{stylish}
\usepackage{lipsum}
\let\bs\blacksquare

\setlength{\parindent}{0pt}

%%%%%%%%%%%%
% Styling %
%%%%%%%%%%%%

\usepackage{enumitem}

%%%%%%%%%%%%%
% Counters  %
%%%%%%%%%%%%%

\setcounter{section}{1}     
            
%BIBLIOGRAPHY
\usepackage[
backend=biber,
style=alphabetic,
]{biblatex}
\addbibresource{bibliography.bib} %Imports bibliography file

%%%%%%%%%%
% Title  %
%%%%%%%%%%
\title{
אלגברה ב' - גיליון תרגילי בית 3 \\
מרחבים שמורים וסכומים ישרים
\\
\vspace{1cm}
\large{20.11.2025}
}
\date{}

\begin{document}
\maketitle

\begin{exercise}
יהיו
$V$
מרחב וקטורי
מעל
$\mbb{F}$,
יהיו
$P,T \in \End_{\mbb{F}}\prs{V}$
כאשר
$P$
איזומורפיזם.
יהי
$W \leq V$.

הראו כי
$W$
הינו
$T$%
־שמור אם ורק אם
$P^{-1}\prs{W}$
הינו
$P^{-1} \circ T \circ P$%
־שמור.
\end{exercise}

\begin{exercise}
יהי
$V$
מרחב וקטורי סוף־מימדי, יהי
$T \in \End_{\mbb{F}}\prs{V}$
ויהיו
$V_1, \ldots, V_k \leq V$
כולם
$T$%
־שמורים וכך שמתקיים
$V = \bigoplus_{i \in [k]} V_i$.

\begin{enumerate}
\item 
הראו כי
\begin{align*}
\ker\prs{T} &= \bigoplus_{i \in [k]} \ker\prs{\left. T \right|_{V_i}} \\
\text{.} \im\prs{T} &= \bigoplus_{i \in [k]} \im\prs{\left. T \right|_{V_i}}
\end{align*}

\item
הראו כי
\begin{align*}
\ker\prs{T - \lambda \id_V} &= \bigoplus_{i \in [k]} \ker\prs{\left. T \right|_{V_i} - \lambda \id_{V_i}}
\end{align*}
לכל
$\lambda \in \mbb{F}$.
הסיקו שהערכים העצמיים של
$T$
הם אלו של כל ה־%
$\left. T \right|_{V_i}$
והראו כי
\begin{align*}
r_{T,a}\prs{\lambda} &= \sum_{i \in \brs{k}} r_{\left. T \right|_{V_i}, a}\prs{\lambda} \\
r_{T,g}\prs{\lambda} &= \sum_{i \in \brs{k}} r_{\left. T \right|_{V_i}, g}\prs{\lambda}
\end{align*}
לכל
$\lambda \in \mbb{F}$
וכאשר
$r_{S,a}\prs{\lambda}, r_{S,g}\prs{\lambda}$
הריבויים האלגברי והגיאומטרי של
$\lambda$
כערך עצמי של
$S \in \End_{\mbb{F}}\prs{V}$.
\end{enumerate}
\end{exercise}

\begin{exercise}
יהי
$V$
מרחב וקטורי סוף־מימדי מעל שדה
$\mbb{F}$
ויהי
$T \in \End_\FF\prs{V}$.
יהיו
\begin{align*}
\set{0} \neq W_1 < \ldots < W_m \leq V
\end{align*}
תת־מרחבים
$T$%
־שמורים
של
$V$
כך ש־%
$W_i$
תת־מרחב ממש של
$W_{i+1}$
לכל
$i \in \brs{m-1}$,
וגם
$W_1 \neq 0$.
נסמן
$k_i \coloneqq \dim_\FF\prs{W_i}$,
ויהי
$B$
בסיס של
$V$
כך ש־%
$k_i$
הוקטורים הראשונים של
$V$
מהווים בסיס ל־%
$W_i$,
לכל
$i \in \brs{m}$.

הראו כי
$\brs{T}_B$
מטריצה מהצורה
\begin{align*}
\pmat{A_1 & * & \cdots & * \\ 0 & A_2 & \ddots & \vdots \\ \vdots & \ddots & \ddots & * \\ 0 & \cdots & 0 & A_m}
\end{align*}
עבור מטריצות
$A_i \in \Mat_{k_i}\prs{\FF}$.

\textbf{רמז:}
היזכרו במסקנה מהכיתה לגבי מציאת בסיס
$B$
עבורו
$\brs{T}_B$
מטריצה משולשת עליונה.
\end{exercise}

\end{document}