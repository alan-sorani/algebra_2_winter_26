\documentclass[a4paper,10pt,twoside,openany]{article}

\usepackage[lang=hebrew]{maths}
\usepackage{hebrewdoc}
\usepackage{stylish}
\usepackage{lipsum}
\let\bs\blacksquare

\setlength{\parindent}{0pt}

%%%%%%%%%%%%
% Styling %
%%%%%%%%%%%%

\usepackage{enumitem}

%%%%%%%%%%%%%
% Counters  %
%%%%%%%%%%%%%

\setcounter{section}{1}     
            
%BIBLIOGRAPHY
\usepackage[
backend=biber,
style=alphabetic,
]{biblatex}
\addbibresource{bibliography.bib} %Imports bibliography file

%%%%%%%%%%
% Title  %
%%%%%%%%%%
\title{
אלגברה ב' - גיליון תרגילי בית 9 \\
משפט ז'ורדן
\\
\vspace{1cm}
\large{תאריך הגשה: 8.1.2026}
}
\date{}

\begin{document}
\maketitle

\begin{exercise}%1
\begin{enumerate}
\item מצאו את צורת ז'ורדן של
$J_n\prs{\lambda}^{-1}$
עבור
$\lambda \in \mbb{C}\setminus\set{0}$
ועבור
$n \in \mbb{N}_+$.

\item תהי
$A \in M_n\prs{\mbb{C}}$
הפיכה.
מצאו את צורת ז'ורדן של
$A^{-1}$.

\item מצאו תנאי הכרחי ומספיק על צורת ז'ורדן של
$A \in M_n\prs{\mbb{C}}$
כך שיתקיים
$A \sim A^{-1}$.
\end{enumerate}
\end{exercise}

\begin{exercise} %2
בשמורת הטבע ליד הטכניון סין יש היום 2 דרקונים, 600 פנדות ו־20000 במבוקים.

כל שנה הדרקונים, הפנדות והבמבוקים מתרבים ומספרם גדל פי 2.

לאחר מכן, כל פנדה אוכלת במבוק אחד וכל דרקון אוכל שתי פנדות.

אז, רשות הטבע והגנים הסינית משחררת לטבע 4 דרקונים ו־10 פנדות, אם עדיין יש פנדות בשמורה.

לבסוף, אם לא נשאר במבוק בסוף השנה, כל הפנדות מתות.

\begin{enumerate}
\item
מיצאו מטריצה
$A \in \Mat_4\prs{\mbb{C}}$
וערכים
$d,p,b$
עבורם מספרי הדרקונים, הפנדות והבמבוקים בסוף השנה ה־%
$t$
הם מקדמים בוקטור
$A^t \pmat{1 \\ d \\ p \\ b}$
לכל
$t \in \mbb{N}$.

\item נשיא הטכניון מתכנן לבקר בסין עוד 30 שנה. האם יהיו פנדות בשמורה בזמן הביקור שלו?

\item הטכניון החליט להעביר את הלימודים מסין למאדים עוד 230 שנה. האם ישארו עד אז פנדות בשמורת הטבע?
\end{enumerate}
\end{exercise}

\begin{exercise}
תהיינה
\begin{align*}
A_1 &\coloneqq \pmat{1 & -2 & 2 \\ 2 & -1 & 2 \\ 2 & -2 & 3} \\
A_2 &\coloneqq \pmat{1 & 2 & 2 \\ 2 & 1 & 2 \\ 2 & 2 & 3} \\
A_3 &\coloneqq \pmat{1 & 2 & -2 \\ 2 & 1 & -2 \\ 2 & 2 & -3}
\end{align*}
מטריצות ב־%
$\Mat_3\prs{\RR}$,
ויהי
$v_0 = \pmat{3 \\ 4 \\ 5}$.

נגיד כי וקטור
$v = \pmat{a \\ b \\ c} \in \RR^3$
הוא
\emph{שלשה פיתגורית}
אם
\[\text{,} f\prs{v} \coloneqq a^2 + b^2 - c^2 = 0\]
וכי וקטור כזה הוא
\emph{שלשה פיתגורית פרימיטיבית}
אם בנוסף המחלק המשותף המקסימלי של
$a,b,c$
שווה
$1$.

\begin{enumerate}
\item מתקיים כי אם
$v \in \RR^3$
שלשה פיתגורית, גם
$A_i v$
שלשה פיתגורית, וגם כי אם
$v \in \RR^3$
פרימיטיבית, גם
$A_i v$
פרימיטיבית.
הראו זאת עבור
$i = 1$.

\textbf{רמז:}
עבור פרימיטיביות, היעזרו במטריצה
$A_1^{-1}$.

\item הסיקו כי
$A_1^k v_0$
שלשה פיתגורית פרימיטיבית לכל
$k \in \NN$.

\item ידוע כי $1$ ערך עצמי של
$A_1$
ושל
$A_3$,
וכי
$-1$
ערך עצמי של
$A_2$.

הראו כי
$A_2, A_3$
לכסינות, ומיצאו מטריצה
$P \in \Mat_3\prs{\RR}$
הפיכה, ומטריצת ז'ורדדן
$J \in \Mat_3\prs{\RR}$
כך שמתקיים
$P A_1 P = J$.

\item נסמן
$A \coloneqq A_1$.
חשבו את
$A^k$
לכל
$k \in \mbb{N}$,
ומצאו שלשה פיתגורית פרימיטיבית מהצורה
$\pmat{a \\ 2025 \\ c}$.
האם יש בקבוצה
$\set{A^k v_0}{k \in \NN}$
וקטור מהצורה
$\pmat{a \\ 2026 \\ c}$?

\end{enumerate}
\end{exercise}

\end{document}