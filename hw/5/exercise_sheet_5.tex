\documentclass[a4paper,10pt,twoside,openany]{article}

\usepackage[lang=hebrew]{maths}
\usepackage{hebrewdoc}
\usepackage{stylish}
\usepackage{lipsum}
\let\bs\blacksquare

\setlength{\parindent}{0pt}

%%%%%%%%%%%%
% Styling %
%%%%%%%%%%%%

\usepackage{enumitem}

%%%%%%%%%%%%%
% Counters  %
%%%%%%%%%%%%%

\setcounter{section}{1}     
            
%BIBLIOGRAPHY
\usepackage[
backend=biber,
style=alphabetic,
]{biblatex}
\addbibresource{bibliography.bib} %Imports bibliography file

%%%%%%%%%%
% Title  %
%%%%%%%%%%
\title{
אלגברה ב' - גיליון תרגילי בית 5 \\
מכפלות פנימיות וניצבות
\\
\vspace{1cm}
\large{5.12.2025}
}
\date{}

\begin{document}
\maketitle

\begin{exercise}
עבור ההעתקות הבאות $f_i$, קיבעו האם $f_i$ מכפלה פנימית.

\begin{enumerate}
\item
\begin{align*}
f_1 \colon \mbb{R}^3 \times \mbb{R}^3 &\to \mbb{R} \\
\prs{\pmat{a\\b\\c}, \pmat{x\\y\\z}} &\mapsto ax + by + az
\end{align*}

\item
\begin{align*}
f_2 \colon \mbb{R}^3 \times \mbb{R}^3 &\to \mbb{R} \\
\prs{\pmat{a\\b\\c}, \pmat{x\\y\\z}} &\mapsto ax + by + cz + xz
\end{align*}

\item
\begin{align*}
f_3 \colon \Mat_n\prs{\mbb{C}} \times \Mat_n\prs{\mbb{C}} &\to \mbb{C} \\
\prs{A,B} &\mapsto \tr\prs{B^t A}
\end{align*}

\item
\begin{align*}
f_4 \colon \mbb{R}_{\leq n}\brs{x} \times \mbb{R}_{\leq n}\brs{x} &\to \mbb{R} \\
\prs{f,g} &\mapsto f\prs{0} g\prs{0} + \ldots + f\prs{n} g\prs{n}
\end{align*}

\item
\begin{align*}
f_5 \colon \mbb{C}_{\leq n}\brs{x} \times \mbb{C}_{\leq n}\brs{x} &\to \mbb{C} \\
\prs{f,g} &\mapsto f\prs{0} g\prs{0} + \ldots + f\prs{n} g\prs{n}
\end{align*}
\end{enumerate}
\end{exercise}

\begin{exercise}
היעזרו באי־שוויון קושי־שוורץ כדי להראות שמתקיים
\[\text{.} \forall x,y,z \in \mbb{R}_+ \colon x + y + z \leq 2 \prs{\frac{x^2}{y+z} + \frac{y^2}{x+z} + \frac{z^2}{x+y}}\]
\textbf{רמז:}
חישבו כיצד לפרש את אגף ימין בעזרת נורמה.
\end{exercise}

\begin{exercise}
יהי
$V$
מרחב מכפלה פנימית עם בסיס אורתונורמלי
$\prs{v_1, \ldots, v_n}$,
ויהיו
$u_1, \ldots, u_n \in V$
עבורם
\[\text{.} \forall i \in \brs{n} \colon \norm{v_i - u_i} < \frac{1}{\sqrt{n}}\]
הוכיחו כי
$\prs{u_1, \ldots, u_n}$
בסיס של
$V$.
\end{exercise}

\end{document}