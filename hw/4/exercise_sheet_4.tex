\documentclass[a4paper,10pt,twoside,openany]{article}

\usepackage[lang=hebrew]{maths}
\usepackage{hebrewdoc}
\usepackage{stylish}
\usepackage{lipsum}
\let\bs\blacksquare

\setlength{\parindent}{0pt}

%%%%%%%%%%%%
% Styling %
%%%%%%%%%%%%

\usepackage{enumitem}

%%%%%%%%%%%%%
% Counters  %
%%%%%%%%%%%%%

\setcounter{section}{1}     
            
%BIBLIOGRAPHY
\usepackage[
backend=biber,
style=alphabetic,
]{biblatex}
\addbibresource{bibliography.bib} %Imports bibliography file

%%%%%%%%%%
% Title  %
%%%%%%%%%%
\title{
אלגברה ב' - גיליון תרגילי בית 4 \\
מרחבים שמורים והטלות
\\
\vspace{1cm}
\large{28.11.2025}
}
\date{}

\begin{document}
\maketitle

\begin{exercise}
\begin{enumerate}
\item
יהי
$V$
מרחב וקטורי ממימד
$n \in \mbb{N}_+$
מעל שדה
$\mbb{F}$
ותהי
$T \in \endo_{\mbb{F}}\prs{V}$.
יהי
$W \leq V$
תת־מרחב
$T$%
־שמור ויהי
$U \leq V$
משלים ישר של
$W$
ב־%
$V$.
הראו כי
$\rest{T}{W} = T_{W,U}$.
\item
מצאו דוגמא עבור
$T \in \endo_{\mbb{F}}\prs{V}$,
$W \leq V$
כלשהו עם משלימים ישרים
$U_1, U_2 \leq V$,
כך ש־%
${T}_{W, U_1} \neq T_{W,U_2}$.
\end{enumerate}
\end{exercise}

\begin{exercise}
יהי
$V$
מרחב וקטורי סוף־מימדי מעל שדה
$\mbb{F}$
ותהיינה
$P,Q \in \End_\FF\prs{V}$
הטלות.

הראו כי אם
$PQ = QP$
אז
$PQ$
הטלה על
$\im\prs{P} \cap \im\prs{Q}$
במקביל ל־%
$\ker\prs{P} + \ker\prs{Q}$.
\end{exercise}

\begin{exercise}
יהי
$T \in \End_{\FF}\prs{V}$
אופרטור לכסין שערכיו העצמיים השונים הם
$\lambda_1, \ldots, \lambda_k$.
הראו כי התת־מרחבים ה־%
$T$
שמורים של
$V$
הם אלו מהצורה
$W_1 \oplus \ldots \oplus W_k$
עבור
$W_i \leq V_{\lambda_i}$.
\end{exercise}

\begin{exercise}
יהי
$T \in \End_\FF\prs{V}$
אופרטור לכסין עם ערכים עצמיים שונים
$\lambda_1, \ldots, \lambda_k$,
ולכל
$i \in \brs{k}$
תהי
$P_i \in \End_\FF\prs{V}$
ההטלה על
$V_i$
במקביל ל־%
$\oplus_{j \in \brs{k} \setminus \set{i}} V_j$.

\begin{enumerate}
\item לכל
$i \in \brs{k}$,
מיצאו  פולינום
$p_i$
עבורו
\[\text{.} p_i\prs{\lambda_i} = \delta_{i,j} \coloneqq \fcases{1 & i = j \\ 0 & \text{otherwise}}\]

\item
הראו כי
$p_i\prs{T} = P_i$
לכל
$i \in \brs{k}$.
\end{enumerate}
\end{exercise}

\end{document}