\documentclass[a4paper,10pt,twoside,openany]{article}

\usepackage[lang=hebrew]{maths}
\usepackage{hebrewdoc}
\usepackage{stylish}
\usepackage{lipsum}
\let\bs\blacksquare

\setlength{\parindent}{0pt}

%%%%%%%%%%%%
% Styling %
%%%%%%%%%%%%

\usepackage{enumitem}

%%%%%%%%%%%%%
% Counters  %
%%%%%%%%%%%%%

\setcounter{section}{1}     
            
%BIBLIOGRAPHY
\usepackage[
backend=biber,
style=alphabetic,
]{biblatex}
\addbibresource{bibliography.bib} %Imports bibliography file

%%%%%%%%%%
% Title  %
%%%%%%%%%%
\title{
אלגברה ב' - גיליון תרגילי בית 11 \\
תבניות בילינאריות וחוק האינרציה של סילבסטר
\\
\vspace{1cm}
\large{תאריך הגשה: 23.1.2026}
}
\date{}

\begin{document}

\maketitle

\begin{exercise}
תהי
$A \in \Mat_n\prs{\mbb{R}}$
סימטרית ויהיה
$m$
סכום הריבויים האלגבריים של הערכים העצמיים החיוביים של
$A$.

\begin{enumerate}
\item הראו כי יש תת־מרחב
$W \leq \mbb{R}^n$
ממימד
$m$
עבורו
$\trs{Aw, w} > 0$
לכל
$w \in W \setminus \set{0}$.

\item יהי
$W' \leq \mbb{R}^n$
תת־מרחב נוסף עבורו מתקיים
$\trs{Aw, w} > 0$
לכל
$w \in W' \setminus \set{0}$.
הראו כי
$\dim W' \leq m$.
\end{enumerate}
\end{exercise}

\begin{exercise}
יהי
$V = \Mat_2\prs{\mbb{R}}$.

\begin{enumerate}
    \item מיצאו מכפלה פנימית
    $f$
    על
    $V$
    כך ש־%
    \[B = \pmat{E_{1,1} + E_{1,2}, E_{1,2} + E_{2,1}, E_{2,1} + E_{2,2}, E_{2,2}}\]
    הוא בסיס אורתונורמלי לפי
    $f$.
    
    \item
    תהי
    \[g\prs{A,B} = \tr\prs{AB}\]
    תבנית בילינארית סימטרית על
    $V$.
    מיצאו בסיס
    $C$
    של
    $V$
    ומטריצה
    $S = \diag\prs{I_{n_+}, I_{n_-}, 0_{n_0}}$
    עבורם
    $\brs{g}_C = S$.
    
    \item
יהי
\[\tilde{B} = \prs{\pmat{1 & 1 \\ 0 & 0}, \pmat{-1 & 0 \\ 1 & 0}, \pmat{-1 & -1 \\ 1 & 1}, \pmat{0 & 0 \\ 0 & 1}}\]
ותהי
$\tilde{f}$
תבנית בילינארית עבורה
$\brs{\tilde{f}}_{\tilde{B}} = I$.
    
    מיצאו בסיס
    $D$
    של
    $V$
    עבורו
    $\brs{\tilde{f}}_D, \brs{g}_D$
    מטריצות אלכסוניות.
\end{enumerate}
\end{exercise}

\begin{exercise}
יהי
\begin{align*}
\text{,} V = \set{p \in \RR_{\leq n}\brs{x}}{\substack{p\prs{0} = p\prs{1} \\ p'\prs{0} = p'\prs{1}}}
\end{align*}
ותהי
\begin{align*}
\rho \colon V \times V &\to \RR \\
\text{.} \hphantom{lalala} \prs{f,g} &\mapsto \int_0^1 f\prs{x} g''\prs{x} \diff x
\end{align*}

\begin{enumerate}
\item הראו כי
$\rho$
תבנית בילינארית סימטרית על
$V$.

\textbf{רמז:}
היעזרו ב%
\href{https://en.wikipedia.org/wiki/Integration_by_parts}{אינטגרציה בחלקים}.

\item עבור
$n = 4$,
מיצאו בסיס
$B$
של
$V$
עבורו
$\brs{\rho}_B$
מטריצה מהצורה
\begin{align*}
\text{.} \diag\prs{I_{n_+}, - I_{n_-}, 0_{n_0}}
\end{align*}
\end{enumerate}
\end{exercise}

\end{document}