\documentclass[a4paper,10pt,twoside,openany]{article}

\usepackage[lang=hebrew]{maths}
\usepackage{hebrewdoc}
\usepackage{stylish}
\usepackage{lipsum}
\let\bs\blacksquare

\setlength{\parindent}{0pt}

%%%%%%%%%%%%
% Styling %
%%%%%%%%%%%%

\usepackage{enumitem}

%%%%%%%%%%%%%
% Counters  %
%%%%%%%%%%%%%

\setcounter{section}{0}     
            
%BIBLIOGRAPHY
\usepackage[
backend=biber,
style=alphabetic,
]{biblatex}
\addbibresource{bibliography.bib} %Imports bibliography file

%%%%%%%%%%
% Title  %
%%%%%%%%%%
\title{
אלגברה ב' - הצעה לפתרון בוחן אמצע
\\
סמסטר חורף 2025-2026
}
\date{}

\begin{document}
\maketitle

\begin{exercise}
\begin{enumerate}[label = (\alph*)]
\item
הראו שהנוסחה
\begin{align*}
\trs{\pmat{x_1 \\ x_2 \\ x_3}, \pmat{y_1 \\ y_2 \\ y_3}}
\coloneqq
x_1 \overline{y_1} +
\prs{x_1 + x_2}\overline{\prs{y_1 + y_2}} +
x_3 \overline{y_3}
\end{align*}
נותנת מכפלה פנימית על המרחב הוקטורי המרוכב
$\CC^3$.

\item
מיצאו בסיס אורתונורמלי למרחב המכפלה הפנימית שמתקבל.

\item
נתון האופרטור
$T \colon \CC^3 \to \CC^3$
המוגדר על ידי
\begin{align*}
\text{.} T\pmat{x_1 \\ x_2 \\ x_3} = \pmat{x_1 \\ 2 x_2 \\ 3 x_3}
\end{align*}
הראו ש־%
$T$
איננו אופרטור נורמלי (ביחס למכפלה הפנימית מהסעיפים הקודמים).
\end{enumerate}
\end{exercise}

\begin{solution}
\begin{enumerate}[label = (\alph*)]
\item
נראה כי ההעתקה הנתונה הינה מכפלה פנימית על ידי כך שנראה שהיא בי־סמי־לינארית, הרמיטית, וחיובית.

\begin{description}
\item[חיובית:]
חיוביות אומרת כי
$\trs{x,x} \geq 0$
לכל
$x \in \CC^3$
וגם כי
$x = 0$
אם
$\trs{x, x} = 0$.

יהי
$x = \pmat{x_1, x_2, x_3} \in \CC^3$.
מתקיים
\begin{align*}
\trs{x, x} &= x_1 \overline{x_1} + \prs{x_1 + x_2} \overline{\prs{x_1 + x_2}} + x_3 \overline{x_3}
\\&=
\abs{x_1}^2 + \abs{x_1 + x_2}^2 + \abs{x_3}^2
\end{align*}
וכיוון שמתקיים
$\abs{z} \geq 0$
לכל
$z \in \CC$
נקבל כי
$\trs{x,x} \geq 0$.

אם
$\trs{x, x} = 0$,
נקבל כי
\[\text{,} \abs{x_1}^2 + \abs{x_1 + x_2}^2 + \abs{x_3}^2 = 0\]
וכיוון שכל אחד מהגורמים הינו אי־שלילי, נקבל כי
\begin{align*}
\abs{x_1}^2 &= 0 \\
\abs{x_1 + x_2}^2 &= 0 \\
\abs{x_3}^2 &= 0
\end{align*}
ואז גם
\begin{align*}
x_1 &= 0 \\
x_1 + x_2 &= 0 \\
\text{.} x_3 &= 0
\end{align*}
מהצבה של
$x_1 = 0$
במשוואה השנייה נקבל כי
$x_2 = 0$,
ולכן
$x = 0$.

\item[הרמיטיות:]
הרמיטיות משמעותה שלכל
$x,y \in \CC^3$
מתקיים
$\trs{y,x} = \overline{\trs{x,y}}$.

יהיו
$x,y \in \CC^3$.
מתקיים
\begin{align*}
\trs{y,x} &= \overline{\overline{\trs{y,x}}}
\\&=
\overline{\overline{y_1 \overline{x_1} + \prs{y_1 + y_2} \overline{x_1 + x_2} + y_3 \overline{x_3}}}
\\&=
\overline{\overline{y_1} \overline{\overline{x_1}} + \overline{\prs{y_1 + y_2}} \overline{\overline{\prs{x_1 + x_2}}} + \overline{y_3} \overline{\overline{x_3}}}
\\&=
\overline{ \overline{y_1} x_1 + \overline{ \prs{y_1 + y_2} } \prs{x_1 + x_2} + \overline{y_3} x_3}
\\&= \overline{\trs{x,y}}
\end{align*}
כאשר בשוויון השני השתמשנו בכך שלכל
$z_1, z_2 \in \CC$
מתקיים
$\overline{z_1 + z_2} = \overline{z_1} + \overline{z_2}$
ובשלישי בכך שלכל
$z \in \CC$
מתקיים
$\overline{\overline{z}} = z$.

\item[בי־סמי־לינאריות]:
בי־סמי־לינאריות משמעותה לינאריות ברכיב הראשון וסמי־לינאריות ברכיב השני.

נראה ראשית לינאריות ברכיב הראשון. יהי
$\alpha \in \CC$
ויהיו
$x,y,z \in \CC^3$.
מתקיים
\begin{align*}
\trs{\alpha x + z, y} &= \prs{\alpha x_1 + z_1} \overline{y_1} + \prs{\alpha x_1 + z_1 + \alpha x_2 + z_2} \overline{\prs{y_1 + y_2}} + \prs{\alpha x_3 + z_3} \overline{y_3}
\\&= \alpha x_1 \overline{y_1} + z_1 \overline{y_1} + \alpha \prs{x_1 + x_2} \overline{\prs{y_1 + y_2}} + \prs{z_1 + z_2}\overline{\prs{y_1 + y_2}} + \alpha x_3 \overline{y_3} + z_3 \overline{y_3}
\\&=
\alpha \prs{x_1 \overline{y_1} + \prs{x_1 + x_2} \overline{\prs{y_1 + y_2}} + x_3 \overline{y_3}} + \prs{z_1 \overline{y_1} + \prs{z_1 + z_2} \overline{\prs{y_1 + y_2}} + z_3 \overline{y_3}}
\\&= \alpha \trs{x, y} + \trs{z, y}
\end{align*}
כנדרש.

נראה כעת סמי־לינאריות ברכיב השני. יהי
$\alpha \in \CC$
ויהיו
$x,y,z \in \CC$.
יש להראות כי
\begin{align*}
\text{.} \trs{x, \alpha y + z} = \bar{\alpha} \trs{x, y} + \trs{x,z}
\end{align*}
אכן,
\begin{align*}
\trs{x, \alpha y + z} &= \overline{\trs{\alpha y + z, x}}
\\&= \overline{\alpha \trs{y, x} + \trs{z,x}}
\\&= \overline{\alpha} \overline{\trs{y,x}} + \overline{\trs{z,x}}
\\&= \overline{\alpha} \trs{x,y} + \trs{x,z}
\end{align*}
כאשר בשוויון הראשון והאחרון השתמשנו בהרמיטיות של
$\trs{\cdot, \cdot}$
(שהוכחנו קודם), וכאשר בשוויון השני השתמשנו בלינאריות ברכיב הראשון (שגם הוכחנו קודם).
\end{description}

\item

יהי
$\St = \prs{e_1, e_2, e_3}$
הבסיס הסטנדרטי של
$\CC^3$.
נבצע את תהליך גרם־שמידט על
$\St$,
שייתן לנו בסיס
$B$
אורתונורמלי.

נסמן בכל שלב ב־%
$w_i$
את הוקטור המתקבל מ־%
$e_i$
לאחר חיסור ההטלה על המרחב הנפרש על ידי הוקטורים שכבר בנינו, וב־%
$v_i$
את הנרמול שלו, שיהיה הוקטור ה־%
$i$
בבסיס
$B$.

תחילה, יש לנרמל את הוקטור הראשון,
$v_1 = \frac{e_1}{\norm{e_1}}$.
מתקיים
\[\norm{e_1}^2 = \trs{e_1, e_1} = 1 \cdot 1 + \prs{1 + 0}\prs{1+0} + 0 \cdot 0 = 2\]
ולכן
$\norm{e_1} = \sqrt{2}$
ונקבל כי
$v_1 = \frac{1}{\sqrt{2}} e_1$.

כעת,
\begin{align*}
w_2 &= e_2 - \trs{e_2, v_1} v_1
\\&= e_2 - \trs{e_2, \frac{1}{\sqrt{2}} e_1} \cdot \frac{1}{\sqrt{2}} e_1
\\&= e_2 - \frac{1}{2} \trs{e_2, e_1} e_1
\\&= e_2 - \frac{1}{2} \prs{0 + 1 + 0} e_1
\\&= e_2 - \frac{1}{2} e_1
\end{align*}
ואז
\begin{align*}
\norm{w_2}^2 = \trs{w_2, w_2} = \prs{-\frac{1}{2}}^2 + \prs{\frac{1}{2}}^2 + 0 = \frac{1}{2}
\end{align*}
ולכן
\begin{align*}
\text{.} v_2 = \frac{w_2}{\norm{w_2}} = \sqrt{2} \prs{e_2 - \frac{1}{2} e_1}
\end{align*}

לבסוף,
\begin{align*}
w_3 &= e_3 - \trs{e_3, v_2} v_2 - \trs{e_3, v_1} v_1 = e_3
\end{align*}
כיוון שמתקיים
\begin{align*}
\trs{e_3, v_2} &= \sqrt{2} \prs{0 \cdot \prs{- \frac{1}{2}} + 0 \cdot 1 + 1 \cdot 0} = 0 \\
\text{.} \trs{e_3, v_1} &= \frac{1}{\sqrt{2}} \prs{0 \cdot 1 + 0 \cdot 0 + 1 \cdot 0} = 0
\end{align*}
אז
\begin{align*}
\norm{w_3}^2 = \trs{w_3, w_3} = 0 \cdot 0 + 0 \cdot 0 + 1 \cdot 1 = 1
\end{align*}
ונקבל
\begin{align*}
\text{.} v_3 = \frac{w_3}{\norm{w_3}} = w_3 = e_3
\end{align*}

בסך הכל, נקבל בסיס אורתונורמלי
\begin{align*}
B &= \prs{\frac{1}{\sqrt{2}} e_1, \sqrt{2} \prs{e_2 - \frac{1}{2} e_1}, e_3}
\\&= \prs{\pmat{\frac{1}{\sqrt{2}} \\ 0 \\ 0}, \pmat{\frac{-\sqrt{2}}{2} \\ \sqrt{2} \\ 0}, \pmat{0 \\ 0 \\ 1}}
כנדרש.
\end{align*}

\item
נציע שתי דרכי פתרון.

\begin{description}
\item[דרך 1:]
ראינו כי עבור אופרטור נורמלי מעל
$\CC$,
וקטורים עצמיים של ערכים עצמיים שונים הינם ניצבים זה לזה.
נשים לב כי
$T\prs{e_1} = e_1$
וכי
$T\prs{e_2} = 2 e_2$,
לכן
$e_1, e_2$
וקטורים עצמיים של ערכים עצמיים שונים. אך מתקיים
\[\text{,} \trs{e_1, e_2} = 1 \cdot 0 + \prs{\prs{1 + 0} \cdot \prs{0 + 1}} + 0 \cdot 0 = 1 \neq 0\]
ולכן הם אינם ניצבים.
לכן לא יתכן כי
$T$
נורמלי.

\item[דרך 2:]
ממשפט, עבור אופרטור
$S$
על מרחב מכפלה פנימית עם בסיס אורתונורמלי
$C$,
מתקיים
\begin{align*}
\text{.} \brs{S^*}_C = \brs{S}_C^*
\end{align*}
לכן,
כאשר
$B$
הבסיס שמצאנו בסעיף הקודם, מתקיים
\begin{align*}
\text{.} \brs{T^*}_B = \brs{T}_B^*
\end{align*}
מתקיים תמיד
$\brs{S_1 S_2}_C = \brs{S_1}_C \brs{S_2}_C$
ולכן
\begin{align*}
\brs{T^* T}_B &= \brs{T}_B^* \brs{T}_B \\
\text{.} \brs{T T^*}_B &= \brs{T}_B \brs{T}_B^*
\end{align*}
לכן, אם נראה כי
$\brs{T}_B, \brs{T}_B^*$
אינן מתחלפות, נקבל כי
$T^* T \neq T T^*$,
ולכן
$T$
אינה נורמלית.

נחשב את
$\brs{T}_B$
לפי הגדרת מטריצה מייצגת,
\begin{align*}
\text{.} \brs{T}_B &= \pmat{\vert & \vert & \vert \\ \brs{T\prs{v_1}}_B & \brs{T\prs{v_2}}_B & \brs{T\prs{v_3}}_B \\ \vert & \vert & \vert}
\end{align*}
מתקיים
\begin{align*}
T\prs{v_1} &= T\prs{\frac{1}{\sqrt{2}} e_1} = \frac{1}{\sqrt{2}} T\prs{e_1} = \frac{1}{\sqrt{2}} e_1 = v_1 \\
T\prs{v_2} &= T\prs{\sqrt{2} \pmat{- \frac{1}{2} \\ 1 \\ 0}}
\\&= \sqrt{2} \pmat{-\frac{1}{2} \\ 2 \\ 0}
\\&= 2 \sqrt{2} \pmat{- \frac{1}{2} \\ 1 \\ 0} + \pmat{\frac{\sqrt{2}}{2} \\ 0 \\ 0}
\\&= 2 v_2 + \pmat{\frac{1}{\sqrt{2}} \\ 0 \\ 0}
\\&= 2 v_2 + v_1 \\
\text{,} T\prs{v_3} &= T\prs{e_3} = 3 e_3 = 3 v_3
\end{align*}
ולכן
\begin{align*}
\text{.} \brs{T}_B &= \pmat{1 & 1 & 0 \\ 0 & 2 & 0 \\ 0 & 0 & 3}
\end{align*}
אז
\begin{align*}
\brs{T}_B^* &= \overline{\brs{T}_B}^t = \pmat{1 & 0 & 0 \\ 1 & 2 & 0 \\ 0 & 0 & 3}
\end{align*}
ומתקיים
\begin{align*}
\text{,} \brs{T}_B \brs{T}_B^* = \pmat{2 & 2 & 0 \\ 2 & 4 & 0 \\ 0 & 0 & 9} \neq \pmat{1 & 1 & 0 \\ 1 & 5 & 0 \\ 0 & 0 & 9} = \brs{T}_B^* \brs{T}_B
\end{align*}
כנדרש.

\end{description}
\end{enumerate}
\end{solution}

\newpage

\begin{exercise}
נתון מרחב מכפלה פנימית
$V$,
ונתון אופרטור הטלה (לא בהכרח אורתוגונלית)
$P \colon V \to V$.

נסמן את התת־מרחב
$W \coloneqq \im\prs{P}$
של
$V$,
וב־%
$P_W \colon V \to V$
את אופרטור ההטלה האורתוגונלית על
$W$.

הראו שמתקיים
$P = P_W$
אם ורק אם מתקיים
$P = P^*$.
\end{exercise}

\begin{solution}
נראה גרירה דו־כיוונית.

\begin{itemize}
\item נניח כי
$P = P_W$.

יהיו
\[B_W \coloneqq \prs{w_1, \ldots, w_m}\]
וגם
\[B_{W^\perp} \coloneqq \prs{u_1, \ldots, u_k}\]
בסיסים אורתונורמליים
$W$
ושל
$W^\perp$
בהתאמה. קיימים בסיסים כאלו כיוון שלכל מרחב מכפלה פנימית קיים בסיס אורתונורמלי.

ראינו בהרצאה כי לכל
$U \leq V$,
מתקיים
$V = U \oplus U^\perp$.
לכן
$V = W \oplus W^\perp$.
ראינו בתרגול שעבור בסיסים
$B_1, B_2$
של
$U_1, U_2 \leq V$
כך ש־%
$V = U_1 \oplus U_2$,
מתקיים כי
$B_1 * B_2$
בסיס של
$V$.
לכן
$B \coloneqq B_W * B_{W^\perp}$
בסיס של
$V$.
הוא גם אורתונורמלי, כיוון שכל הוקטורים בו מאורך
$1$,
וכולם ניצבים שכן
\begin{align*}
\trs{w_i, w_j} &= \delta_{i,j} \\
\trs{u_i, u_j} &= \delta_{i,j} \\
\trs{w_i, u_j} &= 0
\end{align*}
כאשר שני השוויונות הראשונים מתקיימים כי
$B_W, B_{W^\perp}$
בסיסים אורתונורמליים, והשוויון השלישי כי כל וקטור ב־%
$W^\perp$
ניצב לכל וקטור ב־%
$W$.

ההטלה האורתוגונלית על
$W$
היא ההטלה על
$W$
במקביל ל־%
$W^\perp$,
ולכן
\begin{align*}
\text{.} \brs{P}_B = \pmat{I_m & 0 \\ 0 & 0_k}
\end{align*}

ראינו שכאשר
$C$
בסיס אורתונורמלי, ו־%
$T$
אופרטור על
$V$,
מתקיים
$\brs{T^*}_C = \brs{T}_C^*$,
לכן נקבל כי
\begin{align*}
\brs{P^*}_B = \brs{P}_B^* = \overline{\pmat{I_m & 0 \\ 0 & 0_k}}^t = \pmat{I_m & 0 \\ 0 & 0_k} = \brs{P}_B
\end{align*}
ולכן
$P^* = P$.

\item נניח כי
$P = P^*$.
האופרטור
$P_W$
הינו ההטלה על
$W = \im\prs{P}$
במקביל ל־%
$\im\prs{P}^\perp$,
והאופרטור
$P$
הינו ההטלה על
$\im\prs{P}$
במקביל ל־%
$\ker\prs{P}$,
לכן יש להראות בעצם שמתקיים
\[\text{.} \ker\prs{P} = \im\prs{P}^{\perp}\]
נכתוב שתי דרכים להראות זאת.

\begin{description}
\item[דרך 1:]
נראה תחילה כי
\[\text{.} \ker\prs{P} \subseteq \im\prs{P}^{\perp}\]
יהי
$v \in \ker\prs{P} = \ker\prs{P^*}$,
ויהי
$w \in \im\prs{P}$.
אם נראה שמתקיים
$\trs{v,w} = 0$
נקבל כי
$v \in W^\perp = \im\prs{P}^\perp$
ולכן את ההכלה.
כיוון ש־%
$w \in \im\prs{P}$,
קיים
$u \in V$
עבורו
$w = P\prs{u}$.
אכן
\begin{align*}
\text{.} \trs{v,w} &= \trs{v, P\prs{u}} = \trs{P^*\prs{v}, u} = \trs{0, u} = 0
\end{align*}

מצד שני, ראינו בהרצאה שלכל תת־מרחב
$W' \leq V$
מתקיים
\[\text{,} V = W' \oplus \prs{W'}^\perp\]
ולכן מתקיים
\begin{align*}
V = W \oplus W^\perp
\end{align*}
ואז
\begin{align*}
\dim\prs{V} = \dim\prs{W} \oplus \dim\prs{W^\perp}
\end{align*}
כיוון שהמימד של סכום ישר של מרחבים וקטוריים הוא סכום המימדים.
לכן
\begin{align*}
\text{,} \dim\prs{\im\prs{P}^\perp} = \dim\prs{V} - \dim\prs{\im\prs{P}} = \dim\prs{\ker\prs{P}}
\end{align*}
כאשר השוויון האחרון מתקיים לפי משפט המימדים עבור גרעין ותמונה.

קיבלנו כי
\[\text{,} \ker\prs{P} \subseteq \im\prs{P}^{\perp}\]
וכי למרחבים
$\ker\prs{P}$
ו־%
$\im\prs{P}^{\perp}$
אותו מימד, ולכן יש שוויון, כנדרש.

\item[דרך 2:]

כיוון ש־%
$P = P^*$,
לפי משפט הפירוק הספקטרלי, קיים בסיס אורתונורמלי
$B$
המלכסן את
$P$
(מעל
$\RR$,
השוויון
$P = P^*$
הוא בדיוק התנאי לכך, ומעל
$\CC$,
מתקיים
$P^* P = P P = P P^*$
כיוון ש־%
$P = P^*$,
וזה התנאי לכך).
כיוון ש־%
$P$
הטלה, הערכים העצמיים היחידים האפשריים עבורה הם
$0,1$.
לכן, ניתן לכתוב
\begin{align*}
\brs{P}_B = \pmat{I_m & 0_{m \times k} \\ 0_{k \times m} & 0_{k \times k}}
\end{align*}
כאשר יתכן כי
$m$
או
$k$
שווים לאפס.

נסמן
\[B = \prs{v_1, \ldots, v_n}\]
ואז
\[\text{.} P\prs{v_i} = \fcases{v_i & i \leq m \\ 0 & i > m}\]
נקבל כי
\begin{align*}
\Span\prs{v_1, \ldots, v_m} &= \im\prs{P} \\
\text{.} \Span\prs{v_{m+1}, \ldots, v_n} &= \ker\prs{P}
\end{align*}
אבל,
$B$
אורתונורמלי ולכן
\begin{align*}
\text{,} \Span\prs{v_{m+1}, \ldots, v_n} = \Span\prs{v_1, \ldots, v_m}^\perp
\end{align*}
ונקבל בסך הכל כי
\[\text{,} \ker\prs{P} = \im\prs{P}^\perp\]
כנדרש.
\end{description}
\end{itemize}
\end{solution}

\newpage

\begin{exercise}
נתון מרחב מכפלה פנימית
\textbf{ממשי}
$V$,
ונתונה העתקה אורתוגונלית
$T \colon V \to V$
שהיא לכסינה.

(לא נתון ש־%
$T$
לכסינה אורתוגונלית!)

\begin{enumerate}[label = (\alph*)]
\item
הראו שהערכים העצמיים האפשריים היחידים של
$T$
הם
$1$
ו־%
$-1$.
\item
הראו ש־%
$T$
בהכרח לכסינה אורתוגונלית.
\end{enumerate}
\end{exercise}

\begin{solution}
\begin{enumerate}[label = (\alph*)]
\item
ראינו בתרגול כי העתקה אורתוגונלית שומרת על הנורמה. כלומר, אצלנו,
$\norm{T\prs{v}} = \norm{v}$
לכל
$v \in V$.
לכן, אם
$\lambda$
ערך עצמי עם וקטור עצמי
$v$
מתקיים
\begin{align*}
\text{.} \norm{v} = \norm{T\prs{v}} = \norm{\lambda v} = \abs{\lambda} \norm{v}
\end{align*}
כיוון שוקטור עצמי בהכרח שונה מאפס, וכיוון שהנורמה של וקטור שונה מאפס הינה חיובית, נוכל לחלק את שני האגפים ב־%
$\norm{v}$
ולקבל כי
$\abs{\lambda} = 1$,
ולכן
$\lambda \in \set{\pm 1}$.

\item
כיוון שנתון כי
$T$
לכסינה, וכיוון שהראנו שהערכים העצמיים של $T$ נמצאים בקבוצה $\set{\pm 1}$, קיים בסיס
$B$
עבורו
\begin{align*}
\text{,} \brs{T}_B &= \pmat{I_m & 0_{m \times k} \\ 0_{k \times m} & -I_k}
\end{align*}
כאשר יתכן כי
$m$
או
$k$
שווים לאפס.
אז
\begin{align*}
\brs{T^{-1}}_B &= \brs{T}_B^{-1}
\\&=
\pmat{I_m & 0_{m \times k} \\ 0_{k \times m} & -I_k}^{-1}
\\&=
\pmat{I_m & 0_{m \times k} \\ 0_{k \times m} & -I_k}
\\&= \brs{T}_B
\end{align*}
ולכן
$T^{-1} = T$.
מצד שני, נתון כי
$T$
אורתוגונלית, מה שאומר לפי ההגדרה כי
$T^{-1} = T^*$.

נקבל כי
$T^* = T$,
ומשפט הפירוק הספקטרלי מעל
$\RR$
אומר בדיוק שזה שקול לכך ש־%
$T$
לכסינה אורתוגונלית.
\end{enumerate}
\end{solution}

\end{document}