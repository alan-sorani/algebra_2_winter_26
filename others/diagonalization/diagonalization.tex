\documentclass[a4paper,10pt,twoside,openany]{article}

\usepackage[lang=hebrew]{maths}
\usepackage{polynom}
\usepackage{hebrewdoc}
\usepackage{stylish}
\usepackage{lipsum}
\let\bs\blacksquare

\setlength{\parindent}{0pt}

%%%%%%%%%%%%
% Styling %
%%%%%%%%%%%%

\usepackage{enumitem}

%%%%%%%%%%%%%
% Counters  %
%%%%%%%%%%%%%

\setcounter{section}{0}     
            
%BIBLIOGRAPHY
\usepackage[
backend=biber,
style=alphabetic,
]{biblatex}
\addbibresource{bibliography.bib} %Imports bibliography file

\title{
אלגברה ב' (01040168) - אביב 2026
\\
תרגילים על לכסון
\\
אלן סורני
\\
הרשימות עודכנו לאחרונה בתאריך ה־%
\today
}
\date{}

\begin{document}
\maketitle

\section{ערכים עצמיים ולכסון}

\begin{definition}[ערך עצמי, ווקטור עצמי]
יהי
$T \in \End_{\FF}\prs{V}$.
ערך
$\lambda \in \mbb{F}$
עבורו קיים
$v \in V \setminus \set{0}$
כך ש־%
$T\prs{v} = \lambda v$
נקרא
\emph{ערך עצמי של
$T$},
ווקטור
$v$
המקיים שוויון זה נקרא
\emph{וקטור עצמי של
$T$
עבור הערך העצמי
$\lambda$}.
\end{definition}

\begin{definition}[גרעין של אופרטור לינארי]
יהי
$T \in \End_{\FF}\prs{V}$.
נגדיר את
\emph{הגרעין של
$T$}
בתור
\[\text{.} \ker\prs{T} \coloneqq \set{v \in V}{T\prs{v} = 0}\]
\end{definition}

\begin{proposition} \label{proposition:eigenvalue-conditions}
התנאים הבאים שקולים.

\begin{enumerate}
\item $\lambda \in \mbb{F}$
ערך עצמי של
$T$.
\item האופרטור
$T - \lambda \id_V$
אינו הפיך.
\item $\ker\prs{T - \lambda \id_V} \neq \set{0}$.
\end{enumerate}
\end{proposition}

\begin{definition}[מרחב עצמי]
יהי
$T \in \End_{\FF}\prs{V}$ עם ערך עצמי
$\lambda \in \FF$.
\emph{המרחב העצמי של
$T$
עם ערך עצמי
$\lambda$}
הוא
$\ker\prs{T - \lambda \id_V}$.
\end{definition}

\begin{definition}[ריבוי גיאומטרי]
יהי
$T \in \End_{\FF}\prs{V}$ עם ערך עצמי
$\lambda \in \FF$.
נגדיר את
\emph{הריבוי הגיאומטרי של
$\lambda$
כערך עצמי של
$T$}
בתור
\begin{align*}
\text{.} r^T_g\prs{\lambda} = \dim_{\FF}\prs{\ker\prs{T - \lambda \id_V}}
\end{align*}
כאשר האופרטור
$T$
ברור מההקשר, נסמן לעתים
$r_g\prs{\lambda}$
במקום
$r^T_g\prs{\lambda}$.
\end{definition}

\begin{proposition} \label{proposition:sum-multiplicities}
יהי
$V$
מרחב וקטורי סוף־מימדי מעל שדה
$\FF$,
ויהי
$T \in \End_{\FF}\prs{V}$
עם ערכים עצמיים שונים
$\lambda_1, \ldots, \lambda_k$.
אז:
\begin{enumerate}
\item $r^T_g\prs{\lambda_i} \geq 1$
לכל
$i \in \brs{k}$.
\item \begin{align*}
\text{.} \sum_{i \in \brs{k}} r^T_g\prs{\lambda_i} \leq \dim_{\FF}\prs{V}
\end{align*}
\end{enumerate}
\end{proposition}

\begin{exercise}
יהי
$V = \Mat_2\prs{\RR}$.
מיצאו את הערכים העצמיים של
\begin{align*}
T \colon V &\to V \\
\text{.} \hphantom{lalala} A &\mapsto A^t
\end{align*}
לכל ערך עצמי, מיצאו את המרחב העצמי המתאים ואת הריבוי הגיאומטרי.
\end{exercise}

\begin{solution}
מתקיים
\[\text{.} T\pmat{a & b \\ c & d} = \pmat{a & c \\ b & d}\]
אם
$\lambda \in \FF$
ערך עצמי של
$T$,
נקבל כי קיימים
$a,b,c,d \in \FF$
לא כולם
$0$
עבורם
\begin{align} \label{eq:transpose-eigenvalues}
\text{.} \pmat{a & c \\ b & d} = \lambda \pmat{a & b \\ c & d}
\end{align}

אם
$\lambda = 1$
זה שקול לכך ש־%
$b = c$,
ולכן
$\lambda = 1$
ערך עצמי של
$T$
עם מרחב עצמי
\[\text{,} \set{\pmat{a & b \\ b & d}}{a,b,d \in \FF} = \Span\prs{E_{1,1}, E_{1,2} + E_{2,1}, E_{2,2}}\]
וריבוי גיאומטרי
$r_g\prs{1} = 3$.
אם
$\lambda \neq 1$,
חייב להתקיים
$a = d = 0$
עבור השוויון
\eqref{eq:transpose-eigenvalues},
וכן חייב להתקיים
$c = \lambda b$
וגם
$b = \lambda c$.
הצבה של השוויון האחרון נותנת לנו כי
$c = \lambda^2 c$,
וכיוון ש־%
$\lambda \neq 1$
נקבל כי
$\lambda = -1$.
אז
$-1$
ערך עצמי של
$T$
עם מרחב עצמי
\[\text{,} \set{\pmat{0 & b \\ -b & 0}}{b \in \FF} = \Span\prs{E_{1,2} - E_{2,1}}\]
וריבוי גיאומטרי
$r_g\prs{-1} = 1$.

קיבלנו כי
$r_g\prs{1} = 3$
וכי
$r_g\prs{-1} = 1$,
כאשר
$\dim_\RR\prs{V} = 4$,
ולכן לפי טענה
\ref{proposition:sum-multiplicities}
אין ערכים עצמיים נוספים.
\end{solution}

\begin{definition}[אופרטור לכסין, מטריצה אלכסונית, ובסיס מלכסן]
יהי
$V$
מרחב וקטורי סוף־מימדי מעל שדה
$\FF$,
ויהי
$T \in \End_{\FF}\prs{V}$.
האופרטור
$T$
נקרא
\emph{לכסין}
אם קיים בסיס
$B$
של
$V$
וקיימות
$\lambda_1, \ldots, \lambda_n \in \FF$
עבורם
\[\text{.} \brs{T}_B = \diag\prs{\lambda_1, \ldots, \lambda_n} \coloneqq
\pmat{
\lambda_1 & 0 & \cdots & 0 & 0 \\
0 & \lambda_2 & \ddots & \ddots & 0 \\
\vdots & \ddots & \ddots & \ddots & \vdots \\
0 & \ddots & \ddots & \lambda_{n-1} & 0 \\
0 & 0 & \cdots & 0 & \lambda_n
}\]
מטריצה מהצורה
$\diag\prs{\lambda_1, \ldots, \lambda_n}$
נקראת
\emph{מטריצה אלכסונית},
ובסיס
$B$
כנ"ל נקרא
\emph{בסיס מלכסן עבור
$T$}.
\end{definition}

\begin{exercise}
מיצאו מרחב וקטורי
$V$
סוף־מימדי מעל
$\CC$,
ואופרטור
$T \in \End_\FF\prs{V}$
כך שסכום הריבויים הגיאומטריים של הערכים העצמיים של
$T$
קטן מ־%
$\dim_\FF\prs{V}$.
\end{exercise}

\begin{solution}
נרצה למשל אופרטור בעל ערך עצמי יחיד
$0$
אך שהריבוי הגיאומטרי שלו שווה
$1$.

ניקח את האופרטור
$T\pmat{x \\ y} = \pmat{y \\ 0}$
על
$\RR^2$.

כעת,
$T\pmat{x \\ y} = \lambda \pmat{x \\ y}$
אם ורק אם
$\lambda x = y$
וגם
$\lambda y = 0$,
וכאשר
$\lambda \neq 0$
זה גורר בהכרח
$x = y = 0$.
לכן לא יתכנו ערכים עצמיים שונים מאפס.

מתקיים
$T\pmat{x \\ y} = 0 \cdot \pmat{x \\ y}$
ואם ורק אם
$y = 0$,
לכן המרחב העצמי של
$0$
כערך עצמי של
$T$
הוא
$\Span\prs{\pmat{1 \\ 0}}$,
ולכן הריבוי הגיאומטרי הוא
$1$,
ולא
$\dim_\RR\prs{\RR^2} = 2$.

\end{solution}

\begin{definition}[מטריצה לכסינה]
מטריצה
$A \in \Mat_n\prs{\FF}$
נקראת
\emph{לכסינה}
אם האופרטור
\begin{align*}
L_A \colon \FF^n &\to \FF^n \\
v &\mapsto Av
\end{align*}
הינו לכסין.
\end{definition}

\begin{exercise}
מיצאו מטריצה
$A \in \Mat_n\prs{\RR}$
שאינה לכסינה, אך כך שהמטריצה
$A^{\CC} \in \Mat_n\prs{\CC}$
המרוכבת בעלת אותם מקדמים כמו
$A$
הינה לכסינה.
\end{exercise}

\begin{solution}
נחפש מטריצה מרוכבת עם מקדמים ממשיים כך שיש לה ערכים עצמיים שאינם ממשיים.
ידוע כי מכפלת הערכים העצמיים של מטריצה
$\pmat{a & b \\ c & d}$
היא
$ad - bc$.

לכן ניקח למשל את המטריצה
\[\text{,} A = \pmat{0 & 1 \\ -1 & 0} \in \Mat_2\prs{\RR}\]
שמכפלת הערכים העצמיים שלה היא
$1$
אך סכומם
$0$,
מה שלא יתכן עבור שני מספרים ממשיים.

מתקיים
\[\text{.} A \pmat{x \\ y} = \pmat{y \\ -x}\]
נחפש ערכים עצמיים
$\lambda \in \CC$,
כלומר ערכים עבורם
\[\text{.} \lambda \pmat{x \\ y} = \pmat{y \\ -x} \]
שוויון זה שקול למערכת המשוואות
\begin{align*}
\lambda x = y \\
\lambda y = -x
\end{align*}
והצבה של השוויון הראשון בשני נותנת את השוויון
$\lambda^2 x = -x$.
אז
$\lambda^2 = -1$,
ולכן אם
$\lambda$
ערך עצמי של
$A$
מתקיים בהכרח כי
$\lambda \in \set{\pm i}$.
לכן למטריצה
$A$
אין ערכים עצמיים, ולכן היא אינה לכסינה.

נראה כעת כי המטריצה
\[A^\CC = \pmat{0 & 1 \\ -1 & 0} \in \Mat_2\prs{\CC}\]
אכן לכסינה.

מתקיים
\begin{align*}
A^\CC - i\id_V = \pmat{-i & 1 \\ -1 & -i}
\end{align*}
וכאן השורה השנייה שווה לשורה הראשונה כפול
$-i$.
לכן,
\[\prs{A^\CC - \prs{-i}\id_V} \prs{e_1 + i e_2} = 0\]
ואז
$e_1 + i e_2$
וקטור עצמי של $A$ עבור הערך העצמי
$i$.

כמו כן,
\begin{align*}
A^\CC - \prs{-i}\id_V = \pmat{i & 1 \\ -1 & i}
\end{align*}
וכעת השורה השנייה שווה לראשונה כפול
$i$,
ואז
$e_1 - i e_2$
וקטור עצמי של
$A$
עבור הערך העצמי
$-i$.

נקבל כי
$B = \prs{e_1 + i e_2, e_1 - i e_2}$
בסיס עבורו
\begin{align*}
L_{A^\CC}\prs{e_1 + i e_2} &= i \prs{e_1 + i e_2} \\
\text{,} L_{A^\CC}\prs{e_1 - i e_2} &= -i \prs{e_1 - i e_2}
\end{align*}
ולכן
\begin{align*}
\brs{L_A}_B = \pmat{i & 0 \\ 0 & -i}
\end{align*}
מטריצה אלכסונית.
\end{solution}

\printbibliography
\end{document}
