\documentclass[a4paper,10pt,twoside,openany]{article}

\usepackage[lang=hebrew]{maths}
\usepackage{polynom}
\usepackage{hebrewdoc}
\usepackage{stylish}
\usepackage{lipsum}
\let\bs\blacksquare

\setlength{\parindent}{0pt}

%%%%%%%%%%%%
% Styling %
%%%%%%%%%%%%

\usepackage{enumitem}

%%%%%%%%%%%%%
% Counters  %
%%%%%%%%%%%%%

\setcounter{section}{0}     
            
%BIBLIOGRAPHY
\usepackage[
backend=biber,
style=alphabetic,
]{biblatex}
\addbibresource{bibliography.bib} %Imports bibliography file

\title{
אלגברה ב' (01040168) - חורף 2026
\\
תרגול 3 - מרחבים שמורים והטלות
\\
אלן סורני
\\
הרשימות עודכנו לאחרונה בתאריך ה־%
\today
}
\date{}

\begin{document}
\maketitle

\section{סכומים ישרים של העתקות}

\begin{exercise}
יהיו
$V_1, V_2, W_1, W_2$
מרחבים וקטוריים סוף־מימדיים מעל שדה
$\FF$,
עם בסיסים
$B_1, B_2, C_1, C_2$
בהתאמה.
תהיינה
$T \in \hom_\FF\prs{V_1, V_2}$
ו־%
$S \in \hom_\FF\prs{W_1, w_2}$.

הראו שמתקיים
\begin{align*}
\text{.} \brs{T \oplus S}^{B_1 * C_1}_{B_2 * C_2} = \pmat{\brs{T}^{B_1}_{B_2} & 0 \\ 0 & \brs{S}^{C_1}_{C_2}}
\end{align*}
\end{exercise}

\begin{solution}
נסמן
\begin{align*}
B_1 &= \prs{u_1, \ldots, u_n} \\
\text{.} C_1 &= \prs{w_1, \ldots, w_m}
\end{align*}
ניזכר בהגדרת מטריצה מייצגת ונראה כי
$\brs{T \oplus S}^{B_1 * C_1}_{B_2 * C_2}$
שווה למטריצה
\begin{align*}
& \pmat{
\vert & & \vert & \vert & & \vert
\\
\brs{\prs{T \oplus S}\prs{u_1, 0}}_{B_2 * C_2} & \cdots & \brs{\prs{T \oplus S}\prs{u_n, 0}}_{B_2 * C_2} & \brs{\prs{T \oplus S}\prs{0, w_1}}_{B_2 * C_2} & \cdots & \brs{\prs{T \oplus S}\prs{0, w_m}}_{B_2 * C_2}\\
\vert & & \vert & \vert & & \vert
}
\\\text{.}&=
\pmat{
\vert & & \vert & \vert & & \vert
\\
\brs{\prs{T\prs{u_1}, 0}}_{B_2 * C_2} & \cdots & \brs{\prs{T\prs{u_n}, 0}}_{B_2 * C_2} & \brs{\prs{0, S\prs{w_1}}}_{B_2 * C_2} & \cdots & \brs{\prs{0, S\prs{w_m}}}_{B_2 * C_2}\\
\vert & & \vert & \vert & & \vert
}
\end{align*}
אבל,
\[\brs{\prs{v, 0}}_{B_2 * C_2} = \pmat{\brs{v}_{B_2} \\ 0 \\ \vdots \\ 0}\]
לכל
$v \in V_2$,
כי
\[\prs{v, 0} = \alpha_1 \prs{v_1, 0} + \ldots + \alpha_\ell \prs{v_\ell, 0}\]
כאשר
\[\text{,} v = \alpha_1 v_1 + \ldots \alpha_\ell v_\ell\]
ובאותו אופן
\[\text{.} \brs{\prs{0, w}}_{B_2 * C_2} = \pmat{0 \\ \vdots \\ 0 \\ \brs{w}_{C_2}}\]
\end{solution}

\section{מרחבים שמורים}

כאשר יש לנו אופרטור
$T$
על מרחב וקטורי
$V$,
יכול להיות נוח לנסות להבין את
$T$
לפי איך שהוא פועל על תת־מרחבים קטנים יותר.
אבל, ניתן לצמצם את
$T$
\emph{לאופרטור}
על תת־מרחב, רק אם התת־מרחב הינו
\emph{$T$%
־שמור}.

\begin{definition}[מרחב שמור]
יהי
$V$
מרחב וקטורי מעל שדה
$\mbb{F}$,
ויהי
$T \in \End_\FF\prs{V}$.

תת־מרחב
$W \leq V$
נקרא
\emph{$T$%
־שמור}
אם
\[\text{.} T(W) \coloneqq \set{T\prs{w}}{w \in W} \subseteq W\]
\end{definition}

\begin{definition}[צמצום למרחב שמור]
יהי
$T \in \End_\FF\prs{V}$
ויהי
$W \leq V$
תת־מרחב
$T$%
־שמור.

\emph{הצמצום של
$T$
ל־%
$W$}
הוא
\begin{align*}
\left. T \right|_W \colon W &\to W \\
\text{.} \hphantom{lalala} w &\mapsto T\prs{w}
\end{align*}
\end{definition}

\begin{remark}
שימו לב שהסימון הוא אותו סימון כמו הצמצום של המקור, אך במסגרת הקורס צמצום אופרטורים יתייחס לזה שבהגדרה אלא אם כן יצוין מפורשות אחרת.
\end{remark}

\begin{exercise}
יהי
$\mbb{C}$
כמרחב וקטורי ממשי ויהי
\begin{align*}
T \colon \mbb{C} &\to \mbb{C} \\
\text{.} \hphantom{lala} z &\mapsto iz
\end{align*}
מצאו את התת־מרחבים ה־%
$T$%
־שמורים של
$\mbb{C}$
והסיקו כי
$T$
אינו לכסין מעל
$\mbb{R}$.
\end{exercise}

\begin{solution}
$\mbb{C}, \set{0}$
תת־מרחבים
$T$%
־שמורים.

נניח כי
$W \leq \mbb{C}$
מרחב
$T$%
־שמור נוסף. אז
$\dim_{\mbb{R}}\prs{W} = 1$
ולכן יש
\[z_0 \in \mbb{C}^\times \ceq \set{z \in \mbb{C}}{z \neq 0}\]
עבורו
$W = \spn\set{z_0}$.
נקבל
$T\prs{z_0} \in W$
לכן
$T\prs{z_0} = c z_0$
עבור
$c \in \mbb{R}$.
אבל
$c z_0 = i z_0$
גורר
$c = i$
בסתירה.

תת־מרחבים $T$־שמורים $1$־מימדיים של
$\mbb{C}$
הם
$\Span_{\mbb{R}}\prs{v}$
עבור
$v$
וקטור עצמי של
$T$.
 לכן אין ל־%
$T$
וקטורים עצמיים, ולכן הוא אינו לכסין מעל
$\mbb{R}$.
\end{solution}

\begin{exercise}
ניזכר כי כל מטריצה דומה למטריצה משולשת עליונה, וכי הערכים העצמיים של מטריצה משולשת עליונה מופיעים על האלכסון.

תהיינה
$A_1, \ldots, A_k$
מטריצות
כך ש־%
$A_i \in \Mat_{m_i}\prs{\mbb{F}}$
לכל
$i \in \brs{k}$.
הראו כי הערכים העצמיים של המטריצה
\begin{align*}
\diag\prs{A_1, \ldots, A_k} \coloneqq \pmat{A_1 & 0 & \cdots & 0 \\ 0 & \ddots & \ddots & \vdots \\ \vdots & \ddots & \ddots & 0 \\ 0 & \cdots & 0 & A_k}
\end{align*}
הם הערכים העצמיים של המטריצות
$A_1, \ldots, A_k$,
וכי הריבוי האלגברי של ערך עצמי
$\lambda$
של
$A$
הוא סכום הריבויים האלגבריים שלו כערך עצמי של
$A_1, \ldots, A_k$.
\end{exercise}

\begin{solution}
לכל
$i \in \brs{k}$,
קיימת
$P_i \in \Mat_{m_i}\prs{\FF}$
עבורה
$U_i \coloneqq P_i^{-1} A P$
מטריצה משולשת עליונה.

אז
\begin{align*}
\diag\prs{P_1^{-1}, \ldots, P_k^{-1}} \diag\prs{A_1, \ldots, A_k} \diag\prs{P_1, \ldots, P_k} &= \diag\prs{P_1^{-1} A_1 P_1, \ldots, P_k^{-1} A_k P_k} \\&= \diag\prs{U_1, \ldots, U_k}
\end{align*}
מטריצה משולשת עליונה, ומספר הפעמים שערך
$\lambda$
מופיע על האלכסון שלה הוא סכום מספרי הפעמים שהוא מופיע על אלכסוני המטריצות
$U_1, ..., U_k$.
\end{solution}

\section{הטלות}

\begin{definition}[הטלה]
יהי
$V$
מרחב וקטורי מעל שדה
$\mbb{F}$.
אופרטור
$T \in \End_\FF\prs{V}$
נקרא
\emph{הטלה}
אם קיימים תת־מרחבים וקטוריים
$U,W \leq V$
עבורם
$V = U \oplus W$
וגם
\begin{align*}
\text{.} \forall u \in U, w \in W : T\prs{u+w} = u
\end{align*}
במקרה זה הוא נקרא
\emph{ההטלה על
$U$
במקביל ל־%
$W$}
ומסומן
$P_{U,W}$.
\end{definition}

\begin{proposition}
אופרטור
$T$
הינו הטלה אם ורק אם
$T^2 = T$.
\end{proposition}

\begin{exercise}
יהי
$V$
מרחב וקטורי סוף־מימדי מעל שדה
$\mbb{F}$.

\begin{enumerate}
\item תהי
$P \in \End_{\mbb{F}}\prs{V}$
הטלה. הראו כי
$V = \ker\prs{P} \oplus \im\prs{P}$.

\item הראו כי
$T \in \End_{\mbb{F}}\prs{V}$
הטלה אם ורק אם קיים בסיס
$B$
של
$V$
עבורו
\[\text{.} \brs{T}_B = \pmat{0 & & & & & \\ & \ddots & & & & \\ & & 0 & & & \\ & & & 1 & & \\ & & & & \ddots & \\ & & & & & 1}\] 
\end{enumerate}
\end{exercise}

\begin{solution}
\begin{enumerate}
\item כיוון ש־%
$P$
הטלה, קיימים תת־מרחבים
$U, W \leq V$
עבורם
$V = U \oplus W$
וגם
$P\prs{u+w} = u$
לכל
$u \in U, w \in W$.

אז
$\im\prs{P} = U$
כי התמונה מוכלת ב־%
$U$
וכי לכל
$u \in U$
מתקיים
$P\prs{u} = u$.

בנוסף,
$W \subseteq \ker\prs{P}$
כי לכל
$w \in W$
מתקיים
$P\prs{w} = P\prs{0 + w} = 0$.
ממשפט המימדים, מתקיים כי
\[\dim_\FF \prs{\ker\prs{P}} = \dim_\FF\prs{V} - \dim_\FF\prs{\im\prs{P}}\]
אך גם כיוון ש־%
$V = U \oplus W$,
מתקיים כי
\[\text{.} \dim_\FF \prs{W} = \dim_\FF\prs{V} - \dim_\FF\prs{U} = \dim_\FF\prs{V} - \dim_\FF\prs{\im\prs{P}}\]
נקבל כי
$W$
תת־מרחב של
$\ker\prs{P}$
מאותו מימד כמו
$\ker\prs{P}$,
ולכן יש שוויון.

בסה"כ קיבלנו כי
$V = U \oplus W$
וכי
\begin{align*}
U &= \im\prs{P}, \\
\text{,} W &= \ker\prs{P}
\end{align*}
ולכן
\begin{align*}
\text{,} V = \im\prs{P} \oplus \ker\prs{P}
\end{align*}
כנדרש.

\item
נניח כי
$T$
הטלה. במקרה זה
$V = \ker\prs{T} \oplus \im\prs{T}$.
עבור בסיסים
\begin{align*}
C &= \prs{c_1, \ldots, c_m} \\
D &= \prs{d_{m+1}, \ldots, d_\ell}
\end{align*}
של
$\ker\prs{T}, \im\prs{T}$
בהתאמה, נקבל כי
$C * D$
בסיס של
$V$.
לכל
$c_i \in C $
מתקיים
$T\prs{c_i} = 0$,
לכן
$\dim\prs{\ker\prs{T}}$
העמודות הראשונות של
$\brs{T}_{C * D}$
הן עמודות אפסים.
לכל
$d_i \in D$
יש
$u_i \in V$
עבורו
\[\text{,} d_i = T\prs{u_i} = T^2\prs{u_i} = T\prs{T\prs{u_i}} = T\prs{d_i}\]
לכן
\[\brs{T\prs{d_i}}_{C * D} = \brs{d_i}_{C * D} = e_i\]
ולכן העמודה ה־%
$i$
עבור
$i \geq m$
היא
$e_i$,
ונקבל את הנדרש.

להיפך, נניח שקיים בסיס
$B = \prs{v_1, \ldots, v_n}$
כנ"ל. אז
$\brs{T^2}_B = \brs{T}_B^2 = \brs{T}_B$
ולכן
$T^2 = T$
ונקבל כי
$T$
הטלה.
\end{enumerate}
\end{solution}

\begin{exercise}[לא נעשה בתרגול]
הוכיחו או הפריכו את הטענות הבאות.

\begin{enumerate}
\item סכום של הטלות שונות הוא הטלה.
\item כפל בסקלר של הטלה הוא הטלה.
\item הרכבה של הטלות היא הטלה.
\end{enumerate}
\end{exercise}

\begin{solution}
\begin{enumerate}
\item הטענה לא נכונה.
יהי
$V = \RR^3$
עם הבסיס הסטנדרטי
$\St = \prs{e_1, e_2, e_3}$,
ויהיו
\begin{align*}
U_1 &= \Span\prs{e_1, e_2} \\
W_1 &= \Span\prs{e_3} \\
U_2 &= \Span\prs{e_1} \\
\text{.} W_2 &= \Span\prs{e_2, e_3}
\end{align*}
אז
$V = U_1 \oplus W_2 = U_2 \oplus W_2$
כי
$\prs{e_1, e_2} * \prs{e_3}, \prs{e_1} * \prs{e_2, e_3}$
בסיסים של
$V$.

אבל,
\begin{align*}
\prs{P_{U_1, W_1} + P_{U_2, W_2}}\prs{e_1} &= e_1 + e_1 = 2 e_1
\end{align*}
ולכן
$2$
ערך עצמי של
$P_{U_1, W_1} + P_{U_2, W_2}$,
בעוד להטלה יתכנו רק ערכים עצמיים
$0$
או
$1$.

\item
הטענה לא נכונה.
יהי
$V$
מרחב וקטורי ממימד
$n \in \NN_+$
מעל שדה
$\FF$.
אז הזהות
$\id_V = P_{V,\set{0}}$
הינה הטלה, אך להעתקה
$2 \id_V$
ערך עצמי
$2$
ולכן היא אינה הטלה.

\item
הטענה לא נכונה.

יהי
$V = \RR^2$
ונסתכל על ההטלות
\begin{align*}
P_1 &= P_{\Span\prs{e_1}, \Span\prs{e_2}} \\
\text{.} P_2 &= P_{\Span\prs{e_1 + e_2}, \Span\prs{e_1 - e_2}}
\end{align*}
אז
\begin{align*}
\prs{P_1 \circ P_2}\prs{e_1} &= \prs{P_1 \circ P_2}\prs{\frac{e_1 + e_2}{2} + \frac{e_1 - e_2}{2}}
\\&=
P_1\prs{\frac{e_1 + e_2}{2}}
\\&=
\frac{e_1}{2}
\end{align*}
ולכן
$\frac{1}{2}$
ערך עצמי של
$P_1 \circ P_2$,
ולכן היא אינה הטלה.

נשים לב כי אם
$P_1, P_2$
הטלות מתחלפות, מתקיים
\begin{align*}
\prs{P_1 \circ P_2}^2 = P_1^2 \circ P_2^2 = P_1 \circ P_2
\end{align*}
ואז
$P_1 \circ P_2$
כן הטלה.
\end{enumerate}
\end{solution}

\begin{definition}[דחיסה של העתקה]
יהי
$V$
מרחב וקטורי סוף־מימדי מעל שדה
$\mbb{F}$,
יהי
$T \in \End_{\FF}\prs{V}$,
יהיו
$U,W \leq V$
ויהיו
$B,C$
בסיסים של
$U,W$
בהתאמה.

נגדיר את
\emph{הדחיסה של
$T$
ל־%
$U$
במקביל ל־%
$W$}
בתור
\begin{align*}
T_{U,W}: U &\to U \\
\text{.} \hphantom{lalalala} u &\mapsto P_{U,W} \circ T
\end{align*}
\end{definition}

\begin{exercise}[לא נעשה בתרגול]
יהי
$V$
מרחב וקטורי סוף־מימדי, יהיו
$U,W \leq V$
עם בסיסים
$B, C$
בהתאמה, ויהי
$T \in \End_\FF\prs{V}$.

הראו כי
\begin{align*}
\brs{T}_{B * C} = \pmat{\brs{\left. T \right|_U}_B & * \\ 0 & \brs{T_{W,U}}_C}
\end{align*}
כאשר
$U$
הינו
$T$%
־שמור, וכי
\begin{align*}
\brs{T}_{B * C} = \pmat{\brs{T_{U,W}}_B & 0 \\ * & \brs{\left. T \right|_W}_C}
\end{align*}
כאשר
$W$
הינו
$T$%
־שמור.
\end{exercise}

\begin{solution}
נסמן
\begin{align*}
B &= \prs{u_1, \ldots, u_m} \\
\text{.} C &= \prs{w_1, \ldots, w_\ell}
\end{align*}

נניח כי
$U$
הינו
$T$%
־שמור.
המקרה בו
$W$
שמור הינו אנלוגי ומושאר כתרגיל.

לפי הגדרת מטריצה מייצגת,
$m$
העמודות הראשונות של
$\brs{T}_{B * C}$
הן וקטורי הקואורדינטות של
$T\prs{u_1}, \ldots, T\prs{u_m}$
לפי
$B * C$.
אך כיוון ש־%
$U$
הינו
$T$%
־שמור, מתקיים
\[T\prs{u_i} = \left. T \right|_U\prs{u_i} \in U\]
לכל
$i \in \brs{m}$,
ולכן
$\brs{T\prs{u_i}}_{B * C}$
הינו הוקטור שמתקבל לאחר הוספת
$\ell$
אפסים בסוף הוקטור
$\brs{\left. T \right|_U\prs{u_i}}_B$.
לכן, מטריצת
$m$
העמודות הראשונות של $\brs{T}_{B * C}$ היא
\begin{align*}
\text{.} \pmat{\brs{\left. T \right|_U}_B \\ 0}
\end{align*}

כמו כן, לכל
$i \in \brs{\ell}$
מתקיים כי
\begin{align*}
T\prs{w_i} &= P_{U,W}\prs{T\prs{w_i}} + P_{W,U}\prs{T\prs{w_i}}
\\&= P_{U,W}\prs{T\prs{w_i}} + T_{W,U}\prs{w_i}
\end{align*}
כאשר הגורם הראשון שייך ל־%
$U$,
ולכן $\ell$ המקדמים האחרונים בוקטור
$\brs{T\prs{w_i}}_{B*C}$
הם בדיוק מקדמי הוקטור
$\brs{T_{W,U}\prs{w_i}}$.
לכן, מטריצת
$\ell$
העמודות האחרונות של
$\brs{T}_{B * C}$
היא מהצורה
\begin{align*}
\text{,} \pmat{* \\ \brs{T_{W,U}\prs{w_i}}}
\end{align*}
כנדרש.
\end{solution}

\begin{exercise}
יהי
$V = \mbb{R}^2$
עם הבסיס הסטנדרטי
$\St = \prs{e_1, e_2}$.

נסמן
$V^* \coloneqq \hom_\RR\prs{V, \RR}$
וגם
\[\St^* \coloneqq \prs{e_1^*, e_2^*}\]
כאשר
\begin{align*}
e_1^*\prs{\pmat{x \\ y}} &= x \\
e_2^*\prs{\pmat{x \\ y}} &= y
\end{align*}
ההטלות על
$\Span{e_1}$
במקביל ל־%
$\Span\prs{e_2}$,
ועל
$\Span\prs{e_2}$
במקביל ל־%
$\Span\prs{e_1}$.

תהי
\begin{align*}
T \colon V &\to V \\
\text{.} \hphantom{lalala} \pmat{x \\ y} &\mapsto \pmat{x+y \\ y}
\end{align*}

\begin{enumerate}
\item
חשבו את
$\brs{T}_{\St}$.

\item נגדיר
\begin{align*}
T^* \colon V^* &\to V^* \\
\text{.} \hphantom{lalala} \phi &\mapsto \phi \circ T 
\end{align*}
חשבו את
$\brs{T^*}_{\St}$.
\end{enumerate}
\end{exercise}

\begin{solution}
\begin{enumerate}
\item
לפי הגדרת מטריצה מייצגת,
\begin{align*}
\brs{T}_{\St} &= \pmat{\vert & \vert \\ \brs{T\prs{e_1}}_{\St} & \brs{T\prs{e_2}}_{\St} \\ \vert & \vert}
\\ \text{.} \hphantom{\brs{T}_{\St}} &= \pmat{1 & 1 \\ 0 & 1}
\end{align*}

\item
לפי הגדרת מטריצה מייצגת,
\begin{align*}
\text{.} \brs{T^*}_{\St^*} &= \pmat{\vert & \vert \\ \brs{T^*\prs{e_1^*}}_{\St^*} & \brs{T^*\prs{e_2^*}}_{\St^*} \\ \vert & \vert}
\end{align*}
כעת,
\begin{align*}
\prs{T^*\prs{e_1^*}}\pmat{x \\ y} &= e_1^*\prs{T\pmat{x \\ y}}
\\&= e_1^* \pmat{x + y \\ y}
\\&= x+y
\end{align*}
וכן
\begin{align*}
\prs{T^*\prs{e_2^*}}\pmat{x \\ y} &= e_2^*\prs{T\pmat{x \\ y}}
\\&= e_2^* \pmat{x + y \\ y}
\\&= y
\end{align*}
ולכן
\begin{align*}
T^*\prs{e_1^*} &= e_1^* + e_2^* \\
\text{,} T^*\prs{e_2^*} &= e_2^*
\end{align*}
ונקבל כי
\begin{align*}
\text{.} \brs{T^*}_{\St^*} &= \pmat{1 & 0 \\ 1 & 1} = \brs{T}_{\St}^t
\end{align*}
\end{enumerate}
\end{solution}

\printbibliography
\end{document}
