\documentclass[a4paper,10pt,twoside,openany]{article}

\usepackage[lang=hebrew]{maths}
\usepackage{polynom}
\usepackage{hebrewdoc}
\usepackage{stylish}
\usepackage{lipsum}
\let\bs\blacksquare
\usepackage{ytableau}

\setlength{\parindent}{0pt}
\newcommand{\Std}{\mrm{Std}}
\DeclareMathOperator{\Sym}{Sym}

%%%%%%%%%%%%
% Styling %
%%%%%%%%%%%%

\usepackage{enumitem}

%%%%%%%%%%%%%
% Counters  %
%%%%%%%%%%%%%

\setcounter{section}{0}     
            
%BIBLIOGRAPHY
\usepackage[
backend=biber,
style=alphabetic,
]{biblatex}
\addbibresource{bibliography.bib} %Imports bibliography file

\title{
אלגברה ב' (01040168) - חורף 2026
\\
תרגול 10 - פולינומים אופייני ומינימלי, והדטרמיננטה
\\
אלן סורני
\\
הרשימות עודכנו לאחרונה בתאריך ה־%
\today
}
\date{}

\begin{document}
\maketitle

\section{פולינומים אופייני ומינימלי}

\begin{definition}[פולינום אופייני]
יהי $V$ מרחב וקטורי סוף־מימדי ויהי
$T \in \End_{\FF}\prs{V}$
עם ערכים עצמיים שונים
$\lambda_1, \ldots, \lambda_k$
כאשר הערך העצמי
$\lambda_i$
מריבוי אלגברי
$r_i \coloneqq r_a\prs{\lambda_i}$
וכאשר
\[\text{.} \sum_{i \in \brs{k}} r_i = \dim\prs{V}\]

\emph{הפולינום האופייני של
$T$}
הוא
\begin{align*}
\text{.} p_T\prs{x} \coloneqq \prod_{i \in \brs{k}} \prs{x - \lambda_i}^{r_i}
\end{align*}
\end{definition}

\begin{definition}[פולינום מינימלי]
יהי $V$ מרחב וקטורי סוף־מימדי ויהי
$T \in \End_{\FF}\prs{V}$.
\emph{הפולינום המינימלי של
$T$}
הוא הפולינום
$p \in \FF\brs{x}$
המתוקן מהמעלה המינימלית עבורו
$p\prs{T} = 0$.
נסמנו
$m_i\prs{x}$.
\end{definition}

\begin{proposition}
נניח כי
$\FF$
סגור אלגברית.
אז ל־%
$T$
כנ"ל יש פולינום מינימלי, והחזקה של
$\prs{x-\lambda_i}$
בפולינום המינימלי היא אינדקס הנילפוטנטיות של
$\left. T - \lambda_i \id_V \right|_{V'_{\lambda_i}}$,
שהוא גודל בלוק ז'ורדן המקסימלי עם ערך עצמי
$\lambda_i$
בצורת ז'ורדן של
$T$.
\end{proposition}

\begin{proposition}
יהי
$T$
כנ"ל ויהי
$f \in \FF\brs{x}$
עבורו
$f\prs{T} = 0$.
אז
$m_T \mid f$.
כלומר, קיים פולינום
$g \in \FF\brs{x}$
עבורו
$m_T\prs{x} = f\prs{x} g\prs{x}$.
\end{proposition}

\begin{corollary}[משפט קיילי־המילטון]
יהי
$V$
מרחב וקטורי סוף־מימדי מעל שדה
$\FF$.
לכל
$T \in \End_\FF\prs{V}$
מתקיים
$p_T\prs{T} = 0$.
\end{corollary}

\begin{remark}
עבור מטריצה ריבועית
$A$,
נגדיר
\begin{align*}
p_A &\coloneqq p_{L_A}, \\
\text{,} m_A &\coloneqq m_{L_A}
\end{align*}
ובאותו אופן אם
$f\prs{A} = 0$
אז
$m_A \mid f$,
ומתקיים
$p_A\prs{A} = 0$.
\end{remark}

\begin{exercise}
יהי
$\mbb{F} = \mbb{C}$
ותהי
$T \in \endo_{\mbb{F}}\prs{V}$.
יהי
$m \in \mbb{N}_+$
עבורו
$T^m = \id_V$.
הראו כי
$T$
לכסין.
\end{exercise}

\begin{solution}
כדי להראות ש־%
$T$
לכסין מספיק להראות שכל שורשי
$m_T$
הינם מריבוי
$1$.
מההנחה, מתקיים
$m_T \mid \prs{x^m - 1}$
ולכן די להראות שכל שורשי
$x^m - 1$
הם מריבוי
$1$.
אכן, יש לפולינום זה
$m$
שורשים שונים
\[\text{.} \set{e^{\frac{2 \pi i k}{m}}}{k \in [m]} = \set{\mrm{cis}\prs{\frac{2 \pi k}{m}}}{k \in [m]}\]
\end{solution}

\begin{exercise}
\begin{enumerate}
\item
תהיינה
$A,B \in \Mat_6\prs{\mbb{C}}$
המקיימות
\begin{enumerate}[label = (\roman*)]
\item $p_A = p_B$.
\item $m_A = m_B$
וזהו פולינום ממעלה
$5$.
\end{enumerate}
הראו כי
$A \sim B$.

\item
מצאו
$A,B \in \Mat_6\prs{\mbb{C}}$
שאינן דומות וכך שמתקיים
\begin{enumerate}[label = (\roman*)]
\item $p_A = p_B$.
\item $m_A = m_B$
וזהו פולינום ממעלה
$4$.
\end{enumerate}
\end{enumerate}
\end{exercise}

\begin{solution}
\begin{enumerate}
\item נתון
$\deg m_A = 5$,
וזהו סכום גדלי הבלוקים המקסימליים של הערכים העצמיים השונים של
$A$.
לכן יש ערך עצמי
$\lambda$
מריבוי גיאומטרי
$2$
ובלוק מגודל
$1$,
וכל ערך עצמי אחר הוא מריבוי גיאומטרי
$1$.
אז
\[\text{.} A \sim \pmat{\lambda & 0 & \cdots & & 0 \\ 0 & J_m\prs{\lambda_1} & \ddots & & \\ \vdots & \ddots & \ddots & & \vdots \\ & & & \ddots & 0 \\ 0 & & \cdots & 0 & J_{m_r}\prs{\lambda_r}}\]
נתון
$m_A = m_B$
לכן באותו אופן
\[B \sim \pmat{\lambda & 0 & \cdots & & 0 \\ 0 & J_m\prs{\lambda_1} & \ddots & & \\ \vdots & \ddots & \ddots & & \vdots \\ & & & \ddots & 0 \\ 0 & & \cdots & 0 & J_{m_r}\prs{\lambda_r}}\]
ונסיק כי
$A \sim B$.

\item נסתכל על
\begin{align*}
A &\ceq \pmat{J_1\prs{0} & & \\  & J_1\prs{0} & \\ & & J_4\prs{0}} \\
B &\ceq \pmat{J_2\prs{0} & \\ & J_4\prs{0}}
\end{align*}
ונקבל
$m_A = m_B = x^4$
אבל
$A \not\sim B$
כי יש להן צורת ז'ורדן שונה.
\end{enumerate}
\end{solution}

\section{הדטרמיננטה}

\begin{definition}[תמורה]
תהי
$X$
קבוצה.
\emph{תמורה על
$X$}
היא העתקה הפיכה
$\sigma \colon X \to X$.
\end{definition}

\begin{notation}
נסמן ב־%
$\Sym\prs{n}$
את התמורות על הקבוצה
$\brs{n}$.
\end{notation}

\begin{definition}[סימן של תמורה]
תהי
$\sigma \in \Sym\prs{n}$.
\emph{הסימן של
$\sigma$}
הוא
\begin{align*}
\mrm{sgn}\prs{\sigma} \coloneqq \prs{-1}^k
\end{align*}
כאשר
\begin{align*}
\text{.} k \coloneqq \abs{\set{\prs{i,j} \in \brs{n}^2}{\substack{i < j \\ \sigma\prs{i} > \sigma\prs{j}}}}
\end{align*}
\end{definition}

\begin{definition}[הדטרמיננטה]
תהי
$A = \prs{a_{i,j}}_{i,j \in \brs{n}} \in \Mat_n\prs{\FF}$.
\emph{הדטרמיננטה של
$A$}
היא
\begin{align*}
\text{.} \det\prs{A} \coloneqq \sum_{\sigma \in \Sym\prs{n}} \mrm{sgn}\prs{\sigma} \prod_{i \in \brs{k}} a_{i, \sigma\prs{i}}
\end{align*}
\end{definition}

\begin{definition}[תבנית
$k$%
־לינארית]
יהי
$V$
מרחב וקטורי מעל שדה
$\FF$.
תבנית
$k$%
־לינארית על
$V$
היא העתקה
\begin{align*}
f\prs{\cdot, \ldots, \cdot} \colon V^k &\to \FF
\end{align*}
שהינה לינארית בכל רכיב.
\end{definition}

\begin{definition}[תבנית מתחלפת]
תהי
$f$
תבנית
$k$%
־לינארית על מרחב וקטורי
$V$
מעל
$\FF$.
נגיד כי
$f$
\emph{תבנית מתחלפת}
אם לכל
$\sigma \in \Sym\prs{k}$
ולכל
$v_1, \ldots, v_k \in V$
מתקיים
\[\text{.} f\prs{v_{\sigma\prs{1}}, \ldots, v_{\sigma\prs{k}}} = \mrm{sgn}\prs{\sigma} f\prs{v_1, \ldots, v_k}\]
\end{definition}

\begin{exercise}
נוכל לחשוב על הדטרמיננטה כעל העתקה
\[\prs{\FF^n}^n \to \FF\]
שלוקחת וקטורים
$v_1, \ldots, v_n \in \FF^n$
לדטרמיננטה של
\[\text{.} \pmat{\vert & & \vert \\ v_1 & \cdots & v_n \\ \vert & & \vert} \in \Mat_n\prs{\FF}\]

\begin{enumerate}
\item הראו שהעתקה זאת לינארית בכל רכיב.

\item הראו שמתקיים
\[\det\prs{v_1, \ldots, v_i, v_{i+1}, \ldots, v_n} = - \det\prs{v_1, \ldots, v_{i+1}, v_{i}, \ldots, v_n}\]
והסבירו איך נובע ש־%
$\det$
תבנית
$n$%
־לינארית מתחלפת.
היעזרו בכך שמתקיים
\[\mrm{sgn}\prs{\sigma \tau} = \mrm{sgn}\prs{\sigma} \mrm{sgn}\prs{\tau}\]
לכל זוג תמורות
$\sigma, \tau \in \Sym\prs{k}$.
\end{enumerate}
\end{exercise}

\begin{solution}
\begin{enumerate}
\item
יהיו
$v_1, \ldots, v_{i-1}, v_{i+1} v_n, u,w \in \FF^n$
ותהי
$\alpha \in \FF$.
נסמן
$v_i = \alpha u + w$.
נראה שמתקיים
\begin{align*}
\text{.} \det\prs{v_1, \ldots, v_{i-1}, \alpha u + w, v_{i+1}, \ldots, v_n} = \alpha \det{v_1, \ldots, v_{i-1}, u, v_{i+1}, \ldots, v_n} + \det{v_1, \ldots, v_{i-1}, w, v_{i+1}, \ldots, v_n}
\end{align*}

אכן,
\begin{align*}
\det\prs{v_1, \ldots, v_{i-1}, \alpha u + w, v_{i+1}, \ldots, v_n} &= \sum_{\sigma \in \Sym\prs{n}} \prod_{j \in \brs{k}} \mrm{sgn}\prs{\sigma} \prs{v_i}_{\sigma^{-1}\prs{i}}
\\&=
\sum_{\sigma \in \Sym\prs{n}} \mrm{sgn}\prs{\sigma} \prs{\prod_{\substack{j \in \brs{n} \\ \sigma\prs{j} \neq i}}  \prs{v_{\sigma\prs{j}}}_j} \cdot \prs{v_i}_{\sigma^{-1}\prs{i}}
\\&=
\sum_{\sigma \in \Sym\prs{n}} \mrm{sgn}\prs{\sigma} \prs{\prod_{\substack{j \in \brs{n} \\ \sigma\prs{j} \neq i}}  \prs{v_{\sigma\prs{j}}}_j} \cdot \prs{\alpha u_{\sigma^{-1}\prs{i}} + w_{\sigma^{-1}\prs{i}}}
\\&=
\alpha \sum_{\sigma \in \Sym\prs{n}} \mrm{sgn}\prs{\sigma} \prs{\prod_{\substack{j \in \brs{n} \\ \sigma\prs{j} \neq i}}  \prs{v_{\sigma\prs{j}}}_j} \cdot u_{\sigma^{-1}\prs{i}}
\\&+ \sum_{\sigma \in \Sym\prs{n}} \mrm{sgn}\prs{\sigma} \prs{\prod_{\substack{j \in \brs{n} \\ \sigma\prs{j} \neq i}}  \prs{v_{\sigma\prs{j}}}_j} \cdot w_{\sigma^{-1}\prs{i}}
\\&=
\alpha \det\prs{v_1, \ldots, v_{i-1}, u, v_{i+1}, \ldots, v_n} + \det\prs{v_1, \ldots, v_{i-1}, w, v_{i+1}, \ldots, v_n}
\end{align*}
כאשר במעבר האחרון השתמשנו בכך ש־%
$u,w$
העמודות ה־%
$i$
במטריצות שעמודותיהן
$\prs{v_1, \ldots, v_{i-1}, u, v_{i+1}, \ldots, v_n}$
ו־%
$\prs{v_1, \ldots, v_{i-1}, w, v_{i+1}, \ldots, v_n}$
בהתאמה.

\item תהי
$\sigma \in \Sym\prs{k}$.
בחישוב הדטרמיננטה באגף שמאל יופיע הגורם
\[\text{,} \mrm{sgn}\prs{\sigma} \prod_{j \in \brs{k}} \prs{v_{\sigma\prs{j}}}_j\]
ואילו בחישוב הדטרמיננטה של אגף ימין יופיע הגורם
\[\text{.} \mrm{sgn}\prs{\sigma} \prs{\prod_{\substack{j \in \brs{k} \\ \sigma\prs{j} \notin \set{i, i+1}}} \prs{v_{\sigma\prs{j}}}_j}
\cdot \prs{v_{i}}_{\sigma^{-1}\prs{i+1}} \cdot \prs{v_{i+1}}_{\sigma^{-1}\prs{i}}\]
נסתכל על התמורה
$\tau = \prs{i, i+1}$
ועל התמורה
$\tilde{\sigma} \coloneqq \sigma \circ \tau$.
אז המכפלה שבגורם הראשון תופיע בגורם
\begin{align*}
\mrm{sgn}\prs{\tilde{\sigma}} \prod_{j \in \brs{k}} \prs{v_{\tilde{\sigma}}\prs{j}}_j
\end{align*}
בחישוב הדטרמיננטה שבאגף ימין.
אבל,
$\mrm{sgn}\prs{\tau} = -1$
ולכן
$\mrm{sgn}\prs{\sigma \circ \tau} = -\mrm{sgn}\prs{\sigma}$,
ולכן
\begin{align*}
\mrm{sgn}\prs{\sigma} \prod_{j \in \brs{k}} \prs{v_{\sigma\prs{j}}}_j = - \mrm{sgn}\prs{\tilde{\sigma}} \prod_{j \in \brs{k}} \prs{v_{\tilde{\sigma}}\prs{j}}_j
\end{align*}
ונקבל כי
\begin{align*}
\det\prs{v_1, \ldots, v_n} &= \sum_{\sigma \in \Sym\prs{n}} \mrm{sgn}\prs{\sigma} \mrm{sgn}\prs{\sigma} \prod_{j \in \brs{k}} \prs{v_{\sigma\prs{j}}}_j
\\&=
- \sum_{\sigma \in \Sym\prs{n}} \mrm{sgn}\prs{\tilde{\sigma}} \prod_{j \in \brs{k}} \prs{v_{\tilde{\sigma}}\prs{j}}_j
\\&=
- \sum_{\sigma' \in \Sym\prs{n}} \mrm{sgn}\prs{\sigma'} \prod_{j \in \brs{k}} \prs{v_{\sigma'}\prs{j}}_j \\
- \det\prs{v_1, \ldots, v_{i+1}, v_i, \ldots, v_n}
\end{align*}
כאשר השתמשנו בכך שהעתקה
$\sigma \mapsto \tilde{\sigma}$
הפיכה (היא ההופכית של עצמה).

כיוון שכל תמורה ניתן לכתוב כהרכבת חילופים, וכיוון שהסימן הוא כפלי, נקבל מכך את הנדרש. אם
$\sigma = \tau_1 \circ \ldots \tau_m$
עבור חילופים
$\tau_i$,
נקבל כי הדטרמיננטה
$\det\prs{v_{\sigma\prs{1}}, \ldots, v_{\sigma\prs{k}}}$
היא
$\prs{-1}^m$,
וגם כי
$\mrm{sgn}\prs{\sigma} = \prs{-1}^m$.
\end{enumerate}
\end{solution}

\begin{proposition}
$\det\prs{A} = \det\prs{A^t}$.
\end{proposition}

\begin{corollary}
תהי
$A \in \Mat_n\prs{\FF}$.
הדטרמיננטה של המטריצה המתקבלת מ־%
$A$
לאחר
\begin{enumerate}
\item הוספת כפולה של שורה או עמודה לשורה או עמודה (בהתאם) אחרת היא
$\det\prs{A}$.

\item החלפת שורות או עמודות של
$A$
היא
$\prs{-1} \det\prs{A}$.

\item כפל של שורה או עמודה בסקלר
$\alpha$
היא
$\alpha \det\prs{A}$.
\end{enumerate}

בנוסף,
אם
$B \in \Mat_n\prs{\FF}$,
מתקיים
$\det\prs{AB} = \det\prs{BA}$,
ואם
$A$
הפיכה מתקיים
$\det\prs{A^{-1}} = \det\prs{A}^{-1}$.
\end{corollary}

\begin{exercise}
עבור פולינום מתוקן
$p \in \mbb{F}\brs{x}$
נכתוב
\[p\prs{x} = \sum_{i = 0}^n c_i x^i\]
כאשר
$c_n = 1$,
ונגדיר את
\emph{המטריצה המלווה של
$p$}
על ידי
\[\text{.} C\prs{p} \ceq \pmat{0 & 0 & \cdots & 0 & -c_0 \\ 1 & 0 & \cdots & 0 & -c_1 \\ 0 & 1 & \cdots & 0 & -c_2 \\ \vdots & \vdots & \ddots & \vdots & \vdots \\ 0 & 0 & \dots & 1 & -c_{n-1}} \in M_n\prs{\mbb{F}}\]

יהי
$p \in \mbb{F}\brs{x}$
ממעלה
$n$.
מצאו את הפולינום האופייני ואת הפולינום המינימלי של
$C\prs{p}$.
\end{exercise}

\begin{solution}
\begin{itemize}
\item
מתקיים
\begin{align*}
\text{.} p_{C\prs{p}} &= \det\prs{I - xC\prs{p}}
= \det\pmat{x & 0 & \cdots & 0 & c_0 \\ -1 & x & \cdots & 0 & c_1 \\ 0 & -1 & \cdots & 0 & -c_2 \\ \vdots & \vdots & \ddots & \ddots & \vdots \\ 0 & 0 & \dots & -1 & x + c_{n-1}}
\end{align*}
ניעזר בכך שהטרמיננטה אינווריאנטית תחת הוספת כפולה של שורה לשורה אחרת.
נוסיף את השורה האחרונה כפול
$x$
לזאת שלפניה. לאחר מכן נוסיף את השורה ה־%
$n-1$
כפול
$x$
לזאת שלפניה ונמשיך כך עד שנקבל מטריצה
\begin{align*}
\pmat{0 & 0 & \cdots & 0 & y \\ -1 & 0 & \cdots & 0 & * \\ 0 & -1 & \cdots & 0 & * \\ \vdots & \vdots & \ddots & \ddots & \vdots \\ 0 & 0 & \dots & -1 & *}
\end{align*}
כאשר
\[\text{.} y = a_0 + x\prs{a_1 + x\prs{a_2 + x\prs{\ldots \prs{a_{n-2} + x\prs{a_{n-1} + x}} \ldots}}} = \sum_{i=0}^n c_i x^i = f\]
אז
\[\text{.} p_{C\prs{p}} = \prs{-1}^{n-1} \det \pmat{-1 & x & \cdots & 0 \\ 0 & -1 & \cdots & 0 \\ \vdots & \vdots & \ddots & \vdots \\ 0 & 0 & \cdots & - 1} = \prs{-1}^{n-1} \prs{-1}^{n-1} f = f\]
\item
יהי
$g \in \mbb{F}_{n-1}\brs{x}$
ונכתוב
\[\text{.} g\prs{x} = \sum_{i = 0}^{n-1} b_i x^i\]
אז
\begin{align*}
g\prs{C\prs{p}}\prs{e_1} &= \sum_{i = 0}^{n-1} b_i C\prs{p}^i e_1
\\&= \sum_{i=0}^{n-1} b_i e_{1+i}
\end{align*}
וביטוי זה שונה מאפס כאשר
$g \neq 0$
כי אז יש
$b_i \neq 0$
עבור
$i$
כלשהו.

לכן
$\deg\prs{m_{C\prs{p}}} \geq n$
ולכן
$m_{C\prs{p}} = p_{C\prs{p}}$.
\end{itemize}
\end{solution}

\begin{exercise}
יהי
$V$
מרחב וקטורי ממימד $n \in\NN_+$ מעל שדה
$\FF$
ותהי
$T \in \End_\FF\prs{V}$.
נגדיר
\begin{align*}
\det\prs{T} \coloneqq \det\prs{\brs{T}_B}
\end{align*}
כאשר
$B$
בסיס של
$V$.

הראו שהדטרמיננטה הזאת מוגדרת היטב.
\end{exercise}

\begin{solution}
יהיו
$B,C$
שני בסיסים של
$V$.
עלינו להראות כי
$\det\prs{\brs{T}_B} = \det\prs{\brs{T}_C}$.
ניזכר כי מתקיים
\begin{align*}
\text{.} \brs{T}_B = \prs{M^B_C}^{-1} \brs{T}_C M^B_C
\end{align*}
אז, מכפליות הדטרמיננטה,
\begin{align*}
\text{.} \det\prs{\brs{T}_B} &= \det\prs{\prs{M^B_C}^{-1} \brs{T}_C M^B_C}
\\&= \det\prs{\prs{M^B_C}^{-1}} \det\prs{\brs{T}_C} \det\prs{M^B_C}
\\&= \det\prs{\prs{M^B_C}}^{-1} \det\prs{\brs{T}_C} \det\prs{M^B_C}
\\&= \det\prs{\brs{T}_C}
\end{align*}
כאשר בשוויון השלישי השתמשנו בכך שאם
$A$
הפיכה מתקיים
$\det\prs{A^{-1}} = \det\prs{A}^{-1}$.
\end{solution}

\printbibliography
\end{document}
