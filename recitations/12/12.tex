\documentclass[a4paper,10pt,twoside,openany]{article}

\usepackage[lang=hebrew]{maths}
\usepackage{polynom}
\usepackage{hebrewdoc}
\usepackage{stylish}
\usepackage{lipsum}
\let\bs\blacksquare

\setlength{\parindent}{0pt}
\newcommand{\Std}{\mrm{Std}}
\DeclareMathOperator{\Char}{char}

%%%%%%%%%%%%
% Styling %
%%%%%%%%%%%%

\usepackage{enumitem}

%%%%%%%%%%%%%
% Counters  %
%%%%%%%%%%%%%

\setcounter{section}{0}     
            
%BIBLIOGRAPHY
\usepackage[
backend=biber,
style=alphabetic,
]{biblatex}
\addbibresource{bibliography.bib} %Imports bibliography file

\title{
אלגברה ב' (01040168) - חורף 2026
\\
תרגול 12 - קריטריון סילבסטר, ותבניות ריבועיות
\\
אלן סורני
\\
הרשימות עודכנו לאחרונה בתאריך ה־%
\today
}
\date{}

\begin{document}
\maketitle

\section{קריטריון סילבסטר}

\begin{theorem}[סילבסטר]
תהי
$f \in \mrm{Bil}\prs{V}$
תבנית בילינארית סימטרית על מרחב מכפלה פנימית
$V$,
עם סימן
$\prs{r,s,t}$.
יהי
$B$
בסיס של
$V$
ותהי
$A = \brs{f}_B$.
נסמן ב־%
$\gamma_k \coloneqq \det\prs{\prs{a_{i,j}}_{i,j \in [k]}}$
את
\emph{המינורים הראשיים}
של
$A$
ונניח כי
$\gamma_k \neq 0$
לכל
$k \in [n]$.
אז
$t = 0$,
מספר הערכים החיוביים בקבוצה הסדורה
$\prs{\gamma_1, \frac{\gamma_2}{\gamma_1}, \ldots, \frac{\gamma_n}{\gamma_{n-1}}}$
הוא
$r$,
ומספר הערכים השליליים בה הוא
$s$.
\end{theorem}

\begin{exercise}
תהי
$f \in \Bil\prs{V}$,
יהי
$B$
בסיס של
$V$
ותהי
$A = \brs{f}_B$.

הראו כי
$f$
מכפלה פנימית אם ורק אם המינורים הראשיים של
$A$
חיוביים.
\end{exercise}

\begin{solution}
נניח כי כל המינורים הראשיים חיוביים. נקבל כי כל האיברים בקבוצה הסדורה
$\prs{\gamma_1, \frac{\gamma_2}{\gamma_1}, \ldots, \frac{\gamma_n}{\gamma_{n-1}}}$
חיוביים, ולכן ממשפט סילבסטר הסימן של
$f$
הוא
$\prs{n,0,0}$.
תבנית בילינארית סימטרית הינה מכפלה פנימית אם הסימן שלה כזה, לכן
$f$
מכפלה פנימית.

נניח כי
$f$
מכפלה פנימית. אז
$\gamma_k = \det\prs{\prs{a_{i,j}}_{i,j \in [k]}}$
כאשר
$\prs{a_{i,j}}_{i,j \in [k]}$
מטריצה מייצגת של
$\left. f \right|_{\Span\prs{b_1, \ldots, b_k}}$.
צמצום של מכפלה פנימית הוא מכפלה פנימית, והדטרמיננטה של מטריצה מוגדרת חיובית לחלוטין היא חיובית, לכן כל המינורים הראשיים חיוביים.
\end{solution}

\begin{exercise}
תהי
$f \in \mrm{Bil}\prs{V}$
ויהי
$B$
בסיס של
$V$.
מיצאו תנאי הכרחי ומספיק על המינורים
$\gamma_i\prs{\brs{f}_B}$
לכך ש־%
$f$
מוגדרת שלילית (לחלוטין).
\end{exercise}

\begin{solution}
ניעזר בתרגיל הקודם, ונשים לב כי
$f$
מוגדרת שלילית אם ורק אם
$-f$
מוגדרת חיובית לחלוטין. אכן, מתקיים
$f\prs{v,v} < 0$
לכל
$v \in V$
אם ורק אם
$-f\prs{v,v} > 0$
לכל
$v \in V$.

כעת,
$-f$
מוגדרת חיובית אם ורק אם המינורים הראשיים של
$-A \coloneqq \brs{-f}_B$
חיוביים. כעת,
\[\gamma_k\prs{A} = \det\prs{\prs{a_{i,j}}_{i,j \in [k]}} = \prs{-1}^k \det\prs{\prs{-a_{i,j}}_{i,j \in [k]}} = \prs{-1}^k \gamma_k\prs{-A}\]
לכן זה מתקיים אם ורק אם
$\sgn\prs{\gamma_k\prs{A}} = \prs{-1}^k$
לכל
$k \in [n]$.
\end{solution}

\begin{exercise}
תהי
\[\text{.} A = \pmat{1 & 0 & 1 & 0 & 1 \\ 0 & 2 & 3 & 1 & 1 \\ 1 & 3 & -7 & 1 & 2 \\ 0 & 1 & 1 & 1 & 1 \\ 1 & 1 & 1 & 2 & 1} \in \Mat_5\prs{\mbb{R}}\]
מיצאו את הסימן של
$A$.
\end{exercise}

\begin{solution}
ניעזר בקריטריון סילבסטר.
מתקיים
\begin{align*}
    \det\prs{1} &= 1 \\
    \det\prs{\pmat{1 & 0 \\ 0 & 2}} &= 2 \\
    \det\prs{\pmat{1 & 0 & 1 \\ 0 & 2 & 3 \\ 1 & 3 & 7}} &= 1 \det\pmat{2 & 3 \\ 3 & 7} + 1 \det\pmat{0 & 2 \\ 1 & 3} = -23-2 = -25 \\
    \det\pmat{1 & 0 & 1 & 0 \\ 0 & 2 & 3 & 1 \\ 1 & 3 & -7 & 1 \\ 0 & 1 & 1 & 1} &= -25 - \det\pmat{1 & 0 & 0 \\ 0 & 2 & 1 \\ 1 & 3 & 1} + \det\pmat{1 & 1 & 0 \\ 0 & 3 & 1 \\ 1 & -7 & 1} = -25 - \prs{-1} + \prs{10+1} = -13 \\
    \text{.} \det A &= {\ldots} = 13
\end{align*}
נקבל כי המינורים הראשיים לא מתאפסים, וכי הסימן שלהם מתחיל חיובי ומתחלף פעמיים. מקריטריון סילבסטר נקבל כי הסימן של
$A$
הוא
$\prs{3,2,0}$.
\end{solution}

\begin{exercise}
הראו כי
\[A \ceq \pmat{-1 & 1 & 0 & 1 \\ 1 & -3 & 0 & 0 \\ 0 & 0 & -1 & -1 \\ 1 & 0 & -1 & -3}\]
מוגדרת שלילית.
\end{exercise}

\begin{solution}
נחשב את המינורים הראשיים.
מתקיים
\begin{align*}
    \det\prs{-1} &= -1 < 0 \\
    \det\pmat{-1 & 1 \\ 1 & -3} &= 3-1 = 2 > 0 \\
    \det\pmat{-1 & 1 & 0 \\ 1 & -3 & 0 \\ 0 & 0 & -1} &= -1 \cdot \det\pmat{-1 & 1 \\ 1 & -3} = -2 < 0 \\
    \det A &= -3 \cdot \det\pmat{-1 & 1 & 0 \\ 1 & -3 & 0 \\ 0 & 0 & -1} +1 \cdot \det\pmat{-1 & 1 & 1 \\ 1 & -3 & 0 \\ 0 & 0 & -1} - 1 \cdot \det\pmat{1 & 0 & 1 \\ -3 & 0 & 0 \\ 0 & -1 & -1} \\&= 6 - \prs{3-1} - \prs{-3}\prs{-1}\prs{1} = 6 - 2 - 3 = 1 > 0
\end{align*}
ולכן מהתרגיל הקודם
$A$
מוגדרת שלילית.
\end{solution}

\section{תבניות ריבועיות}

\begin{definition}[תבנית ריבועית]
\emph{תבנית ריבועית
על
$\FF^n$}
(עבור
$\FF$
כללי, שאינו בהכרח
$\RR$
או
$\CC$)
היא העתקה מהצורה
\begin{align*}
q \colon \FF^n &\to \FF \\
\pmat{x_1 \\ \vdots \\ x_n} &\mapsto \sum_{\substack{i,j \in \brs{n} \\ i \leq j}} c_{i,j} x_i x_j
\end{align*}
עבור
$c_{i,j} \in \FF$.
\end{definition}

\begin{remark}
מטריצה סימטרית
$A$
מגדירה תבנית ריבועית על ידי
$q\prs{\vec{x}} = \vec{x}^t A x$.
להיפך, אם
$q$
תבנית ריבועית נוכל לכתוב
\[q\prs{x} = \sum_{i \in \brs{n}} \sum_{j \in \brs{n}} a_{i,j} x_i x_j\]
כאשר
$a_{i,j} = a_{j,i}$.
אז נגדיר
$A \in M_n\prs{\mbb{F}}$
עם
\[\text{.} A_{i,j} = a_{i,j}\]
היא מקיימת
$q\prs{\vec{x}} = \vec{x}^t A x$.
זה מגדיר איזומורפיזם בין מרחבי המטריצות הסימטריות והתבניות הריבועיות.
למשל, במקום להגדיר
$g\prs{\pmat{x_1 \\ x_2}} = x_1^2 + x_2^2$,
נכתוב לפעמים את
$g$
בתור פולינום
$g\prs{x_1, x_2} = x_1^1 + x_2^2 \in \FF\brs{x_1, x_2}$
בשני משתנים.
\end{remark}

\begin{notation}
עבור
$A \in M_n\prs{\mbb{F}}$
סימטרית נסמן ב־%
$g_A$
את התבנית הריבועית
\[\text{.} g_A\prs{\vec{x}} = \vec{x}^t A \vec{x}\]
\end{notation}

\begin{exercise}
יהי
$\mbb{F} \ceq \quot{\mbb{Z}}{5\mbb{Z}}$
השדה בן
$5$
האיברים.
יתכן כי ראיתם את הסימון
$\mbb{Z}_5$
במקום זאת.
תהי
\begin{align*}
g\prs{x_1, x_2, x_3, x_4} &= 2 x_1 x_2 + 2 x_2 x_3 + 2 x_3 x_4 \in \mbb{F}\brs{x_1, x_2, x_3, x_4}
\end{align*}
תבנית ריבועית מעל
$\mbb{F}$.

\begin{enumerate}
\item מצאו מטריצה
$A \in M_4\prs{\mbb{F}}$
אלכסונית עבורה
$g = g_A$.

\item מצאו
$P \in M_4\prs{\mbb{F}}$
עבורה
$P A P^t$
אלכסונית.

\item האם יש
$P \in M_4\prs{\mbb{F}}$
עבורה
$P A P^t$
אלכסונית עם
$\pm 1, 0$
על האלכסון?

\item האם כל
$A \in M_4\prs{\mbb{F}}$
חופפת למטריצה עם
$\pm 1, 0$
על האלכסון?
\end{enumerate}
\end{exercise}

\begin{solution}
\begin{enumerate}
\item

לא מופיע ביטוי
$x_i^2$.
לכן נסתכל רק על המקדם של
$x_i x_j$
עבור
$i \neq j$.
כדי שיתקיים
$a_{i,j} = a_{j,i}$
 ניקח את חצי המקדם של
$x_i x_j$
 כפי שהוא כתוב (כי לא מופיע גם $x_j x_i$).
נקבל
\begin{align*}
\text{.} A &= \pmat{0 & 1 & 0 & 0 \\ 1 & 0 & 1 & 0 \\ 0 & 1 & 0 & 1 \\ 0 & 0 & 1 & 0}
\end{align*}

\item

נלכסן לפי שורה ועמודה כדי להביא את המטריצה לצורה אלכסונית.
כאשר יש
$0$
על כל האלכסון, נוסיף כפולה של שורה או עמודה אחרת, כדי שיופיע מספר שונה מאפס במקום ה־%
$\prs{1,1}$.
נקבל
\begin{align*}
\text{.} A &\mapsto \pmat{1 & 1 & 0 & 0 \\ 1 & 0 & 1 & 0 \\ 1 & 1 & 0 & 1 \\ 0 & 0 & 1 & 0} \mapsto
\pmat{2 & 1 & 1 & 0 \\ 1 & 0 & 1 & 0 \\ 1 & 1 & 0 & 1 \\ 0 & 0 & 1 & 0}
\end{align*}
כעת, נשתמש באיבר על האלכסון כדי לאפס את שאר האיברים, ונמשיך הלאה.
\begin{align*}
A &\mapsto \pmat{1 & 1 & 0 & 0 \\ 1 & 0 & 1 & 0 \\ 1 & 1 & 0 & 1 \\ 0 & 0 & 1 & 0} \mapsto
\pmat{2 & 1 & 1 & 0 \\ 1 & 0 & 1 & 0 \\ 1 & 1 & 0 & 1 \\ 0 & 0 & 1 & 0}
\\&\mapsto 
\pmat{2 & 0 & 0 & 0 \\ 1 & 2 & 3 & 0 \\ 1 & 3 & 2 & 1 \\ 0 & 0 & 1 & 0} \mapsto
\pmat{2 & 0 & 0 & 0 \\ 0 & 2 & 3 & 0 \\ 0 & 3 & 2 & 1 \\ 0 & 0 & 1 & 0}
\\&\mapsto
\pmat{2 & 0 & 0 & 0 \\ 0 & 2 & 0 & 0 \\ 0 & 3 & 0 & 1 \\ 0 & 0 & 1 & 0} \mapsto
\pmat{2 & 0 & 0 & 0 \\ 0 & 2 & 0 & 0 \\ 0 & 0 & 0 & 1 \\ 0 & 0 & 1 & 0}
\\&\mapsto
\pmat{2 & 0 & 0 & 0 \\ 0 & 2 & 0 & 0 \\ 0 & 0 & 1 & 1 \\ 0 & 0 & 1 & 0} \mapsto
\pmat{2 & 0 & 0 & 0 \\ 0 & 2 & 0 & 0 \\ 0 & 0 & 2 & 1 \\ 0 & 0 & 1 & 0} \\ \text{.} \hphantom{A} &\mapsto
\pmat{2 & 0 & 0 & 0 \\ 0 & 2 & 0 & 0 \\ 0 & 0 & 2 & 0 \\ 0 & 0 & 1 & 2} \mapsto
\pmat{2 & 0 & 0 & 0 \\ 0 & 2 & 0 & 0 \\ 0 & 0 & 2 & 0 \\ 0 & 0 & 0 & 2}
\end{align*}
ניקח את
$P$
להיות המכפלה של 5 המטריצות האלמנטריות שמתאימות לפעולות הדירוג.

\item

כן. נוכל בדירוג להוסיף
$3$
פעמים את השורה/עמודה הרלוונטית כדי לקבל
$3\cdot2 \equiv 1 \mod{5}$
על האלכסון.

\begin{align*}
A &\mapsto \pmat{3 & 1 & 0 & 0 \\ 1 & 0 & 1 & 0 \\ 3 & 1 & 0 & 1 \\ 0 & 0 & 1 & 0} \mapsto
\pmat{1 & 1 & 3 & 0 \\ 1 & 0 & 1 & 0 \\ 3 & 1 & 0 & 1 \\ 0 & 0 & 1 & 0}
\\&\mapsto 
\pmat{1 & 0 & 0 & 0 \\ 1 & 4 & 3 & 0 \\ 3 & 3 & 1 & 1 \\ 0 & 0 & 1 & 0} \mapsto
\pmat{1 & 0 & 0 & 0 \\ 0 & 4 & 3 & 0 \\ 0 & 3 & 1 & 1 \\ 0 & 0 & 1 & 0}
\\&\mapsto
\pmat{1 & 0 & 0 & 0 \\ 0 & 3 & 4 & 0 \\ 0 & 1 & 3 & 1 \\ 0 & 1 & 0 & 0} \mapsto
\pmat{1 & 0 & 0 & 0 \\ 0 & 1 & 3 & 1 \\ 0 & 3 & 4 & 0 \\ 0 & 1 & 0 & 0}
\\&\mapsto
\pmat{1 & 0 & 0 & 0 \\ 0 & 1 & 0 & 0 \\ 0 & 3 & 0 & 2 \\ 0 & 1 & 2 & 4} \mapsto
\pmat{1 & 0 & 0 & 0 \\ 0 & 1 & 0 & 0 \\ 0 & 0 & 0 & 2 \\ 0 & 0 & 2 & 4} \\&\mapsto
\ldots \mapsto
\pmat{1 & 0 & 0 & 0 \\ 0 & 1 & 0 & 0 \\ 0 & 0 & 4 & 2 \\ 0 & 0 & 2 & 0}  \\&\mapsto
\ldots \mapsto \pmat{1 & 0 & 0 & 0 \\ 0 & 1 & 0 & 0 \\ 0 & 0 & 4 & 0 \\ 0 & 0 & 0 & 4} \\\text{.}\hphantom{A}
&= \pmat{1 & 0 & 0 & 0 \\ 0 & 1 & 0 & 0 \\ 0 & 0 & -1 & 0 \\ 0 & 0 & 0 & -1}
\end{align*}

\item
לא.
למשל
$A = \pmat{3 & 0 & 0 & 0 \\ 0 & 1 & 0 & 0 \\ 0 & 0 & 1 & 0 \\ 0 & 0 & 0 & 1}$
כי
$\det\prs{A} = 3$
אבל אם
$\det\prs{A} = 0$
נקבל
$3 = \det\prs{A} = a^2 \cdot 0 = 0$
בסתירה, ואם
$\det\prs{A} \in \set{\pm 1}$
נקבל
$3 = \det\prs{A} = \pm a^2$.
לא יתכן
$3 = a^2$
כי
$3$
אינו ריבוע ב־%
$\quot{\mbb{Z}}{5\mbb{Z}}$,
ולא יתכן
$3 = -a^2$
כי אז
$2 = -3 = a^2$
ואילו גם
$2$
אינו ריבוע ב־%
$\quot{\mbb{Z}}{5\mbb{Z}}$.
\end{enumerate}
\end{solution}

\begin{exercise}
תהי
\[\text{.} g\prs{x_1, x_2, x_3} = 2 x_1 x_2 + 2 x_1 x_3 - x_2^2 - i x_3^2 \in \mbb{C}\brs{x_1, x_2, x_3}\]

\begin{enumerate}
\item מצאו מטריצה
$A$
עבורה
$g = g_A$.

\item הראו כי
$g$
אינה מנוונת.

\item מצאו בסיס
$B$
של
$\mbb{C}^3$
עבורו
$\brs{g}_B = I_3$.
\end{enumerate}
\end{exercise}

\begin{solution}
\begin{enumerate}
\item כמו מקודם, נכתוב
\[\text{.} A = \pmat{0 & 1 & 1 \\ 1 & -1 & 0 \\ 1 & 0 & -i}\]

\item
מתקיים
\begin{align*}
\text{.} \pmat{a,b,c} A \pmat{x\\y\\z} &= \pmat{a,b,c}\pmat{y+z\\x-y\\x-iz} = a\prs{y+z} + b\prs{x-y} + c\prs{x-iz}
\end{align*}
נניח כי
$\pmat{a,b,c} A \pmat{x\\y\\z} = 0$
לכל
$\pmat{x\\y\\z} \in \mbb{C}^3$
ונראה כי
$\pmat{a,b,c} = \prs{0,0,0}$.

אם ניקח
$\prs{x\\y\\z} = \pmat{1\\1\\-i}$
נקבל
\[0 = a\prs{1-i} + b\prs{1-1} + c\prs{1+i^2} = a\prs{1-i}\]
לכן
$a = 0$.
אם ניקח
$\pmat{x\\y\\z} = \pmat{-i\\1\\-1}$
נקבל
\[0 = b\prs{-i-1}\]
ולכן
$b = 0$,
ואם ניקח
$\pmat{x\\y\\} = \pmat{1\\1\\-1}$
נקבל
\[0 = c\prs{1+i}\]
ולכן
$c = 0$.
בסך הכל
$\pmat{a,b,c} = \pmat{0,0,0}$,
כנדרש.

\item נבצע דירוג לפי שורה ועמודה על
$A$

\begin{align*}
A &\mapsto \pmat{1 & 0 & 1 \\ -1 & 1 & 0 \\ 0 & 1 & - i} \mapsto
\pmat{-1 & 1 & 0 \\ 1 & 0 & 1 \\ 0 & 1 & - i}
\\&\mapsto
\pmat{-1 & 0 & 0 \\ 1 & 1 & 1 \\ 0 & 1 & - i} \mapsto
\pmat{-1 & 0 & 0 \\ 0 & 1 & 1 \\ 0 & 1 & - i}
\\&\mapsto
\pmat{-1 & 0 & 0 \\ 0 & 1 & 0 \\ 0 & 1 & -1 - i}
\mapsto
\pmat{-1 & 0 & 0 \\ 0 & 1 & 0 \\ 0 & 0 & -1 - i}
\\&\mapsto
\pmat{1 & 0 & 0 \\ 0 & 1 & 0 \\ 0 & 0 & -1 - i}
\\&\mapsto
\pmat{1 & 0 & 0 \\ 0 & 1 & 0 \\ 0 & 0 & 1}
\end{align*}

כאשר בשני השלבים האחרונים חילקנו ב־%
$\sqrt{-1}$
וב־%
$\sqrt{-1-i}$
בהתאמה, כל אחד מהם בשורה ובעמודה.
נקבל כי עבור
\[P = \pmat{1 & 0 & 0 \\ 0 & 1 & 0 \\ 0 & 0 & \frac{1}{\sqrt{-1-i}}} \pmat{-i & 0 & 0 \\ 0 & 1 & 0 \\ 0 & 0 & 1} \pmat{1 & 0 & 0 \\ 0 & 1 & 0 \\ 0 & -1 & 1} \pmat{1 & 0 & 0 \\ 1 & 1 & 0 \\ 0 & 0 & 1} \pmat{0 & 1 & 0 \\ 1 & 0 & 0 \\ 0 & 0 & 0} = \pmat{0 & -i & 0 \\ 1 & 1 & 0 \\ -\frac{1}{\sqrt{-i-1}} & -\frac{1}{\sqrt{-i-1}} & \frac{1}{\sqrt{-i-1}}}\]
מתקיים
$PAP^t = I_3$.
נכתוב
\[B = \prs{\pmat{0\\-i\\0}, \pmat{1\\1\\0}, \pmat{-\frac{1}{\sqrt{-i-1}}\\-\frac{1}{\sqrt{-i-1}}\\\frac{1}{\sqrt{-i-1}}}}\]
כאשר הוקטורים ב־%
$B$
הם עמודות
$P^t$.
אז
$P^t = P^B_E$
ונקבל
\[\text{.} I_3 = P A P^t = \prs{P^B_E}^t \brs{g}_E P^B_E = \brs{g}_B\]
\end{enumerate}
\end{solution}

\end{document}

\printbibliography
\end{document}
