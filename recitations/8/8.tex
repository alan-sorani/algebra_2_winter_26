\documentclass[a4paper,10pt,twoside,openany]{article}

\usepackage[lang=hebrew]{maths}
\usepackage{polynom}
\usepackage{hebrewdoc}
\usepackage{stylish}
\usepackage{lipsum}
\let\bs\blacksquare

\setlength{\parindent}{0pt}
\newcommand{\Std}{\mrm{Std}}

%%%%%%%%%%%%
% Styling %
%%%%%%%%%%%%

\usepackage{enumitem}

%%%%%%%%%%%%%
% Counters  %
%%%%%%%%%%%%%

\setcounter{section}{0}     
            
%BIBLIOGRAPHY
\usepackage[
backend=biber,
style=alphabetic,
]{biblatex}
\addbibresource{bibliography.bib} %Imports bibliography file

\title{
אלגברה ב' (01040168) - חורף 2026
\\
תרגול 8 - נילפוטנטיות ומרחבים עצמיים מוכללים
\\
אלן סורני
\\
הרשימות עודכנו לאחרונה בתאריך ה־%
\today
}
\date{}

\begin{document}
\maketitle

\section{נילפוטנטיות}

\begin{definition}[העתקה נילפוטנטית]
העתקה
$T \in \endo_{\mbb{F}}\prs{V}$
נקראת
\emph{נילפוטנטית}
אם יש
$k \in \mbb{N}_+$
עבורו
$T^k = 0$.
ה־%
$k$
המינימלי הזה נקרא
\emph{אינדקס הנילפוטנטיות של
$T$}.
\end{definition}

\begin{remark}
נניח לאורך התרגול כי
$V$
מרחב וקטורי סוף־מימדי.
\end{remark}

\begin{exercise}
יהי
$V$
מרחב וקטורי ממימד
$n \in \mbb{N}_+$
ותהי
$T \in \endo_{\mbb{F}}\prs{V}$
נילפוטנטית מאינדקס
$k$.
הראו כי
$k \leq n$.
\end{exercise}

\begin{solution}
\begin{description}
\item[דרך 1:]

ראינו כי
$\ker\prs{T^k} \leq \ker\prs{T^n}$,
לכן אם
$T^k = 0$
נקבל
$\ker\prs{T^k} = V \leq \ker\prs{T^n}$
כלומר
$\ker\prs{T^n} = V$
כלומר
$T^n = 0$.
ממינימליות
$k$
נקבל
$k \leq n$.

\item[דרך 2:]
ראינו כי
\begin{align*}
\text{.} \ker\prs{T} \leq \ker\prs{T^2} \leq \ldots \leq \ker\prs{T^k} = V
\end{align*}
אם נראה שמתקיים
$\ker\prs{T^i} \neq \ker\prs{T^{i+1}}$
לכל
$i \leq k$
נקבל כי
$\dim \ker\prs{T^{i+1}} > \dim \ker\prs{T^i}$
וכיוון ש־%
$\dim \ker\prs{T} \geq 1$
נקבל באינדוקציה כי
$\dim \ker\prs{T^i} \geq i$
לכל
$i \leq k$
ובפרט
$\dim \ker\prs{T^k} \geq k$.
אבל
$\dim \ker\prs{T^k} = \dim\prs{V} = n$,
לכן
$n \geq k$,
כנדרש.
\end{description}

\end{solution}

\begin{exercise}
יהי
$\mbb{F}$
סגור אלגברית ותהי
$T \in \endo_{\mbb{F}}\prs{V}$.
הראו ש־%
$T$
 נילפוטנטית אם ורק אם הערך העצמי היחיד שלה הוא
$0$.
מצאו דוגמא נגדית כאשר
$\mbb{F}$
אינו סגור אלגברית.
\end{exercise}

\begin{solution}
נניח כי
$\lambda \in \mbb{F}$
ערך עצמי של
$T \in \endo_{\mbb{F}}\prs{V}$
נילפוטנטית.
יש
$k \in \mbb{N}_+$
עבורו
$T^k = 0$.
יהי
$v \in V$
וקטור עצמי של
$T$
עם ערך עצמי
$\lambda$.
אז
$T^k v = \lambda^k v = 0$
לכן
$\lambda = 0$.

נניח כי
$S \in \endo_{\mbb{F}}\prs{V}$
עם ערך עצמי יחיד
$0$.
כיוןן ש־%
$\mbb{F}$
סגור־אלגברית,
יש בסיס
$B$
כך ש־%
$\brs{S}_B$
משולשת עליונה, ונקבל שיש על האלכסון שלה אפסים כי
$0$
ערך עצמי יחיד של
$S$.
נכתוב
$B = \prs{v_1, \ldots, v_n}$
וגם
$V_i \ceq \spn\prs{v_1, \ldots, v_i}$
לכל
$i \in [n]$.
נסיק כי
$S V_i \subseteq V_{i-1}$
לכל
$i \in [n]$,
וגם
$S V_1 = 0$.
אז
\[S^n V = S^n V_n = S^{n-1} V_{n-1} = \ldots = S V_1 = 0\]
כלומר
$S^n = 0$.

עבור
$\mbb{F} = \mbb{R}$
ניקח את
$T = L_A$
עבור
\[A = \pmat{0 & 1 & 0 \\ -1 & 0 & 0 \\ 0 & 0 & 0}\]
שהערך העצמי הממשי היחיד שלה הוא
$0$
(כי
$\pm i \notin \mbb{R}$)
אבל
$A^4 = \pmat{1 & 0 & 0 \\ 0 & 1 & 0 \\ 0 & 0 & 0}$
לכן
$T^4 = e_1 = e_1$.
כלומר
$T^4 \neq 0$
ולכן
$T$
אינה נילפוטנטית.
\end{solution}

\begin{examples}
\begin{description}
\item[העתקות נילפוטנטיות:]
ההעתקה
$L_A$
נילפוטנטית עבור כל אחת מהמטריצות הבאות.
\begin{itemize}
\item $A = \pmat{0 & 1 & 0 \\ 0 & 0 & 1 \\ 0 & 0 & 0}$.
\item כל $A$
משולשת עליונה עם
$0$
על כל האלכסון הראשי, כפי שראינו בתרגיל האחרון.
\item $A = \pmat{0 & 0 & 0 \\ 0 & 0 & 1 \\ 1 & 0 & 0}$.
\end{itemize}
\item[העתקות שאינן נילפוטנטיות:]
עבור המטריצות הבאות,
$L_A$
אינה נילפוטנטית.
\begin{itemize}
\item $A = \pmat{1 & 0 & 0 \\ 0 & 1 & 0 \\ 0 & 0 & 0}$.
מתקיים
$Ae_1 = e_1$
ולכן
$1$
ערך עצמי ולא יתכן כי
$L_A$
נילפוטנטית.
\item $A = \pmat{1 & 1 & 1 \\ 1 & 1 & 1 \\ 1 & 1 & 1}$.
מתקיים
$A\prs{e_1 + e_2 + e_3} = 3 \prs{e_1 + e_2 + e_3}$
לכן
$3$
ערך עצמי, ולא יתכן כי
$L_A$
נילפוטנטית.
\item $A = \pmat{0 & 1 & 0 \\ -1 & 0 & 0 \\ 0 & 0 & 0} \in M_3\prs{\mbb{R}}$.
מתקיים
$A^4 = \pmat{1 & 0 & 0 \\ 0 & 1 & 0 \\ 0 & 0 & 0} \neq 0$
לכן
$L_A^4 \neq 0$
וראינו שאינדקס הנילפוטנטיות חייב להיות לכל היותר
$3$
במקרה זה. לכן
$L_A$
אינה נילפוטנטית.
\end{itemize}
\end{description}
\end{examples}

\begin{exercise}
הראו כי
$T \in \endo_{\mbb{F}}\prs{V}$
נילפוטנטית אם ורק אם לכל
$v \in V$
יש
$k \in \mbb{N}_+$
עבורו
$T^k v = 0$.
\end{exercise}

\begin{solution}
נניח כי
$T$
נילפוטנטית מאינדקס
$k$.
אז
$T^k v = 0 v = 0$
לכל
$v \in V$.

נניח להיפך כי לכל
$v \in V$
יש
$k \in \mbb{N}_+$
עבורו
$T^k v = 0$.
אז
$v \in \ker\prs{T^k}$.
אבל, ראינו כי
\[\ker\prs{T} \leq \ker\prs{T^2} \leq \ker\prs{T^3} \leq \ldots\]
מתייצבת, לכן יש
$m \in \mbb{N}_+$
עבורו
$\ker\prs{T^k} \subseteq \ker\prs{T^m}$
לכל
$k \in \mbb{N}_+$.
נקבל
$v \in \ker\prs{T^m}$
לכל
$v \in V$,
לכן
$T^m = 0$.
\end{solution}

\begin{exercise}
הראו שצירוף לינארי של העתקות נילפוטנטיות מתחלפות הוא נילפוטנטי.
מצאו דוגמה נגדית עבור העתקות נילפוטנטיות שאינן מתחלפות.
\end{exercise}

\begin{solution}
יהיו
$T_1. T_2 \in \endo_{\mbb{F}}\prs{V}$
נילפוטנטיות מאינדקסים
$k_1, k_2$
בהתאמה וכך שמתקיים
$T_1 T_2 = T_2 T_1$.
יהי
$\alpha \in \mbb{F}$.

מתקיים
\begin{align*}
\prs{\alpha T_1}^{k_1} = \alpha^{k_1} T_1^{k_1} = \alpha^{k_1} 0 = 0
\end{align*}
לכן
$\alpha T_1$
נילפוטנטית מאינדקס קטן או שווה ל־%
$k$.
כעת
\begin{align*}
\text{.} \prs{T_1 + T_2}^{k_1 + k_2 + 1} &= \sum_{i=0}^{k_1 + k_2} \binom{k_1 + k_2}{i} T_1^i T_2^{k_1 + k_2 - i}
\end{align*}
כאשר
$i \geq k_1$
מתקיים
$T_1^i = 0$
וכאשר
$i < k_1$
מתקיים
$k_1 + k_2 - i \geq k_2$
לכן
$T_2^{k_1 + k_2 - i} = 0$.
נקבל בסך הכל כי
$T_1 + T_2$
נילפוטנטית מאינדקס קטן או שווה
$k_1 + k_2$.

בלי ההנחה שההעתקות מתחלפות, אפשר לקחת
\begin{align*}
T_1 &= \pmat{0 & 1 \\ 0 & 0}, \\
T_2 &= \pmat{0 & 0 \\ 1 & 0}
\end{align*}
כאשר שתיהן נילפוטנטיות מאינדקס 2, אבל
\[T_1 + T_2 = \pmat{0 & 1 \\ 1 & 0}\]
הפיכה ולכן אינה נילפוטנטית (אין לה ערך עצמי
$0$).
\end{solution}

\begin{exercise}[לא עשינו בתרגול]
תהי
$T \in \endo_{\mbb{F}}\prs{V}$
נילפוטנטית.
הראו כי
$\det\prs{T} = \tr\prs{T} = 0$.
\end{exercise}

\begin{solution}
ראינו בהרצאה שיש בסיס
$B$
של
$V$
כך ש־%
$\brs{T}_B$
משולשת עליונה.
כיוון שכל הערכים העצמיים של
$T$
שווים לאפס, נקבל שיש
$0$
על האלכסון.
אז
$\det\prs{T} = \det\prs{\brs{T}_B} = 0$
כמכפלת איברי האלכסון וגם
$\tr\prs{T} = \tr\prs{\brs{T}_B} = 0$
כסכום איברי האלכסון.
\end{solution}

\begin{exercise}
תהי
$T$
נילפוטנטית מאינדס
$k$.
הראו שהעתקות
$\prs{\id_V \pm T}$
הפיכות ומצאו את ההופכיות שלהן.
\end{exercise}

\begin{solution}
אנו יודעים כי
\[\sum_{k \in \mbb{N}} r^k = \frac{1}{1-r}\]
עבור
$r < 0$.
נרצה אם כן שההופכית של
$\id_V - T$
תהיה
$\id_V + T + \ldots + T^{k-1}$.
אכן,
\begin{align*}
\prs{\id_V - T}\prs{\id_V + T + \ldots + T^{k-1}} &= \sum_{i = 0}^{k-1} T^{i} - \sum_{i = 1}^k T^^i
\\&= \id_V - T^k
\\&= \id_V - 0
\\ \text{.} \hphantom{\prs{\id_V - T}\prs{\id_V + T + \ldots + T^{k-1}}} &= \id_V
\end{align*}

כעת, אם
$T$
נילפוטנטית מאינדקס
$k$
גם
$-T$
נילפוטנטית מאינדקס
$k$,
לכן ההופכית של
$\id_V + T = \id_V - \prs{-T}$
היא
\[\text{.} \id_V - T + T^2 - T^3 + \ldots + \prs{-1}^{k-1} T^{k-1}\]
\end{solution}

\begin{exercise}[לא עשינו בתרגול]
הראו כי
\begin{align*}
D \colon \mbb{F}_n\brs{x} &\to \mbb{F}_n\brs{x} \\
p &\mapsto p'
\end{align*}
נילפוטנטית.
\end{exercise}

\begin{solution}
לכל
$p \in \mbb{F}_n\brs{x}$
מתקיים
$\deg_{\mbb{F}}\prs{p} \leq n$
לכן
$D^{n-1}\prs{p}$
פולינום קבוע, ולכן
$D^n p = 0$.
\end{solution}

\begin{exercise}[לא עשינו בתרגול]
הראו כי
\begin{align*}
D \colon \mbb{F}\brs{x} &\to \mbb{F}\brs{x} \\
p &\mapsto p'
\end{align*}
אינה נילפוטנטית.
\end{exercise}

\begin{solution}
נניח בשלילה כי
$D$
נילפוטנטית מאינדקס
$k$.
אז
$D^k x^{k+1} = 0$,
אבל
\begin{align*}
D^k x^{k+1} &= \prs{k+1} D^{k-1} x^k
\\&= \ldots
\\&= \prs{k+1}!
\\&\neq 0
\end{align*}
בסתירה.
\end{solution}

\begin{exercise}[לא עשינו בתרגול]
יהי
$V$
מרחב וקטורי ממימד
$n \in \mbb{N}_+$
ותהי
$T \in \endo_{\mbb{F}}\prs{V}$
נילפוטנטית מאינדקס
$k$.
לכל
$i \in [k]$
נסמן
$n_i \ceq \dim \ker\prs{T^i}$.
הראו שמתקיים
\begin{align*}
\text{.} 0 < n_1 < n_2 < \ldots < n_{k-1} < n_k = n
\end{align*}
\end{exercise}

\begin{solution}
מתקיים
$\ker\prs{T^k} = \ker\prs{0} = V$
לכן
$n_k = \dim\prs{V} = n$.
אם
$n_1 = 0$
נקבל
$\dim \ker\prs{T} = 0$
לכן
$T$
חד־חד ערכית ולכן הפיכה, בסתירה לנילפוטנטיות.

אם
$n_i = n_{i+1}$
עבור
$i \in [k-1]$
נקבל
$\ker\prs{T^i} = \ker\prs{T^{i+1}}$.
אבל, ראינו שבמקרה זה
$\ker\prs{T^i} = \ker\prs{T^j}$
לכל
$j \geq i$.
בפרט
$\ker\prs{T^i} = \ker\prs{T^k} = V$,
כלומר
$i = k$
בסתירה להנחה
$i \in \brs{k-1}$.
\end{solution}

\begin{exercise}
יהי
$V$
מרחב וקטורי ממימד
$n \in \mbb{N}_+$
עם בסיס
$B \prs{v_1, \ldots, v_n}$.
תהי
\begin{align*}
T \colon V &\to V \\
v_1 &\mapsto 0 \\
\text{.} \forall i > 1: v_i &\mapsto v_{i-1}
\end{align*}
הראו כי
$T$
נילפוטנטית מאינדקס
$n$
וכיתבו את
$\brs{T}_B$.
\end{exercise}

\begin{solution}
מתקיים
\[\brs{T}_B = \pmat{0 & 1 & 0 & \ldots & 0 \\ 0 & 0 & 1 & \ddots & \vdots \\ \vdots & \ddots & 0 & \ddots & 0 \\ 0 & 0 & \ddots & 0 & 1 \\ 0 & 0 & \cdots & 0 & 0}\]
וזאת משולשת עליונה עם
$0$
על האלכסון, לכן כפי שתיארנו מקודם
$T^n = 0$.
מאידך,
$T^{n-1}v_n = v_1 \neq 0$
לכן האינדקס של
$T$
שווה
$n$.
\end{solution}

\newpage

\section{מרחבים עצמיים מוכללים}

\begin{definition}[בלוק ז'ורדן]
עבור
$\lambda \in \FF$
ועבור
$m \in \NN_+$
נסמן
\begin{align*}
\text{.} J_m\pmat{\lambda} \coloneqq \pmat{
\lambda & 1 & 0 & \cdots & 0 \\
0 & \lambda & 1 & \ddots & \vdots \\
\vdots & \ddots & \ddots & \ddots & 0 \\ 
\vdots & \ddots & \ddots & \lambda & 1 \\
0 & \cdots & \cdots & 0 & \lambda
}
\end{align*}
נסמן גם
$J_m \coloneqq J_m\prs{0}$.
\end{definition}

\begin{definition}[מטריצת ז'ורדן]
מטריצה מהצורה
\[\diag\prs{J_{m_1}\prs{\lambda_1}, \ldots, J_{m_k}\prs{\lambda_k}}\]
נקראת
\emph{מטריצת ז'ורדן}.
\end{definition}

\begin{definition}[בסיס ז'ורדן]
בסיס
$B$
עבורו
$\brs{T}_B$
מטריצת ז'ורדן נקרא
\emph{בסיס ז'ורדן עבור
$T$}.
\end{definition}

\begin{theorem}[משפט ז'ורדן]
יהי
$V$
מרחב וקטורי סוף־מימדי מעל שדה סגור אלגברית
$\FF$
ויהי
$T \in \End_\FF\prs{V}$.
קיים בסיס
ז'ורדן עבור
$T$.

בנוסף, $\brs{T}_B$ יחידה כמטריצת ז'ורדן מייצגת של $T$, עד כדי שינוי סדר הבלוקים.
\end{theorem}

\begin{remark}
כדי לדבר על בסיסי ז'ורדן, נשים לב כי אם
$B$
בסיס ז'ורדן כנ"ל, נוכל לכתוב
\begin{align*}
B &= \prs{v_{1,1}, \ldots, v_{1,m_1}, v_{2,1}, \ldots, v_{2,m_2}, \ldots, v_{k,1}, \ldots, v_{k,m_k}}
\\&= \prs{v_{1,1}, \ldots, v_{1,m_1}} * \prs{v_{2,1}, \ldots, v_{2,m_2}} * \ldots * \prs{v_{k,1}, \ldots, v_{k,m_k}}
\end{align*}
כאשר לכל
$i \in \brs{k}$
ולכל
$j \in \brs{m_i}$
מתקיים
\begin{align*}
\text{.} \prs{T-\lambda \id_V} \prs{v_{i,j}} = \fcases{
v_{i,j-1} & j > 1 \\
0 & j = 1
}
\end{align*}
\end{remark}

\begin{definition}[שרשרת ז'ורדן]
יהי
$V$
מרחב וקטורי מעל
$\FF$
ויהי
$T \in \End_\FF\prs{V}$.

\emph{שרשרת ז'ורדן}
של
$T$
עם ערך עצמי
$\lambda \in \FF$
היא קבוצה סדורה
$C \coloneqq \prs{v_1, \ldots, v_m} \subseteq V$
כך שמתקיים
\begin{align*}
\text{.} \prs{T - \lambda \id_V} \prs{v_j} = \fcases{
v_{j-1} & j > 1 \\
0 & j = 1
}
\end{align*}
באופן שקול, מתקיים
\begin{align*}
C = \prs{\prs{T-\lambda \id_V}^{m-1} \prs{v_m} , \prs{T - \lambda \id_V}^{m-2} \prs{v_m}, \ldots, v_m}
\end{align*}
וגם
$\prs{T - \lambda \id_V}^m \prs{v_m} = 0$.
\end{definition}

\begin{remark}
משפט ז'ורדן אומר שאם
$T$
אופרטור על מרחב וקטורי סוף־מימדי מעל שדה סגור אלגברית
$\FF$
אז קיים בסיס של
$B$
המורכב משרשראות ז'ורדן של
$T$.
\end{remark}

\begin{definition}[מרחב עצמי מוכלל]
יהי
$V$
מרחב וקטורי מעל
$\FF$
ממימד
$n \in \NN_+$,
ויהי
$T \in \End_\FF\prs{V}$.

\emph{המרחב העצמי המוכלל של
$T$
עבור ערך
$\lambda \in \FF$}
הוא
\begin{align*}
\text{.} V'_{\lambda, T} &\coloneqq \set{v \in V}{\exists k \in \NN: \prs{T - \lambda}^k\prs{v} = 0}
\end{align*}
\end{definition}

\begin{exercise}
יהי
$V$
מרחב וקטורי סוף־מימדי מעל שדה סגור אלגברית
$\FF$,
ויהי
$T \in \End_\FF\prs{V}$.
יהיו
$\lambda, \mu \in \FF$.
אז
\begin{align*}
\text{.} V'_{\lambda - \mu, T} = V'_{\lambda, T + \mu \id_V}
\end{align*}
\end{exercise}

\begin{solution}
יהי
$v \in V'_{\lambda - \mu, T}$.
לפי ההגדרה, קיים
$k \in \NN_+$
עבורו
\[\text{.} \prs{T - \prs{\lambda - \mu} \id_V}^k \prs{v} = 0\]
אז
\begin{align*}
0 &= \prs{T - \lambda \id_V + \mu \id_V}^k \prs{v}
\\&= \prs{\prs{T + \mu \id_V} - \lambda \id_V}^k\prs{v}
\end{align*}
ושוב לפי ההגדרה נקבל כי
\[\text{.} v \in V'_{\lambda, T + \mu \id_V}\]
לכן
\[\text{.} V'_{\lambda - \mu, T} \subseteq V'_{\lambda, T + \mu \id_V}\]

עבור ההכלה השנייה, נסמן
\begin{align*}
\lambda' &\coloneqq \lambda - \mu \\
\mu' &\coloneqq -\mu \\
T' &\coloneqq T + \mu
\end{align*}
ואז
$\lambda = \lambda' + \mu = \lambda' - \mu'$
וגם
$T = T' - \mu = T' + \mu'$
ונקבל מהכיוון הראשון כי
\begin{align*}
\text{,} V'_{\lambda, T + \mu \id_V} = V'_{\lambda' - \mu', T'} \subseteq V'_{\lambda', T' + \mu' \id_V} = V'_{\lambda - \mu, T}
\end{align*}
כנדרש.
\end{solution}

\begin{exercise}\label{exercise:jordan-nilpotency-index}
יהי
$T$
אופרטור על מרחב וקטורי
$V$
ממימד
$n \in \NN_+$
מעל שדה סגור אלגברית
$\FF$,
ויהי
$\lambda \in \FF$
ערך עצמי של
$T$.
אז
\[\text{.} \prs{T - \lambda \id_V}^{r^T_a\prs{\lambda}} = \prs{T - \lambda \id_V}^n\]
\end{exercise}

\printbibliography
\end{document}
