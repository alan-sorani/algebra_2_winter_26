\documentclass[a4paper,10pt,twoside,openany]{article}

\usepackage[lang=hebrew]{maths}
\usepackage{polynom}
\usepackage{hebrewdoc}
\usepackage{stylish}
\usepackage{lipsum}
\let\bs\blacksquare

\setlength{\parindent}{0pt}
\newcommand{\Std}{\mrm{Std}}
\DeclareMathOperator{\Char}{char}

%%%%%%%%%%%%
% Styling %
%%%%%%%%%%%%

\usepackage{enumitem}

%%%%%%%%%%%%%
% Counters  %
%%%%%%%%%%%%%

\setcounter{section}{0}     
            
%BIBLIOGRAPHY
\usepackage[
backend=biber,
style=alphabetic,
]{biblatex}
\addbibresource{bibliography.bib} %Imports bibliography file

\title{
אלגברה ב' (01040168) - חורף 2026
\\
תרגול 13 - שאלות ממבחנים
\\
אלן סורני
\\
הרשימות עודכנו לאחרונה בתאריך ה־%
\today
}
\date{}

\begin{document}
\maketitle

\section{תרגילים ממבחנים}

\begin{exercise}
\begin{enumerate}
\item הראו כי ל־%
$A \ceq \pmat{0&1\\1&1}, B \ceq \pmat{1&1\\1&0} \in M_2\prs{\mbb{Z}_2}$
אין צורת ז'ורדן ב־%
$M_2\prs{\mbb{Z}_2}$.

\item הראו כי כל מטריצה אחרת ב־%
$M_2\prs{\mbb{Z}_2}$
דומה למטריצת ז'ורדן.
\end{enumerate}
\end{exercise}

\begin{solution}
\begin{enumerate}
\item
מתקיים
\begin{align*}
\det\prs{A}, \det\prs{B} &= -1 = 1 \\
\text{.} \tr\prs{A}, \tr\prs{B} &=  1
\end{align*}
מהשוויון הראשון הערך העצמי היחיד של
$A,B$
הוא
$1$.
אז אם יש צורת ז'ורדן היא
$\pmat{1 & 0 \\ 0 & 1}$
או
$\pmat{1 & 1 \\ 0 & 1}$,
אך בשני המקרים האלו העקבה היא
$2 = 0$,
בסתירה.

\item
תהי
$C \in M_2\prs{\mbb{Z}_2} \setminus \set{A,B}$.
אם
$\tr\prs{C} = 1$
נקבל
\[\text{.} C \in \set{\pmat{0 & 1 \\ 0 & 1}, \pmat{0 & 0 \\ 1 & 1}, \pmat{1 & 1 \\ 0 & 0}, \pmat{1 & 0 \\ 1 & 0}, \pmat{1 & 0 \\ 0 & 0}, \pmat{0 & 0 \\ 0 & 1}}\]
אז
\begin{enumerate}
\item אם
$C = \pmat{0 & 0 \\ 1 & 1}$
נכתוב
$B = \pmat{e_1 + e_2, e_2}$
ואז
\begin{align*}
C \prs{e_1 + e_2} &= C e_1 + C e_2 = e_2 + e_2 = 0 \\
C e_2 &= e_2
\end{align*}
לכן
$B$
בסיס ז'ורדן של
$C$
וצורת ז'ורדן לפיו היא
$\pmat{0 & 0 \\ 0 & 1}$.

\item אם
$C = \pmat{0 & 1 \\ 0 & 1}$
היא המטריצה המשוחלפת של
המטריצה הקודמת, ולכן לפי תרגיל משיעורי הבית יש לה צורת ז'ורדן והיא זהה לזאת הקודמת.

\item אם
$C = \pmat{1 & 1 \\ 0 & 0}$
נכתוב
$B \ceq \pmat{e_1 + e_2, e_1}$
ואז
$B$
בסיס ז'ורדן כמו במקרה הראשון.

\item אם
$C = \pmat{1 & 0 \\ 1 & 0}$
היא המטריצה המשוחלפת של המטריצה הקודמת, ולכן לפי תרגיל משיעורי הבית יש לה צורת ז'ורדן והיא זהה לזאת הקודמת.

\item שאר המטריצות כבר בצורת ז'ורדן.
\end{enumerate}

אם
$\tr\prs{C} = 0$
נקבל
\[\text{.} C \in \set{\pmat{0 & 1 \\ 1 & 0}, \pmat{1 & 1 \\ 1 & 1}, \pmat{0 & 0 \\ 1 & 0}, \pmat{1 & 0 \\ 1 & 1}, \pmat{0 & 1 \\ 0 & 0}, \pmat{1 & 1 \\ 0 & 1}, \pmat{0 & 0 \\ 0 & 0}, \pmat{1 & 0 \\ 0 & 1}}\]
אז
\begin{enumerate}
\item אם
$C = \pmat{0 & 1 \\ 1 & 0}$
נכתוב
$B \ceq \prs{e_1 + e_2, e_1}$
ואז
\begin{align*}
C\prs{e_1 + e_2} &= C e_1 + C e_2 = e_2 + e_1 = e_1 + e_2 \\
C e_1 &= e_2 = e_1 + e_2 - e_1 = \prs{e_1 + e_2} + e_1
\end{align*}
ולכן
$B$
בסיס ז'ורדן וצורת ז'ורדן היא
$\pmat{1 & 1 \\ 0 & 1}$.

\item אם
$C = \pmat{1 & 1 \\ 1 & 1}$
נכתוב
$B = \pmat{e_1 + e_2, e_1}$
ואז
\begin{align*}
C \prs{e_1 + e_2} &= 2\prs{e_1 + e_2} = 0 \\
C e_1 &= e_1 + e_2
\end{align*}
ולכן
$B$
בסיס ז'ורדן וצורת ז'ורדן היא
$\pmat{1 & 1 \\ 0 & 1}$.

\item אם
$C = \pmat{0 & 0 \\ 1 & 0}$
או
$C = \pmat{1 & 0 \\ 1 & 1}$
היא מטריצה משוכלפת של מטריצה ז'ורדן ולכן יש לה צורת ז'ורדן עם בסיס
$B = \pmat{e_2, e_1}$.

\item שאר האופציות כבר בצורת ז'ורדן.
\end{enumerate}
\end{enumerate}
\end{solution}

\begin{exercise}
יהי
$V$
מרחב מכפלה פנימית מרוכב ממימד סופי ותהיינה
$S,T \in \endo_{\mbb{C}}\prs{V}$
אוניטריות.
הוכיחו או מצאו דוגמא נגדית עבור כל אחד מהסעיפים הבאים.

\begin{enumerate}
\item $S+T$
נורמלית.
\item $S \circ T$
נורמלית.
\item אם
$S,T$
מתחלפות אז
$S+T$
נורמלית.
\end{enumerate}
\end{exercise}

\begin{solution}
\begin{enumerate}
\item מתקיים
\begin{align*}
\prs{S+T}^* \circ \prs{S+T} &= \prs{S^* + T^*} \circ \prs{S+T}
\\&=
S^* \circ S + S^* \circ T + T^* \circ S + T^* \circ T
\\&=
2 \id_V + S^* \circ T + T^* \circ S
\end{align*}
וגם
\begin{align*}
\prs{S+T} \circ \prs{S+T}^* &= \prs{S + T} \circ \prs{S^* + T^*}
\\&= S \circ S^* + S \circ T^* + T \circ S^* + T \circ T^*
\\ \hphantom{\prs{S+T} \circ \prs{S+T}^*} &= 2\id_V + S \circ T^* + T \circ S^*
\end{align*}
לכן נרצה לדעת האם בהכרח
\[\text{.} S^* \circ T + T^* \circ S = S \circ T^* + T \circ S^*\]

כדי למצוא דוגמא נגדית, נצטרך שלפחות אחת מבין
$S,T$
לא תהיה צמודה לעצמה.
ניקח
\begin{align*}
S \colon \mbb{C}^2 &\to \mbb{C}^2 \\
\pmat{x \\ y} &\mapsto \pmat{y \\ x}
\end{align*}
וכן
\begin{align*}
T \colon \mbb{C}^2 &\to \mbb{C}^2 \\
\text{.} \pmat{x \\ y} &\mapsto \pmat{\frac{x+y}{\sqrt{2}} \\ \frac{x-y}{\sqrt{2}}}
\end{align*}
$E$
בסיס אורתונורמלי ולכן
\begin{align*}
\brs{S^*}_E &= \brs{S}_E^t = \pmat{0 & 1 \\ 1 & 0}^t = \pmat{0 & 1 \\ 1 & 0} = \brs{S}_E
\end{align*}
וגם
\begin{align*}
\text{.} \brs{T^*}_E &= \brs{T}_E^t = \pmat{1/\sqrt{2} & 1/\sqrt{2} \\ }^t = \pmat{ 0 & 1 \\ 0 & 1}
\end{align*}
כעת
\begin{align*}
\brs{S^* \circ T + T^* \circ S}_E &=
\brs{S^*}_E \brs{T}_E + \brs{T^*}_E \brs{S}_E
\\&=
\pmat{0 & 1 \\ 1 & 0} \pmat{0 & 0 \\ 1 & 1} + \pmat{0 & 1 \\ 0 & 1} \pmat{0 & 1 \\ 1 & 0}
\\&=
\pmat{1 & 1 \\ 0 & 0} + \pmat{1 & 0 \\ 1 & 0}
\end{align*}
אבל
\begin{align*}
\pmat{S \circ T^* + T \circ S^*}_E &= \brs{S}_E \brs{T^*}_E + \brs{T}_E \brs{S^*}_E
\\&= \pmat{0 & 1 \\ 1 & 0} \pmat{0 & 1 \\ 0 & 1} + \pmat{0 & 0 \\ 1 & 1} \pmat{0 & 1 \\ 1 & 0}
\\&= \pmat{0 & 0 \\ 1 & 1} + \pmat{1 & 0 \\ 1 & 0}
\end{align*}
ואלו מטריצות שונות.

\item מתקיים
\begin{align*}
\prs{S \circ T}^* \circ \prs{S \circ T} &= T^* \circ S^* \circ S \circ T
\\&=
T^* \circ S \circ S^* \circ T
\\&=
T ^* \circ T
\\&=
\id_V
\end{align*}
לכן
$S \circ T$
אוניטרית ובפרט נורמלית.

\begin{remark}
ראיתם בתרגול שהרכבה של העתקות אוניטריות היא אוניטרית, ולכן אין צורך בפירוט מעבר לכך כפי שהתרגיל מנוסח.
\end{remark}

\item 
כמו מקודם,
מתקיים
\begin{align*}
\prs{S+T}^* \circ \prs{S+T} &= \prs{S^* + T^*} \circ \prs{S+T}
\\&=
S^* \circ S + S^* \circ T + T^* \circ S + T^* \circ T
\\&=
2 \id_V + S^* \circ T + T^* \circ S
\end{align*}
וגם
\begin{align*}
\prs{S+T} \circ \prs{S+T}^* &= \prs{S + T} \circ \prs{S^* + T^*}
\\&= S \circ S^* + S \circ T^* + T \circ S^* + T \circ T^*
\\ \hphantom{\prs{S+T} \circ \prs{S+T}^*} &= 2\id_V + S \circ T^* + T \circ S^*
\end{align*}
כעת,
$S,T$
מתחלפות, לכן גם
$S^*, T$
מתחלפות וגם
$S, T^*$
מתחלפות, ונקבל שוויון.
\end{enumerate}
\end{solution}

\begin{exercise}
יהיו
$v_1, \ldots, v_n \in \mbb{R}^n$.
הראו כי
$B \ceq \prs{v_1, \ldots, v_n}$
בסיס של
$\mbb{R}^n$
אם ורק אם למטריצת
\textenglish{Gram}
של
$B$
\[\mrm{Gr}\prs{B} \ceq \pmat{\trs{v_1, v_1} & \cdots & \trs{v_1, v_n} \\ \vdots & \ddots & \vdots \\ \trs{v_n, v_1} & \cdots & \trs{v_n, v_n}} = \prs{\trs{v_i, v_j}}_{i,j \in \brs{n}}\]
יש דטרמיננטה חיובית.
\end{exercise}

\begin{solution}
נניח כי
$B$
בסיס.
אז
$\mrm{Gr}\prs{B}$
המטריצה המייצגת של המכפלה הפנימית הסטנדרטית לפי
$B$.
לכן קיים
$a \in \mbb{R}$
עבורו
\[\det\prs{\mrm{Gr}\prs{B}} = a^2 \det\prs{I_n} = a^2\]
(כי
$\mrm{Gr}\prs{B} = P^t I_n P$
וכי הדטרמיננטה כפלית).
כיוון ש־%
$\mrm{Gr}\prs{B}$
הפיכה, לא יתכן
$a = 0$
ולכן
$\det\prs{\mrm{Gr}\prs{B}} = a^2 > 0$.

בכיוון השני, נניח ש־%
$B$
אינו בסיס. אז יש סקלרים
$\alpha_1, \ldots, \alpha_{n-1} \in \mbb{R}$
עבורם
\[\text{.} v_n = \sum_{i \in [n-1]} \alpha_i v_i\]
נקבל
\begin{align*}
\text{.} \prs{\mrm{Gr}\prs{B}} &= \pmat{\trs{v_1, v_1} & \cdots & \trs{v_1, \sum_{i \in [n-1]} \alpha_i v_i} \\ \vdots & \ddots & \vdots \\ \trs{v_n, v_1} & \cdots & \trs{v_i, \sum_{i \in [n-1]} \alpha_i v_i}}
=
\pmat{\trs{v_1, v_1} & \cdots & \sum_{i \in [n-1]} \alpha_i \trs{v_1, v_i} \\ \vdots & \ddots & \vdots \\ \trs{v_n, v_1} & \cdots & \sum_{i \in [n-1]} \alpha_i \trs{v_i, v_i}}
\end{align*}
לכן, העמודה הימנית של
$\mrm{Gr}\prs{B}$
היא צירוף לינארי של שאר העמודות, ולכן
$\det\prs{\mrm{Gr}\prs{B}} = 0$.
\end{solution}

\begin{exercise}
יהי
$V$
מרחב וקטורי סוף־מימדי מעל
$\mbb{R}$.
תהי
$g$
מכפלה פנימית על
$V$
ותהי
$h$
תבנית בילינארית סימטרית על
$V$.
הוכיחו שקיים בסיס
$B$
של
$V$
עבורו
$\brs{h}_B, \brs{g}_B$
שתיהן אלכסוניות.
\end{exercise}

\begin{solution}
נסמן
$n \ceq \dim_{\mbb{R}}\prs{V}$.
יהי
$E$
בסיס אורתונורמלי של
$V$
ביחס ל־%
$g$.
אז
$\brs{g}_E = I_n$.
כעת,
$\brs{h}_E$
סימטרית כי
$h$
סימטרית, ולכן יש מטריצה
$Q \in M_n\prs{\mbb{R}}$
אורתוגונלית עבורה
$Q^t \brs{h}_E Q$
אלכסונית.
נרצה
$Q = P^B_E$
ולכן נגדיר
$B \ceq \prs{Q^{-1} v_1, \ldots, Q^{-1} v_n}$
עבור
$E \ceq \prs{v_1, \ldots, v_n}$.
אז
\begin{align*}
\brs{g}_B &= \prs{P^B_E}^t \brs{g}_E P^B_E = Q^t \brs{g}_E Q = Q^t I_n Q = Q^t Q = I_n \\
\brs{h}_B &= \prs{P^B_E}^t \brs{h}_E P^B_E = Q^t \brs{h}_E Q
\end{align*}
שתיהן אלכסוניות, כנדרש.
\end{solution}

\begin{exercise}
יהי
$V$
מרחב מכפלה פנימית ממימד
$n \in \mbb{N}$
מעל
$\mbb{R}$.
יהיו
$U,W \leq V$
עבורם
$V = U \oplus W$.
עבור
$v = u + w$
כאשר
$u \in U, w \in W$
נגדיר
$T\prs{v} = u - w$.
הניחו כי
$T$
לינארית והראו כי
$T$
צמודה לעצמה אם ורק אם
$U \perp W$.
\end{exercise}

\begin{solution}
נניח כי
$U \perp W$.
יהי
$B$
בסיס אורתונורמלי ל־%
$U$
ויהי
$C$
בסיס אורתונורמלי ל־%
$W$.
אז
$D \ceq B * C$
בסיס אורתונורמלי ל־%
$V$.
נקבל כי בבסיס זה
\[\brs{T}_D = \pmat{\brs{\rest{T}{U}}_B & 0 \\ 0 & \brs{\rest{T}{W}}_C} = \pmat{I_k & 0 \\ 0 & - I_\ell}\]
עבור
$k \ceq \dim U, \ell \ceq \dim W$.
כיוון ש־%
$D$
אורתונורמלי, מתקיים
\begin{align*}
\text{.} \brs{T^*}_D &= \brs{T}_D^t = \pmat{I_k & 0 \\ 0 & - I_{\ell}}^t = \pmat{I_k & 0 \\ 0 & - I_{\ell}} = \brs{T}_D
\end{align*}
לכן,
$T^* = T$,
כנדרש.

נניח כעת כי
$T^* = T$
ויהיו
$u \in U$
ו־%
$w \in W$.
מתקיים
\begin{align*}
\trs{u,w} &= \trs{Tu, w}
\\&= \trs{u, T^*  w}
\\&= \trs{u, T w}
\\&= \trs{u, -w}
\\&= -\trs{u,w}
\end{align*}
לכן
$\trs{u,w} = 0$
ולכן
$u \perp w$.
נקבל כי
$U \perp W$.
\end{solution}

\begin{exercise}
תהי
$A \in \Mat_n\prs{\CC}$
בעלת פולינום מינימלי
\begin{align*}
\text{.} m_A\prs{x} = x^8 \prs{x^2 + 1}
\end{align*}

\begin{enumerate}
\item מיצאו את הפולינום המינימלי של
$A^3$.

\item הוכיחו או הפריכו את כל אחת מהטענות הבאות:

\begin{enumerate}
\item קיימת חזקה של
$A$
שלכסינה מעל
$\RR$.

\item קיימת חזקה של
$A$
שלכסינה מעל
$\CC$.
\end{enumerate}
\end{enumerate}
\end{exercise}

\begin{solution}
\begin{enumerate}
\item הערכים העצמיים של מטריצה הם שורשי הפולינום המינימלי, לכן ל־%
$A$
יש ערכים עצמיים
$0, \pm i$.
החזקה של
$x - \lambda$
בפולינום המינימלי היא הגודל המקסימלי של בלוק ז'ורדן עם ערך עצמי
$\lambda$
בצורת ז'ורדן של
$A$.
לכן, צורת ז'ורדן של
$A$
היא
\begin{align*}
J \coloneqq \diag\prs{J_8\prs{0}, N, i I_{r_{a,A}\prs{i}}, -i I_{r_{a,A}\prs{-i}}}
\end{align*}
עבור מטריצת ז'ורדן
$N$
עם ערך עצמי יחיד
$0$
ובלוקים מגודל לכל היותר
$0$.
נקבל כי
\begin{align*}
\text{.} J^3 = \diag\prs{J_8\prs{0}, N^3, -i I_{r_{a,A}\prs{i}}, i I_{r_{a,A}\prs{-i}}}
\end{align*}
אז
$\pm i$
ערך עצמי מריבוי אלגברי וגיאומטרי
$r_{a,A}\prs{\mp i}$.
כיוון שהריבוי הגיאומטרי שווה למספר בלוקי ז'ורדן בצורת ז'ורדן, והריבוי האלגברי שווה לסכום הגדלים שלהם, נקבל כי יש
$r_{a,A}\prs{\pm}$
בלוקים מגודל
$1$
עם ערך עצמי
$\mp i$.

עבור הערך העצמי
$0$,
ניזכר כי צורת ז'ורדן של
$J_8\prs{0}^3$
היא
\begin{align*}
\diag\prs{J_3\prs{0}, J_3\prs{0}, J_2\prs{0}}
\end{align*}
ושבאותו אופן צורת ז'ורדן של
$J_m\prs{0}^3$
מורכבת מבלוקים בגדלים בקבוצה
\[\text{.} \set{\ceil{\frac{m}{3}}, \floor{\frac{m}{3}}}\]
לכן, גודל הבלוק המקסימלי עם ערך עצמי
$0$
בצורת ז'ורדן של
$J^3$
הוא
$3$.
אך, אם
$P^{-1} A P = J$
נקבל כי
\[P^{-1} A^3 P = \prs{P^{-1} A P}^3 = J^3\]
ולכן זה גם הגודל של הבלוק המקסימלי עם ערך עצמי
$0$
בצורת ז'ורדן של
$A^3$.
נקבל כי
\[\text{.} m_{A^3}\prs{x} = x^3 \prs{x^2 + 1}\]

\item
ניזכר כי הערכים העצמיים של
$A^8$
הם
$\lambda^8$
עבור
$\lambda$
ערך עצמי של
$A$.
לכן הערכים העצמיים של
$A^8$
הם
$\set{0, \prs{\pm i}^8} = \set{0, 1}$.

אם נתייחס ל־%
$A$
כמטריצה ממשית, נקבל כי סכום הריבויים האלגבריים של הערכים העצמיים שלה הוא
$n$,
ולכן יש לה צורת ז'ורדן.

צורת ז'ורדן של
$A^8$
היא צורת ז'ורדן של
\begin{align*}
\text{,} J^8 &= \diag\prs{J_8\prs{0}^8, N^8, i^8 I_{r_{a,A}\prs{i}}, \prs{-i}^8 I_{r_{a,A}\prs{-i}}}
\\&=
\diag\prs{0_{r_{a,A}\prs{0}}, I_{r_{a,A}\prs{i} + r_{a,A}\prs{-i}}}
\end{align*}
שהיא כבר מטריצה אלכסונית, לכן
$A^8$
לכסינה גם מעל
$\RR$
וגם מעל
$\CC$.
\end{enumerate}
\end{solution}

\begin{exercise}
נתונה
$T \in \End_\RR\prs{\RR^n}$
המוגדרת על ידי
\begin{align*}
\text{.} T \pmat{a_1 \\ a_2 \\ \vdots \\ a
_{n-1} \\ a_n} = \pmat{a_2 \\ a_3 \\ \vdots \\ a_n \\ a_1}
\end{align*}

\begin{enumerate}
\item האם
$T$
איזומטריה ביחס למכפלה הפנימית הסטנדרטית על
$\RR^n$?

\item האם
$T$
צמודה לעצמה? אורתוגונלית?

\item מיצאו את צורת ז'ורדן של
$T$
מעל
$\RR$
או הוכיחו שאינה קיימת.
\end{enumerate}
\end{exercise}

\begin{solution}
\begin{enumerate}
\item יהי
$\mrm{St} = \prs{e_1, \ldots, e_n}$
הבסיס הסטנדרטי של
$\RR^n$.
אז
\begin{align*}
\text{,} T\prs{e_i} &= \fcases{e_{i-1} & i > 1 \\ e_n & i = 1}
\end{align*}
ולכן
$T$
שולחת בסיס אורתונורמלי
$\mrm{St}$
לבסיס אורתונורמלי
$\prs{e_n, e_1, \ldots, e_{n-1}}$.
ראינו בכיתה שההעתקה ששולחת בסיס אורתונורמלי לבסיס אורתונורמלי היא אורתוגונלית, ושההעתקה אורתוגונלית היא איזומטריה, ולכן
$T$
איזומטריצה.

\item הראינו בסעיף הקודם ש־%
$T$
אורתוגונלית. לכן
$T^* = T^{-1}$.
נשים לב כי
$T \neq T^{-1}$
כי
$T^2 \neq \id_{\RR^2}$
(למשל, כי
$T^2\prs{e_3} = T\prs{e_2} = e1 \neq e_3$)
ולכן
$T^* = T^{-1} \neq T$
ונקבל כי
$T$
אינה צמודה לעצמה.

\item

יהי
$\mrm{St}$
הבסיס הסטנדרטי של
$\RR^n$.
אז
$T = L_A$
עבור
\begin{align*}
\text{.} A \coloneqq \brs{T}_{\mrm{St}} =
\pmat{
0 & 1 & \cdots & 0 & 0 \\
0 & \ddots & \ddots & \ddots & 0 \\
0 & \ddots & \ddots & \ddots & \vdots \\
\vdots & \ddots & \ddots & \ddots & 1 \\
1 & \cdots & 0 & 0 & 0 \\
}
\end{align*}
יהי
$\xi = \mrm{cis}\prs{\frac{2 \pi}{n}}$,
ואז
\begin{align*}
A \pmat{\xi^n \\ \xi^{n-1} \\ \vdots \\ \xi^2 \\ \xi} = \pmat{\xi^{n-1} \\ \xi^{n-2} \\ \vdots \\ \xi \\ \xi^n}
\end{align*}
כאשר
$\xi^n = \mrm{cis}\prs{2 \pi} = 1$.
נקבל שהוקטור
$\pmat{\xi^n \\ \xi^{n-1} \\ \vdots \\ \xi^2 \\ \xi}$
הוא וקטור עצמי של
$A$
עם ערך עצמי
$\xi^{-1}$,
שאינו ממשי, ולכן אין ל־%
$T$
צורת ז'ורדן מעל
$\RR$.
\end{enumerate}
\end{solution}

\printbibliography
\end{document}
