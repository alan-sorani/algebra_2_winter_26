\documentclass[a4paper,10pt,twoside,openany]{article}

\usepackage[lang=hebrew]{maths}
\usepackage{polynom}
\usepackage{hebrewdoc}
\usepackage{stylish}
\usepackage{lipsum}
\let\bs\blacksquare

\setlength{\parindent}{0pt}

%%%%%%%%%%%%
% Styling %
%%%%%%%%%%%%

\usepackage{enumitem}

%%%%%%%%%%%%%
% Counters  %
%%%%%%%%%%%%%

\setcounter{section}{0}     
            
%BIBLIOGRAPHY
\usepackage[
backend=biber,
style=alphabetic,
]{biblatex}
\addbibresource{bibliography.bib} %Imports bibliography file

\title{
אלגברה ב' (01040168) - חורף 2026
\\
תרגול 2 - סכומים ישרים והטלות
\\
אלן סורני
\\
הרשימות עודכנו לאחרונה בתאריך ה־%
\today
}
\date{}

\begin{document}
\maketitle

\section{סכומים ישרים פנימיים}

\begin{definition}[סכום ישר פנימי]
יהי
$V$
מרחב וקטורי סוף־מימדי מעל
$\mbb{F}$
ויהיו
$V_1, \ldots, V_k \leq V$
תת־מרחבים.
נזכיר כי
\begin{align*}
\text{.} V_1 + \ldots + V_k \coloneqq \set{v_1 + \ldots + v_k}{\forall i \in [k] \colon v_i \in V_i}
\end{align*}
נגיד שהסכום הזה הוא
\emph{סכום ישר}
אם כל
$v \in V_1 + \ldots + V_k$
ניתן לכתיבה
$v = v_1 + \ldots + v_k$
\emph{בצורה יחידה}
עבור
$v_i \in V_i$.
במקרה זה נסמן את הסכום
$\bigoplus_{i \in [k]} V_i = V_1 \oplus \ldots \oplus V_k$.
\end{definition}

\begin{remark}
באופן שקול, הסכום
$V_1 + \ldots + V_k$
ישר אם ורק אם
$v_1 + \ldots + v_k = 0$
עבור
$v_i \in V_i$
גורר
$v_i = 0$
לכל
$i \in [k]$.
\end{remark}

\begin{proposition} \label{proposition:direct-sum}
הסכום
$\sum_{i \in [k]} V_i \coloneqq V_1 + \ldots + V_k$
ישר אם ורק אם
\[V_i \cap \prs{\sum_{j \in \brs{k} \setminus i} V_j} = \set{0}\]
לכל
$i \in [k]$.
\end{proposition}

\begin{definition}[שרשור קבוצות סדורות]
תהיינה
\begin{align*}
A_1 &= \prs{v_{1,1}, \ldots, v_{1, \ell_1}} \\
A_2 &= \prs{v_{2,1}, \ldots, v_{2, \ell_2}} \\
&\vdots \\
A_k &= \prs{v_{k,1}, \ldots, v_{k, \ell_k}}
\end{align*}
קבוצות סדורות.
נגדיר את
\emph{השרשור שלהן}
\begin{align*}
\text{.} A_1 * \ldots * A_k \coloneqq \prs{v_{1,1}, \ldots, v_{1,\ell_1}, v_{2,1}, \ldots, v_{2,\ell_2}, \ldots, v_{k,1}, \ldots, v_{k,\ell_k}}
\end{align*}
זאת הקבוצה הסדורה שהיא שרשור איברי הקבוצות הסדורות
$A_1, \ldots, A_k$
לפי הסדר.
\end{definition}

\newpage

\begin{exercise}
יהי
$V$
מרחב וקטורי סוף־מימדי מעל שדה
$\FF$,
ויהיו
$V_1, V_2 \leq V$
עבורם
$V = V_1 \oplus V_2$.
הראו כי
$\dim_\FF \prs{V} = \dim_\FF\prs{V_1} + \dim_\FF\prs{V_2}$.

הסיקו כי אם
$V_1, \ldots, V_k \leq V$
מקיימים כי
$V = V_1 \oplus \ldots \oplus V_k$
אז
$\dim_\FF\prs{V} = \sum_{i \in \brs{k}} \dim_\FF\prs{V_i}$.
\end{exercise}

\begin{solution}
לפי ההגדרה, אם
$V = V_1 \oplus V_2$
אז
$V_1 \cap V_2 = \set{0}$.

לכן, לפי משפט המימדים,
\begin{align*}
\text{.} \dim_\FF \prs{V} = \dim_\FF\prs{V_1} + \dim_\FF\prs{V_2} - \dim_\FF\prs{V_1 \cap V_2} = \dim_\FF\prs{V_1} + \dim_\FF\prs{V_2}
\end{align*}

נוכיח את המקרה בו
$V = V_1 \oplus \ldots \oplus V_k$
באינדוקציה.
הראנו את מקרה הבסיס
$k=2$.
נניח כי הטענה מתקיימת עבור
$k = m$
ונראה כי היא מתקיימת עבור
$k = m+1$.
יהיו
$V_1, \ldots, V_{m+1} \leq V$
עבורם
$V = V_1 \oplus \ldots \oplus V_{m+1}$.
אז
\begin{align*}
\text{.} V = \prs{V_1 \oplus \ldots \oplus V_m} \oplus V_{m+1}
\end{align*}
מהנחת האינדוקציה נקבל כי
\begin{align*}
\text{,} \dim_\FF\prs{V_1 \oplus \ldots \oplus V_m} = \sum_{i \in \brs{m}} \dim_\FF \prs{V_i}
\end{align*}
וממקרה הבסיס נקבל כי
\begin{align*}
\text{.} \dim_\FF \prs{V} = \dim_\FF\prs{V_1 \oplus \ldots \oplus V_m} + \dim_\FF \prs{V_{m+1}}
\end{align*}
נציב את השוויון הנ"ל עבור הגורם הראשון, ונקבל את הנדרש.
\end{solution}

\begin{exercise}
יהי
$V$
מרחב וקטורי סוף־מימדי ויהיו
$V_1, \ldots, V_k$
תת־מרחבים של
$V$.
התנאים הבאים שקולים.
\begin{enumerate}
\item $V = V_1 \oplus \ldots \oplus V_k$.
\item לכל בחירת בסיסים
$B_i$
של
$V_i$
הקבוצה הסדורה
$B_1 * \ldots * B_k$
היא בסיס של
$V$.
\item קיימים בסיסים
$B_i$
של
$V_i$
כך שהקבוצה הסדורה
$B_1 * \ldots * B_k$
היא בסיס של
$V$.
\item $V = \sum_{i \in [k]} V_i$
וגם
\[\text{.} \dim V = \sum_{i \in [k]} \dim\prs{V_i}\]
\end{enumerate}
\end{exercise}

\begin{solution}
נראה גרירות בין התנאים השונים כדי להראות שהם שקולים זה לזה.

\begin{description}
\item[$1 \implies 2$:]

נניח כי
$V = V_1 \oplus \ldots \oplus V_k$,
ויהיו
$B_1, \ldots, B_k$
בסיסים של
$V_1, \ldots, V_k$
בהתאמה. נרצה להראות כי
$B \coloneqq B_1 * \ldots * B_k$
הינו בסיס של
$V$.

יהי
$v \in V$.
מההנחה, ניתן לכתוב
\begin{align*}
v = v_1 + \ldots + v_k
\end{align*}
עבור
$v_i \in V_i$
לכל
$i \in \brs{k}$.
כיוון ש־%
$B_i$
בסיס של
$V_i$
לכל
$i \in \brs{k}$,
מתקיים
$v_i \in \Span_{\FF}\prs{B_i}$.
לכן
\begin{align*}
v &= v_1 + \ldots + v_k
\\&\in
\Span\prs{B_1} + \ldots + \Span\prs{B_k}
\\&\subseteq
\Span\prs{B}
\end{align*}
כלומר
$B$
פורש את
$V$.

כעת,
\begin{align*}
\text{,} \dim_\FF \prs{V} = \sum_{i \in \brs{k}} \dim_\FF\prs{V_i}
\end{align*}
וגם
\begin{align*}
\text{,} |B| = \sum_{i \in \brs{k}} |B_i| = \sum_{i \in \brs{k}} \dim_\FF\prs{V_i}
\end{align*}
לכן
$B$
קבוצה פורשת של
$V$
מגודל
$\dim_\FF V$,
ולכן בסיס של
$V$.

\item[$2 \implies 3$:]
מיידי.

\item[$3 \implies 4$:]
המימד של
$V$
שווה לגודל של בסיס של
$V$,
והגודל של
$B$
הוא בדיוק הסכום באגף ימין של 4.

נסמן
\begin{align*}
B_i \coloneqq \prs{v_1^{\prs{i}}, \ldots, v_{m_i}^{\prs{i}}}
\end{align*}
לכל
$i \in \brs{k}$,
ויהי
$v \in V$.
אז קיימים
$\alpha^{\prs{i}}_j \in \FF$
עבורם ניתן לכתוב
\begin{align*}
\text{.} v = \sum_{i \in \brs{k}} \sum_{j \in \brs{m_i}} \alpha^{\prs{i}}_j v^{\prs{i}}_j
\end{align*}
אבל,
$B_i$
בסיס של
$V_i$
ולכן
\begin{align*}
v_i \coloneqq \sum_{j \in \brs{m_i}} \alpha^{\prs{i}}_j v^{\prs{i}}_j \in V_i
\end{align*}
וקטור ב־%
$V_i$.
קיבלנו כי
$v$
סכום של וקטורים
$v_i \in V_i$,
לכל
$v \in V$,
ולכן
$V = \sum_{i \in \brs{k}} V_i$.

\item[$4 \implies 1$:]
נראה כי
\[V_i \cap \prs{\sum_{j \in \brs{k} \setminus i} V_j} = \set{0}\]
לכל
$i \in [k]$,
ונקבל את הנדרש לפי טענה
\ref{proposition:direct-sum}.

יהי
$i \in \brs{k}$.
לפי משפט המימדים, מתקיים
\begin{align*}
\dim_\FF \prs{V} &=
\dim_\FF \prs{\sum_{j \in \brs{k}} V_j} \\&=
\dim_\FF \prs{V_i + \sum_{j \in \brs{k} \setminus \set{i}} V_j}
\\ \text{.} \hphantom{\dim_\FF \prs{V}} &=
\dim_\FF \prs{V_i} + \dim_\FF \prs{\sum_{j \in \brs{k} \setminus \set{i}} V_j} - \dim_\FF \prs{V_i \cap \prs{\sum_{j \in \brs{k} \setminus i} V_j}}
\end{align*}
אבל, לפי ההנחה
\begin{align*}
\text{.} \dim_\FF\prs{V} = \sum_{j \in \brs{k}} \dim_\FF \prs{V_j}
\end{align*}
לכן, אם נראה שמתקיים
\[\dim_\FF \prs{\sum_{j \in \brs{k} \setminus i} V_j} = \sum_{j \in \brs{k} \setminus \set{i}} \dim_\FF \prs{V_j}\]
נקבל את הנדרש.

ממשפט המימדים נקבל כי
\[\text{,} \dim_\FF \prs{\sum_{j \in \brs{k} \setminus i} V_j} \leq \sum_{j \in \brs{k} \setminus \set{i}} \dim_\FF \prs{V_j}\]
לכן נותר להראות שאגף שמאל אינו קטן ממש מאגף ימין.
נניח בדרך השלילה שהוא כן. אז
\begin{align*}
\dim_\FF \prs{V} &< \dim_\FF \prs{V_i} + \sum_{j \in \brs{k} \setminus \set{i}} \dim_\FF \prs{V_j} - \dim_\FF \prs{V_i \cap \prs{\sum_{j \in \brs{k} \setminus i} V_j}}
\\&\leq
\sum_{j \in \brs{k}} \dim_\FF V_j
\\\text{,} \hphantom{\dim_\FF \prs{V}} &= \dim_\FF \prs{V}
\end{align*}
בסתירה.
\end{description}
\end{solution}

\begin{definition}[משלים ישר]
יהי
$V$
מרחב וקטורי ויהי
$U \leq V$
תת־מרחב.
\emph{משלים ישר}
$W$
של
$U$
הוא תת־מרחב של
$V$
עבורו
$V = U \oplus W$.
\end{definition}

\begin{exercise}
יהי
$V$
מרחב וקטורי סוף־מימדי מעל שדה
$\mbb{F}$
ויהי
$U \leq V$
תת־מרחב עם בסיס
$B$.
יהי
$C$
בסיס של
$V$.
\begin{enumerate}
\item הראו שניתן להשלים את
$B$
לבסיס של
$V$
על ידי הוספת וקטורים מ־%
$C$.

\item הסיקו שקיים משלים ישר
$W$
של
$U$
עם בסיס של וקטורים מ־%
$C$.
\end{enumerate}
\end{exercise}

\begin{solution}
\begin{enumerate}
\item נסמן
$n \coloneqq \dim_{\mbb{F}}\prs{V}$
ונוכיח את הטענה באינדוקציה על
$m = n - \abs{B}$.

עבור
$m = 0$
מתקיים
$\abs{B} = n$
ולכן
$U = V$.
נניח שהטענה נכונה לכל
$k < m$
ונוכיח אותה עבור
$m$.

אם
$C \subseteq U$,
מתקיים
\[V = \Span_{\mbb{F}}\prs{C} \subseteq \Span_{\mbb{F}}\prs{U} = U\]
ולכן
$V = U$
בסתירה לכך שהמימדים שונים.
לכן, קיים
$c \in C \setminus U$.
אז
$B * \prs{c}$
קבוצה בלתי־תלויה לינארית, כי
$c$
אינו צירוף לינארי של הוקטורים הקודמים. נגדיר
$U' = \Span_{\mbb{F}}\prs{B * \prs{c}}$.
אז
\[n - \dim\prs{U'} = n - \abs{B} - 1 = m-1 < m\]
ולכן ניתן להשתמש בהנחת האינדוקציה ולקבל שניתן להשלים את
$B * \prs{c}$
לבסיס
$\prs{B * \prs{c}} * \prs{c_2, \ldots, c_m}$
של
$V$,
כאשר
$c_i \in C$.
אז
$c, c_2, \ldots, c_m \in C$
משלימים את
$B$
לבסיס של
$V$.

\item בסימונים של הסעיף הקודם,
$B * \prs{c, \ldots, c_m}$
בסיס של
$V$.
נסמן
$D = \prs{c, c_2, \ldots, c_m}$
וגם
$W = \Span_{\mbb{F}}\prs{D}$.
אז
$B * D$
בסיס של
$V$
ולכן
\[\text{,} V = \Span_{\mbb{F}}\prs{B} \oplus \Span_{\mbb{F}}\prs{D} = U \oplus W\]
כנדרש.
\end{enumerate}
\end{solution}

\begin{exercise}
יהי
$V = \mbb{R}_3\brs{x}$
ותהיינה
\begin{align*}
B &= \prs{1+x, x+x^2} \\
C &= \prs{1, x, x^2, x^3}
\end{align*}
קבוצות סדורות של וקטורים מ־%
$V$.
יהי
$U = \Span\prs{B}$.

\begin{enumerate}
\item מיצאו משלים ישר
$W$
של
$U$
ובסיס עבור
$W$
שמורכב מוקטורים ב־%
$C$.
\item
האם
$W$
שמצאתם יחיד? הוכיחו או הפריכו.
\end{enumerate}
\end{exercise}

\begin{solution}
\begin{enumerate}
\item נשלים את
$B$
לבסיס של
$V$
על ידי הוספת וקטורים מ־%
$C$.
נוסיף את
$1 \notin U$
כדי לקבל
$B' = \pmat{1 + x, x + x^2, 1}$
ואז את
$x^3 \notin \Span\prs{B'}$
כדי לקבל בסיס
$B'' = \pmat{1 + x, x + x^2, 1, x^3}$
של
$V$.

נסמן
$D = \prs{1, x^3}$
וניקח
$W = \Span\prs{D}$.
אז
$B'' = B * D$
בסיס,
ולכן
$V = U \oplus W$,
כנדרש.

\item לא.
למשל, יכולנו לקחת
$B' = \pmat{1 + x, x + x^2, x^2}$
ואז
$B'' = \pmat{1 + x, x + x^2, x^2, x^3}$.
במקרה זה היינו מקבלות משלים ישר
$\Span\prs{x^2, x^3}$,
ששונה מ־%
$W$.
\end{enumerate}
\end{solution}

\section{סכומים ישרים חיצוניים}

לפעמים נרצה לדבר על סכומים ישרים של מרחבים וקטורים שאינם תת־מרחבים של אותו מרחב וקטורי. לכן נגדיר
\emph{סכום ישר חיצוני}.
נסמן אותו באותו אופן, אך הסכום הישר החיצוני של תת־מרחבים
$V_1, \ldots, V_k \leq V$
איזומורפי לסכום הישר הפנימי שלהם.

\begin{definition}[סכום ישר חיצוני]
יהיו
$V_1, \ldots, V_k$
מרחבים וקטוריים מעל שדה
$\FF$.
נגדיר את
\emph{הסכום הישר החיצוני של
$V_1, \ldots, V_k$}
בתור המרחב הוקטורי
\begin{align*}
V_1 \oplus \ldots \oplus V_k \coloneqq \set{\prs{v_1, \ldots, v_k}}{v_i \in V_i}
\end{align*}
עם חיבור וכפל בסקלר איבר-איבר.
\end{definition}

\begin{remark}
ניתן לחשוב על סכום ישר חיצוני בתור סכום ישר פנימי.
אם
$V = V_1 \oplus \ldots \oplus V_k$
סכום ישר חיצוני,
נחשוב על התת־מרחב
\begin{align*}
\tilde{V}_i \coloneqq \set{\prs{v_1, \ldots, v_k}}{\substack{v_j \in V_j \\ \forall j \neq i: v_j = 0}}
\end{align*}
בתור עותק של
$V_i$
בתוך
$V$,
וניתן לכתוב את
$V$
בתור הסכום הישר הפנימי
\begin{align*}
\text{.} V = \tilde{V}_1 \oplus \ldots \oplus \tilde{V}_k
\end{align*}

ההעתקה
$i_j \colon V_j \to \tilde{V}_j$
ששולחת וקטור
$v$
לוקטור
$\prs{0, \ldots, 0, v, 0, \ldots, 0}$
נקראת
\emph{השיכון של
$V_j$
בסכום הישר
$V$}
והיא איזומורפיזם לינארי.
\end{remark}

\begin{definition}[שרשור בסיסים עבור סכום ישר חיצוני]
יהיו
$V_1, \ldots, V_k$
מרחבים וקטוריים סוף־מימדיים מעל שדה
$\FF$,
עם בסיסים
$B_1, \ldots, B_k$
בהתאמה.

כאשר נתייחס לשרשור הבסיסים
$B_1 * \ldots * B_k$
כבסיס של הסכום החיצוני
$V_1 \oplus \ldots \oplus V_k$,
נתכוון לביטוי
\[B_1 * \ldots * B_k \coloneqq i_1\prs{B_1} * \ldots * i_k\prs{B_k}\]
כאשר הביטוי בצד ימין הוא שרשור הבסיסים שהגדרנו בעבר.
\end{definition}

\begin{definition}[סכום ישר של העתקות לינאריות]
יהיו
$V_1, \ldots, V_k, W_1, \ldots, W_k$
מרחבים וקטוריים מעל שדה
$\FF$,
ויהיו
$T_i \in \hom\FF\prs{V_i, W_i}$
לכל
$i \in \brs{k}$.

\emph{הסכום הישר של ההעתקות
$T_i$}
הוא
\begin{align*}
T_1 \oplus \ldots \oplus T_k \colon V_1 \oplus \ldots \oplus V_k &\to W_1 \oplus \ldots \oplus W_k \\
\text{.} \hphantom{lalalalalalalalala} \prs{v_1, \ldots, v_k} &\mapsto \prs{T_1\prs{v_1}, \ldots, T_k\prs{v_k}}
\end{align*}
\end{definition}

\printbibliography
\end{document}
