\documentclass[a4paper,10pt,twoside,openany]{article}

\usepackage[lang=hebrew]{maths}
\usepackage{polynom}
\usepackage{hebrewdoc}
\usepackage{stylish}
\usepackage{lipsum}
\let\bs\blacksquare

\setlength{\parindent}{0pt}
\newcommand{\Std}{\mrm{Std}}
\DeclareMathOperator{\Char}{char}

%%%%%%%%%%%%
% Styling %
%%%%%%%%%%%%

\usepackage{enumitem}

%%%%%%%%%%%%%
% Counters  %
%%%%%%%%%%%%%

\setcounter{section}{0}     
            
%BIBLIOGRAPHY
\usepackage[
backend=biber,
style=alphabetic,
]{biblatex}
\addbibresource{bibliography.bib} %Imports bibliography file

\title{
אלגברה ב' (01040168) - חורף 2026
\\
תרגול 11 - תבניות בילינאריות וחוק האינרציה של סילבסטר
\\
אלן סורני
\\
הרשימות עודכנו לאחרונה בתאריך ה־%
\today
}
\date{}

\begin{document}
\maketitle

\section{תבניות בילינאריות}

\begin{definition}[תבנית בילינארית]
יהי
$V$
מרחב וקטורי. תבנית בילינארית
$f$
על
$V$
היא העתקה
$f \colon V \to V$
שהינה לינארית בשני הרכיבים.
\end{definition}

\begin{remark}
מעל
$\mbb{C}$
עוסקות לעתים בתבניות ססקווילינאריות ("אחת־וחצי לינאריות") במקום בתבניות בילינאריות. נדבר על אלו בהמשך.
\end{remark}

\begin{definition}[מטריצה מייצגת של תבנית בילינארית]
יהי
$V$
מרחב וקטורי סוף־מימדי עם בסיס
$B = \prs{v_1, \ldots, v_n}$,
ותהי
$f$
תבנית בילינארית על
$V$.
\emph{המטריצה המייצגת}
של
$f$
לפי
$B$
היא
\begin{align*}
\text{.} \brs{f}_B &= \pmat{f\prs{v_1, v_1} & \cdots & f\prs{v_1, v_n} \\ \vdots & \ddots & \vdots \\ f\prs{v_n,v_1} & \cdots & f\prs{v_n,v_n}}
\end{align*}
\end{definition}

\begin{proposition}
לכל
$u,v \in V$
מתקיים
\[\text{.} \brs{u}_B^t \brs{f}_B \brs{v}_B = f\prs{u,v}\]

בנוסף, אם
$\brs{u}_B^t A \brs{v}_B = f\prs{u,v}$
לכל
$u,v \in V$,
אז
$A = \brs{f}_B$.
\end{proposition}

\begin{comment}
\begin{proof}
נכתוב
\begin{align*}
u &= \sum_{i \in \brs{n}} \alpha_i v_i \\
v &= \sum_{j \in \brs{n}} \beta v_j
\end{align*}
ואז
\begin{align*}
\brs{u}_B^t \brs{f}_B \brs{v}_B &= \sum_{i,j \in \brs{n}} \alpha_i \beta_j \brs{v_i}_B^t \brs{f}_B \brs{v_j}_B
\\&= \sum_{i,j \in \brs{n}} \alpha_i \beta_j e_i^t \brs{f}_B e_j
\\&= \sum_{i,j \in \brs{n}} \alpha_i \beta_j f\prs{v_i, v_j}
\\&= f\prs{\sum_{i \in \brs{n}} \alpha_i v_i, \sum_{j \in \brs{n}} \beta_j v_j}
\\ \text{.} \hphantom{\brs{u}_B^t \brs{f}_B \brs{v}_B} &= f\prs{u,v}
\end{align*}

לבסוף, ניתן לקחת
$u = v_i, v = v_j$
ואז
$\brs{u}_B^t A \brs{v}_B = a_{i,j}$.
לכן אם
$\brs{u}_B^t A \brs{v}_B = f\prs{u,v}$
לכל
$u,v \in V$
נקבל כי
$A = \brs{f}_B$.
\end{proof}
\end{comment}

\begin{corollary}\label{corollary:change-of-bases}
יהיו
$B,C$
שני בסיסים של אותו מרחב וקטורי סוף־מימדי
$V$,
ותהי
$f$
תבנית בילינארית על
$V$.
אז
\begin{align*}
\text{.} \brs{f}_B = \prs{M^B_C}^t \brs{f}_C M^B_C
\end{align*}
\end{corollary}

\begin{proof}
לכל
$u,v \in V$
מתקיים
\begin{align*}
\brs{u}_B^t \prs{M^B_C}^t \brs{f}_C M^B_C \brs{v}_B &=
\prs{M^B_C \brs{u}_B}^t \brs{f}_C M^B_C \brs{v}_B
\\&= \brs{u}_C^t \brs{f}_C \brs{v}_C
\\&= f\prs{u,v}
\end{align*}
לכן מהטענה נקבל שוויון.
\end{proof}

במקרה של אופרטור
$T$
על
$V$
הקשר בין
$\brs{T}_B, \brs{T}_C$
הוא דימיון. המסקנה מראה קשר אחר במקרה של תבניות בילינאריות, שמוביל להגדרה הבאה.

\begin{definition}[מטריצות חופפות]
מטריצות
$A,B \in \Mat_n\prs{\mbb{F}}$
נקראות
\emph{חופפות}
אם קיימת
$P \in \Mat_n\prs{\mbb{F}}$
\textbf{הפיכה}
עבורה
$B = P^t A P$.
\end{definition}

\begin{corollary}
המטריצות
$\brs{f}_B, \brs{f}_C$
המייצגות תבנית בילינארית לפי בסיסים שונים, הינן חופפות.
\end{corollary}

\begin{remark}
ניתן לשחזר את תבנית הבילינארית ממטריצה מייצגת שלה, ולמעשה ההעתקה
$f \mapsto \brs{f}_B$
הינה איזומורפיזם בין מרחב התבניות הבילינאריות והמרחב
$\Mat_n\prs{\mbb{F}}$.
\end{remark}

\begin{definition}[תבנית בילינארית סימטרית]
יהי
$V$
מרחב מכפלה פנימית ותהי
$f$
תבנית בילינארית על
$V$.
נגיד כי
$f$
\emph{סימטרית}
אם
$f\prs{u,v} = f\prs{v,u}$
לכל
$u,v \in V$.
\end{definition}

\begin{definition}
יהי
$V$
מרחב מכפלה פנימית ותהי
$f$
תבנית בילינארית על
$V$.
נגיד כי
$f$
\emph{מוגדרת חיובית לחלוטין}
אם
$f\prs{u,u} > 0$
לכל
$u \in V$.

נגדיר באופן דומה תבנית מוגדרת שלילית לחלוטין%
\slash%
אי־שלילית%
\slash%
אי־חיובית.
\end{definition}

\begin{exercise}
תהיינה
$A,B,P \in \Mat_n\prs{\mbb{R}}$
כאשר
$P$
הפיכה ומתקיים
$B = P^t A P$.

הוכיחו או הפריכו את הטענות הבאות.
\begin{enumerate}
\item $\det\prs{A} = \det\prs{B}$.
\item ל־%
$A,B$
יש אותם ערכים עצמיים.
\item $\rank\prs{A} = \rank\prs{B}$.
\item $A$
הפיכה אם ורק אם
$B$
הפיכה, וכאשר זה מתקיים
$A^{-1}, B^{-1}$
חופפות.
\end{enumerate}
\end{exercise}

\begin{solution}
\begin{enumerate}
\item מתקיים
\[\det\prs{B} = \det\prs{P^t A P} = \det\prs{P^t} \det\prs{A} \det\prs{P}\]
ולכן יש שוויון אם ורק אם
$\det\prs{P^t} \det\prs{P} = 1$.
אבל,
$\det\prs{P^t} = \det\prs{P}$
ולכן זה מתקיים אם ורק אם
$\det\prs{P} \in \set{\pm 1}$.
זה לא חייב להיות המצב. למשל, נוכל לקחת
$A = I_n, B = 4I_n, P = 2I_n$.

\item הדוגמא הקודמת מראה שהערכים העצמיים לא נשמרים.

\item הדרגה נשמרת כי
$P^t, P$
הפיכות.

\item נניח כי
$A$
הפיכה. אז
$B$
הפיכה כי הדרגות שלהן שוות.
נשים לב כי
\[B^{-1} = \prs{P^t A P}^{-1} = P^{-1} A^{-1} \prs{P^t}^{-1}\]
ועבור
$\tilde{P} = \prs{P^t}^{-1}$
נקבל
$B^{-1} = \tilde{P}^t A^{-1} \tilde{P}$.
\end{enumerate}
\end{solution}

\begin{algorithm}[דירוג סימולטני של מטריצות]
ניזכר שמטריצה הפיכה
$P$
הינה כפל של מטריצות אלמנטריות. כפל משמאל במטריצה אלמנטרית
$E$
מתאים לפעולת דירוג על השורות, וכפל ב־%
$E^t$
מתאים לפעולת דירוג זהה על העמודות (החלק השני נכון כי מתקיים
$A E^t = \prs{E A^t}^t$).

נתאר איך להיעזר בכך כדי לקבל שכל מטריצה
$A \in \Mat_n\prs{\FF}$
סימטרית
עבור
$\FF$
ממציין
$\Char\prs{\FF} \neq 2$
(כלומר,
$1+1 \neq 0$),
חופפת למטריצה אלכסונית.

\begin{enumerate}
\item יהי
$i = 1$.
\item אם
$\prs{A}_{1,1} \neq 0$,
נחלק לשני מקרים.
\begin{enumerate}[label=2.\arabic*]
\item 
אם
$\prs{A}_{1,j} = 0$
לכל
$j \in \brs{n}$,
המטריצה מהצורה
\begin{align*}
A = \pmat{0 & 0_{1 \times \prs{n-1}} \\ 0_{\prs{n-1} \times 1} & X}
\end{align*}
ואז די לבצע פעולות שידרגו את
$X$
למטריצה אלכסונית.

נעבור לשלב
$4$.

\item
אם קיים
$j \in \brs{n}$
עבורו
$\prs{A_{i,j}} \neq 0$,
נוסיף את השורה ה־%
$j$
לשורה ה־%
$i$,
ולאחר מכן את העמודה ה־%
$j$
לעמודה ה־%
$i$.

למשל, אם התחלנו אם המטריצה
$A \coloneqq \pmat{0 & 1 & 0 \\ 1 & 0 & 1 \\ 0 & 1 & 0}$,
נדרג
\begin{align*}
A &\xrightarrow{R_1 \mapsto R_1 + R_2} \pmat{1 & 1 & 1 \\ 1 & 0 & 1 \\ 0 & 1 & 0}
\\ &\xrightarrow{C_1 \mapsto C_1 + C_2} \pmat{2 & 1 & 1 \\ 1 & 0 & 1 \\ 1 & 1 & 0}
\end{align*}
ונזכור כי פעולת הדירוג התקבלה על ידי הצמדה
$E A E^t$
כאשר
\[\text{.} E = \pmat{1 & 0 & 0 \\ 1 & 1 & 0 \\ 0 & 0 & 1}\]

במטריצה החדשה שנסמנה $A$ במקום המטריצה הקודמת, המקדם ה־%
$\prs{i,i}$
שונה מאפס.

שימו לב שהמטריצה שהתקבלה אינה המטריצה
$A$
שהתחלנו איתה,
אך לצורך תיאור האלגוריתם, נתייחס אליה בתור
$A$
לאורכו.
\end{enumerate}

\item
לכל
$j > i$,
אם
$\prs{A}_{j,i} \neq 0$,
(או באופן שקול
$\prs{A}_{i,j} \neq 0$),
נוסיף כפולה של השורה ה־%
$i$
לשורה ה־%
$j$
שתאפס את המקדם
$\prs{A}_{j,i}$,
ולאחר מכן נוסיף את אותה כפולה של העמודה ה־%
$i$
לעמודה ה־%
$j$
(שתאפס את המקדם
$\prs{A}_{i,j}$).
למשל, אם התחלנו עם
\[A = \pmat{2 & 1 & 1 \\ 1 & 0 & 1 \\ 1 & 1 & 0}\]
ועבור
$i=1$,
נדרג תחילה
\begin{align*}
A &\xrightarrow{R_2 \mapsto R_2 - \frac{1}{2} R_1} \pmat{2 & 1 & 1 \\ 0 & -\frac{1}{2} & \frac{1}{2} \\ 1 & 1 & 0}
\\&\xrightarrow{C_2 \mapsto C_2 - \frac{1}{2} C_1} \pmat{2 & 0 & 1 \\ 0 & -\frac{1}{2} & \frac{1}{2} \\ 1 & \frac{1}{2} & 0} 
\end{align*}
ונזכור כי פעולת הדירוג מתקבלת על ידי הצמדה
$E A E^t$
כאשר
\[\text{.} E = \pmat{1 & 0 & 0 \\ -\frac{1}{2} & 1 & 0 \\ 0 & 0 & 0}\]
לאחר מכן, נעשה אותה דבר כדי לאפס את
$\prs{A}_{3,1}, \prs{A}_{1,3}$.

\item
אם
$i < n$,
נגדיל את
$i$
ב־%
$1$,
ונחזור לשלב
$2$.

\item
נסמן את המטריצות האלמנטריות שמצאנו במהלך החישוב בתור
$E_1, \ldots, E_k$
לפי הסדר. אז, אם
$A_0$
המטריצה שהתחלנו איתה, ו־%
$D$
המטריצה שקיבלנו בסוף, מתקיים
\[\text{.} D = E_k \cdot \ldots \cdot E_1 A_0 E_1^t \cdot \ldots \cdot E_k^t\]
נסמן
\begin{align*}
P \coloneqq \prs{E_k \cdot \ldots \cdot E_1}^t = E_1^t \cdot \ldots \cdot E_k^t
\end{align*}
ונקבל כי
\[\text{.} D = P^t A P\]
\end{enumerate}
\end{algorithm}

\begin{corollary}
יהי
$F$
שדה ממציין שונה מ־%
$2$,
כלומר שדה בו
$1 + 1 \neq 0$.
אז, כל מטריצה סימטרית מעל
$\FF$
חופפת למטריצה אלכסונית.
\end{corollary}

\begin{exercise}
יהי
$V$
מרחב וקטורי ממימד $n \in \NN_+$ מעל שדה
$\FF$
עם מציין
$\mrm{char}\prs{F} \neq 2$,
ותהי
$f \in \mrm{Bil}\prs{V}$
תבנית בילינארית סימטרית על
$V$.

הראו כי קיים בסיס
$C$
עבורו
$\brs{f}_C$
אלכסונית.
\end{exercise}

\begin{solution}
יהי
$B$
בסיס של
$V$
ותהי
$A \coloneqq \brs{f}_B$.
אז
$A$
סימטרית, ולפי המסקנה קיימת מטריצה
$P \in \Mat_n\prs{\FF}$
הפיכה עבורה
$D \coloneqq P^t A P$.
אז
\begin{align*}
\text{.} D = P^t A P = P^t \brs{f}_B P
\end{align*}
לפי סעיף 3 של תרגיל 5 בגיליון תרגילים 1, קיים בסיס
$C$
עבורו
$P^{-1} = M^B_C$,
כיוון ש־%
$P^{-1}$
הפיכה. לכן,
$P = \prs{M^B_C}^{-1} = M^C_B$
ולכן
\begin{align*}
\text{,} D = \prs{M^C_B}^t \brs{f}_B M^C_B = \brs{f}_C
\end{align*}
כאשר בשוויון האחרון השתמשנו במסקנה
\ref{corollary:change-of-bases}
לגבי מעבר בסיס.
\end{solution}

\begin{exercise}
\begin{enumerate}
    \item תהי
\[\text{.} A = \pmat{2 & 1 & 2 \\ 1 & 2 & 1 \\ 2 & 1 & 2} \in \Mat_3\prs{\mbb{Z}/5\mbb{Z}}\]
מיצאו מטריצה
$P \in \Mat_3\prs{\mbb{Z}/5\mbb{Z}}$
הפיכה
עבורה
$P^t A P$
אלכסונית.

\item האם יכולנו לקחת מטריצה
$P'$
אחרת שתיתן תשובה מתאימה?

\item האם יש מטריצה
$Q \in \Mat_3\prs{\mbb{Z}/5\mbb{Z}}$
הפיכה עבורה
$Q^t A Q = I_3$?

\item האם יש מטריצה
$Q \in \Mat_3\prs{\mbb{Z}/5}$
הפיכה עבורה
$Q^t B Q = I_3$
כאשר
$B = \pmat{2 & 1 & 2 \\ 1 & 1 & 0 \\ 2 & 0 & 1}$?
\end{enumerate}
\end{exercise}

\begin{solution}
\begin{enumerate}
\item 

נדרג את
$A$
למטריצה אלכסונית כאשר בכל שלב נבצע פעולת דירוג שורה ולאחריה פעולת דירוג עמודה מתאימה.

\begin{align*}
    A &= \pmat{2 & 1 & 2 \\ 1 & 2 & 1 \\ 2 & 1 & 2}
    \\&\xrightarrow{R_2 \mapsto R_2 + 2 R_1}
    \pmat{2 & 1 & 2 \\ 0 & 4 & 0 \\ 2 & 1 & 2}
    \\&\xrightarrow{C_2 \mapsto C_2 + 2 C_1}
    \pmat{2 & 0 & 2 \\ 0 & 4 & 0 \\ 2 & 0 & 2}
    \\&\xrightarrow{R_3 \mapsto R_3 - R_1}
    \pmat{2 & 0 & 2 \\ 0 & 4 & 0 \\ 0 & 0 & 0}
    \\&\xrightarrow{C_3 \mapsto C_3 - C_1}
    \pmat{2 & 0 & 0 \\ 0 & 4 & 0 \\ 0 & 0 & 0}
\end{align*}

נקבל כי
\[P^t = \pmat{1 & 0 & 0 \\ 0 & 1 & 0 \\ -1 & 0 & 1} \pmat{1 & 0 & 0 \\ 2 & 1 & 0 \\ 0 & 0 & 1} = \pmat{1 & 0 & 0 \\ 2 & 1 & 0 \\ -1 & 0 & 1}\]
מקיימת את הנדרש, ולכן ניקח
\[\text{.} P = \pmat{1 & 2 & -1 \\ 0 & 1 & 0 \\ 0 & 0 & 1}\]

\item כן. למשל, יכולנו קודם להחליף את השורות הראשונה והשלישית, ולקבל
\[\prs{P'}^t = P^t \pmat{0 & 0 & 1 \\ 0 & 1 & 0 \\ 1 & 0 & 0} = \pmat{-1 & 0 & 1 \\ 2 & 1 & 0 \\ 1 & 0 & 0} \neq P^t\]
ואז
$P' \neq P$.

\item קיבלנו
$P^t A P = \diag\prs{2,4,0}$
וזאת אינה מטריצה מדרגה מלאה. דרגה נשמרת תחת חפיפת מטריצות, ולכן לא יתכן שיש
$Q$
כנדרש.

\item ננסה דירוג לפי שורה ועמודה.
\begin{align*}
    B &= \pmat{2 & 1 & 2 \\ 1 & 1 & 0 \\ 2 & 0 & 1}
    \\&\xrightarrow{R_3 \mapsto R_3 - R_1}
    \pmat{2 & 1 & 2 \\ 1 & 1 & 0 \\ 0 & 4 & 4}
    \\&\xrightarrow{C_3 \mapsto C_3 - C_1}
    \pmat{2 & 1 & 0 \\ 1 & 1 & 4 \\ 0 & 4 & 4}
    \\&\xrightarrow{R_2 \mapsto R_2 + 2 R_1}
    \pmat{2 & 1 & 0 \\ 0 & 3 & 4 \\ 0 & 4 & 4}
    \\&\xrightarrow{C_2 \mapsto C_2 + 2 C_1}
    \pmat{2 & 0 & 0 \\ 0 & 3 & 4 \\ 0 & 4 & 4}
    \\&\xrightarrow{R_3 \mapsto R_3 + 2 R_2}
    \pmat{2 & 0 & 0 \\ 0 & 3 & 4 \\ 0 & 0 & 2}
    \\&\xrightarrow{C_3 \mapsto C_3 + 2 C_2}
    \pmat{2 & 0 & 0 \\ 0 & 3 & 0 \\ 0 & 0 & 2}
\end{align*}
כעת, נוכל לחלק איברים על האלכסון רק בריבוע, כיוון שיש לכפול גם את השורה וגם את העמודה באותו מספר. כיוון ש־%
$3$
אינו ריבוע ב־%
$\mbb{Z}/5\mbb{Z}$
לא נוכל לקבל ככה את מטריצת היחידה.
גם, הדטרמיננטה של המטריצה שקיבלנו היא
$2 \cdot 3 \cdot 2 \equiv 2$.
הדטרמיננטה של מטריצות חופפות נבדלת בכפל בריבוע, אך לא יתכן
$2 = a^2 \det\prs{I_3} = a^2$
כי
$2$
אינו ריבוע.

השיטה הזאת לא עובדת תמיד. יתכן מצב בו מטריצה אינה חופפת ליחידה אך כן בעלת דטרמיננטה שהינה ריבוע.
למשל, המטריצה
$\pmat{0 & 1 \\ 1 & 0} \in \Mat_3\prs{\mbb{Z}/5\mbb{Z}}$
כזאת.
\end{enumerate}
\end{solution}

\begin{exercise}
יהי
$V$
מרחב וקטורי סוף־מימדי מעל
$\mbb{R}$.
תהי
$g$
מכפלה פנימית על
$V$
ותהי
$h$
תבנית בילינארית סימטרית על
$V$.
הוכיחו שקיים בסיס
$B$
של
$V$
עבורו
$\brs{g}_B, \brs{h}_B$
שתיהן אלכסוניות.
\end{exercise}

\begin{solution}
נסמן
$n \coloneqq \dim_{\mbb{R}}\prs{V}$.
יהי
$E$
בסיס אורתונורמלי של
$V$
ביחס ל־%
$g$,
שקיים לפי גרם־שמידט. אז
$\brs{g}_E = I_n$.
כעת,
$\brs{h}_E$
סימטרית כי
$h$
סימטרית, ולכן יש ממשפט הפירוק הסקפטרלי מטריצה
$Q \in \Mat_n\prs{\mbb{R}}$
אורתוגונלית עבורה
$Q^t \brs{h}_E Q$
אלכסונית.
נרצה שיתקיים
$Q = P^B_E$ עבור בסיס
$B$,
ולכן נגדיר
$B \coloneqq \prs{Q^{-1} v_1, \ldots, Q^{-1} v_n}$
כאשר
$E = \prs{v_1, \ldots, v_n}$.
אז
\begin{align*}
\brs{g}_B &= \prs{P^B_E}^t \brs{g}_E P^B_E = Q^t \brs{g}_E Q = Q^t I_n Q = Q^t Q = I_n \\
\brs{h}_B &= \prs{P^B_E}^t \brs{h}_E P^B_E = Q^t \brs{h}_E Q
\end{align*}
שתיהן אלכסוניות, כנדרש.
\end{solution}

\section{חוק האינרציה של סילבסטר}

\begin{theorem}[סילבסטר]
תהי
$A \in \Mat_n\prs{\mbb{R}}$
סימטרית.

מספר הערכים החיוביים%
\textbackslash%
שליליים על האלכסון של מטריצה אלכסונית החופפת ל־%
$A$
אינו תלוי בבחירה של אותה מטריצה.
\end{theorem}

\begin{corollary}
תהי
$A \in \Mat_n\prs{\mbb{R}}$
סימטרית.
$A$
חופפת למטריצה יחידה מהצורה
$\pmat{I_{\prs{n_+}} & & \\ & - I_{\prs{n_{-}}} & \\ & & 0_{\prs{n_0}}}$.
מטריצה זאת נקראת
\emph{צורת סילבסטר הקנונית של
$A$}.
$n_+$
ו־%
$n_-$
נקראים
\emph{אינדקס האינרציה החיובי והשלילי}
של
$A$,
בהתאמה.
ההפרש
$n_+ - n_-$
נקרא
\emph{הסיגנטורה}
של
$A$.
\end{corollary}

\begin{remark}
משפט סילבסטר אומר באופן שקול שלכל תבנית בילינארית סימטרית
$g$
קיים בסיס
$C$
עבורו
$\brs{g}_C$
בצורת סילבסטר.
\end{remark}

\begin{proposition}
המספר
$n_+$
הוא סכום הריבויים האלגבריים של הערכים העצמיים החיוביים של
$A$,
המספר
$n_{-}$
הוא סכום הריבויים האלגבריים של הערכים העצמיים השליליים, והמספר
$n_0$
הוא מימד הגרעין.
\end{proposition}

\begin{exercise}
תהי
$A \in \Mat_n\prs{\mbb{R}}$
סימטרית.
\begin{enumerate}
\item הוכיחו כי
$A, A^3$
חופפות.
\item הוכיחו כי אם
$A$
הפיכה, היא חופפת ל־%
$A^{-1}$.
\item תהי
$B \in \Mat_n\prs{\mbb{R}}$
סימטרית
ונניח כי
\begin{align*}
p_A\prs{x} &= x^2 - 2 \\
\text{.} p_B\prs{x} &= x^2 + 2x - 3
\end{align*}
האם
$A,B$
בהכרח חופפות?
\end{enumerate}
\end{exercise}

\begin{solution}
\begin{enumerate}
\item
ל־%
$A,A^3$
יש אותם אינדקסיי אינרציה כי הערכים העצמיים של
$A^3$
הם
$\lambda^3$
עבור
$\lambda$
ערך עצמי של
$A$.
לכן ל־%
$A,A^3$
אותה צורת סילבסטר, ולכן הן חופפות.

\item כמו מקודם, כאשר הערכים העצמיים של
$A^{-1}$
הם
$\frac{1}{\lambda}$
עבור
$\lambda$
ערך עצמי של
$A$.

\item נמצא שורשים של הפולינומים.
\begin{align*}
x^2 - 2 &= \prs{x+\sqrt{2}}\prs{x-\sqrt{2}} \\
x^2 + 2x - 3 &= x^2 + 3x - \prs{x+3} = \prs{x-1}\prs{x+3}
\end{align*}
לכן לכל אחד מהפולינומים ערך עצמי אחד חיובי ואחד שלילי. לכן צורות סילבסטר של
$A,B$
שתיהן
$\pmat{1 & 0 \\ 0 & -1}$
ולכן הן חופפות.
\end{enumerate}
\end{solution}

\begin{exercise}
מצאו את צורת סילבסטר של
\[\text{.} A = \pmat{0 & 1 & 1 & \cdots & 1 \\ 1 & 0 & -1 & \cdots & -1 \\ 1 & -1 & 0 & \cdots & -1 \\ \vdots & \vdots & \vdots & \ddots & \vdots \\ 1 & -1 & -1 & \cdots & 0} \in \Mat_n\prs{\mbb{R}}\]
\end{exercise}

\begin{solution}
כדי למצוא את צורת סילבסטר, נחפש ערכים עצמיים.
נשים לב כי
$1$
ערך עצמי מריבוי
$n-1$.
אז הערך העצמי הנוסף הוא
$\tr\prs{A} -\prs{n-1}\cdot 1 = -n+1$.
אם
$n = 1$
מתקיים
$A = \prs{0}$.
אחרת צורת סילבסטר היא
$\pmat{I_{n-1} & \\ & -1}$.
\end{solution}

\begin{algorithm}
כדי למצוא בסיס
$C$
עבורו
$\brs{g}_C$
בצורת סילבסטר נבצע את השלבים הבאים.

\begin{enumerate}
\item נמצא בסיס
$\tilde{C} = \prs{v_1, \ldots, v_n}$
של
$V$
עבורו
$\brs{g}_{\tilde{C}}$
אלכסונית, בעזרת לכסון אורתוגונלי.

\item
נגדיר
\[\text{,} u_i = \fcases{\frac{v_i}{\sqrt{\abs{\lambda_i}}} & \lambda_i \neq 0 \\ v_i & \lambda_i = 0}\]
כדי לקבל
$g\prs{u_i, u_i} \in \set{0, -1, 1}$,
לכל
$i$.

\item על ידי בחירת סדר מתאים של ה־%
$u_i$
נקבל בסיס
$C$
המקיים את הנדרש.
\end{enumerate}
\end{algorithm}

\begin{exercise}
תהי
$A = \pmat{2 & 1 & 1 \\ 1 & 2 & 1 \\ 1 & 1 & 2}$.
מצאו מטריצה
$P \in \Mat_3\prs{\mbb{R}}$
עבורה
$P^t A P$
בצורת סילבסטר.
\end{exercise}

\begin{solution}
נתחיל במציאת ערכים עצמיים של
$A$.
נשים לב כי
$1$
ערך עצמי של
$A$
מריבוי
$2$.
הערך העצמי הנוסף הוא
$\tr\prs{A} - 2\cdot 1 = 4$.
בסיס מתאים הוא
$B = \prs{\pmat{1\\1\\1}, \pmat{1\\0\\-1}, \pmat{0\\1\\-1}}$.

נבצע את תהליך גרם־שמידט על כל אחד מהמרחבים העצמיים, כדי לקבל בסיס אורתונורמלי
\[\text{.} \tilde{C} \ceq \prs{v_1, v_2, v_3} = \prs{\pmat{1/\sqrt{3} \\ 1/\sqrt{3} \\ 1/\sqrt{3}}, \pmat{1/\sqrt{2} \\ 0 \\ -1/\sqrt{2}}, \pmat{-\sqrt{6}/6 \\ \sqrt{6}/3 \\ -\sqrt{6}/6}}\]
נחלק את הוקטורים העצמיים ב־%
$\sqrt{\lambda_i}$
ונקבל בסיס
\[\text{.} C \ceq \prs{\pmat{1/2\sqrt{3}\\1/2\sqrt{3}\\1/2\sqrt{3}}, \pmat{1/\sqrt{2} \\ 0 \\ -1/\sqrt{2}}, \pmat{-\sqrt{6}/6 \\ \sqrt{6}/3 \\ -\sqrt{6}/6}}\]
תהי
\[P = \pmat{1/2\sqrt{3} & 1/\sqrt{2} & -\sqrt{6}/6 \\ 1/2\sqrt{3} & 0 & \sqrt{6}/3 \\ 1/2\sqrt{3} & -1/\sqrt{2} & -\sqrt{6}/6}\]
ונקבל כי
\[P^t A P = I_3\]
בצורת סילבסטר.
\end{solution}

\begin{remark}
לפעמים חישוב הערכים העצמיים יכול להיות מסובך. נוכל בעצם למצוא את צורת סילבסטר ואת הבסיס המתאים לה בלי מציאת המרחבים העצמיים.
לשם כך, נשים לב כי אם
$E$
מטריצה אלמנטרית, הכפל
$A \mapsto A E^t$
נקבע על פי אותן פעולות על עמודות
$A$
כמו פעולות
$E$
על השורות. נוכל אם כן לדרג את
$A$
לפי שורה ועמודה במקביל, כאשר בכל שלב נבצע את אותה פעולה על השורות ואז על העמודות (או להיפך). נקבל מטריצה
$P$
כמכפלת המטריצות האלמנטריות, שעבורה
$PAP^t$
תהיה בצורת סילבסטר.
\end{remark}

\printbibliography
\end{document}
