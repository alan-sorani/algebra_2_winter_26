\documentclass[a4paper,10pt,twoside,openany]{article}

\usepackage[lang=hebrew]{maths}
\usepackage{polynom}
\usepackage{hebrewdoc}
\usepackage{stylish}
\usepackage{lipsum}
\let\bs\blacksquare

\setlength{\parindent}{0pt}

%%%%%%%%%%%%
% Styling %
%%%%%%%%%%%%

\usepackage{enumitem}

%%%%%%%%%%%%%
% Counters  %
%%%%%%%%%%%%%

\setcounter{section}{0}     
            
%BIBLIOGRAPHY
\usepackage[
backend=biber,
style=alphabetic,
]{biblatex}
\addbibresource{bibliography.bib} %Imports bibliography file

\title{
אלגברה ב' (01040168) - חורף 2026
\\
תרגול 1 - מטריצות מייצגות, ומרחבי העתקות לינאריות
\\
אלן סורני
\\
הרשימות עודכנו לאחרונה בתאריך ה־%
\today
}
\date{}

\begin{document}
\maketitle

\section*{סימונים}

\begin{itemize}
\item[-]
$\brs{n} = \set{1, \ldots, n}$.
\item[-]
$\mbb{N} = \set{0, 1, 2, 3, \ldots}$
קבוצת השלמים האי־שליליים.
\item[-]
$\mbb{N}_+ = \set{1, 2, 3, \ldots}$
קבוצה השלמים החיוביים.
\item[-]
$\sum_{i = 1}^n a_i = \sum_{i \in \brs{n}} a_i = \sum_{i=1}^n a_i = a_1 + a_2 + \ldots + a_n$
\item[-] $\Mat_{m \times n}\prs{\mbb{F}}$ הוא מרחב המטריצות עם
$m$
שורות ו־%
$n$
עמודות, עם מקדמים בשדה
$\mbb{F}$.
\item[-]
$\mbb{F}^n = \Mat_{n \times 1}\prs{\mbb{F}}$
\item[-]
$\Mat_{n}\prs{\mbb{F}} = \Mat_{n \times n}\prs{\mbb{F}}$
\item[-]
$\hom_{\mbb{F}}\prs{V,W}$
הוא מרחב ההעתקות הלינאריות
$V \to W$
כאשר
$V,W$
מרחבים וקטוריים מעל
$\mbb{F}$.
\item[-]
$\End_{\mbb{F}}\prs{V} = \hom_{\mbb{F}}\prs{V,V}$
\end{itemize}

\section{וקטורי קואורדינטות ומטריצות מייצגות}

\begin{definition}[וקטור קואורדינטות לפי בסיס]
יהי
$V$
מרחב וקטורי ממימד
$n \in \mbb{N}$
מעל שדה
$\mbb{F}$,
ויהי
$B = \prs{v_1, \ldots, v_n}$
בסיס של
$V$.
כל
$v \in V$
ניתן לכתוב בצורה יחידה בתור
\[v = \alpha_1 v_1 + \ldots + \alpha_n v_n\]
עבור
$\alpha_i \in \mbb{F}$.
נגדיר את
\emph{וקטור הקואורדינטות של
$v$
כזה לפי הבסיס
$B$}
בתור
\[\text{.} \brs{v}_B = \pmat{\alpha_1 \\ \vdots \\ \alpha_n}\]
\end{definition}

\begin{definition}[מטריצה מייצגת]
יהיו
$V,W$
מרחבים וקטוריים ממימדים
$n,m \in \mbb{N}$
בהתאמה מעל אותו שדה
$\mbb{F}$,
ותהי
$T \in \Hom_{\mbb{F}}\prs{V,W}$
העתקה לינארית מ־%
$V$
ל־%
$W$.
יהיו
\begin{align*}
B = \prs{v_1, \ldots, v_n}
\end{align*}
ו־%
$C$
בסיסים של
$V$
ו־%
$W$
בהתאמה.

נגדיר את
\emph{המטריצה המייצגת}
$\brs{T}^B_C$
בתור
\begin{align*}
\text{.} \brs{T}^B_C \coloneqq \pmat{\vert & & \vert \\ \brs{T\prs{v_1}}_C & \cdots & \brs{T\prs{v_n}}_C \\ \vert & & \vert}
\end{align*}

אם
$V = W$
וכן
$B = C$,
נסמן
\begin{align*}
\text{.} \brs{T}_B \coloneqq \brs{T}^B_B
\end{align*}
\end{definition}

\begin{proposition} \label{proposition:matrix-of-composition}
יהיו
$U,V,W$
מרחבים וקטוריים סוף־מימדיים מעל שדה
$\mbb{F}$,
עם בסיסים
$B,C,D$
בהתאמה,
ותהיינה
$S \in \Hom_{\mbb{F}}\prs{U,V}$
ו־%
$T \in \Hom_{\mbb{F}}\prs{V,W}$.

אז
\begin{align*}
\text{.} \brs{T \circ S}^B_D = \brs{T}^C_D \brs{S}^B_C
\end{align*}
\end{proposition}

\begin{definition}[מטריצת מעבר בסיס]
יהי
$V$
מרחב וקטורי ממימד
$n \in \mbb{N}$,
עם בסיסים
$B,C$,
ותהי
$\id_V$
העתקת הזהות על
$V$.
נגדיר
\begin{align*}
\text{.} M^B_C = \brs{\id_V}^B_C
\end{align*}
\end{definition}

\begin{proposition} \label{proposition:change-of-basis-matrices}
יהי
$V$
מרחב וקטורי סוף־מימדי מעל שדה
$\mbb{F}$,
ויהיו
$B,C,D$
שלושה בסיסים של
$V$.

אז:
\begin{enumerate}
\item $M^B_C$
מטריצה הפיכה, ומתקיים
$\prs{M^B_C}^{-1} = M^C_B$.
\item $M^B_C = M^D_C M^B_D$.
\end{enumerate}
\end{proposition}

\begin{remark} \label{remark:change-of-basis-matrices}
למטריצה
$M^B_C$
\emph{לא נקרא}
מטריצת מעבר מהבסיס
$B$
לבסיס
$C$
או להיפך.

מצד אחד, היא מקיימת
\[\text{,} \forall v \in V : M^B_C \brs{v}_B = \brs{v}_C\]
ולכן יש הקוראים לה מטריצת מעבר מהבסיס
$B$
לבסיס
$C$.

מצד שני, אם
$V = \mbb{F}^n$,
\[{\mrm{St}} = \prs{e_1, \ldots, e_n}\]
הבסיס הסטנדרטי של
$V$,
ונכתוב
\begin{align*}
B &= \St \\
\text{,} C &= \prs{c_1, \ldots, c_n}
\end{align*}
נקבל כי
\begin{align*}
M^B_C c_i &= M^{\St}_C \brs{c_i}_{\St}
\\&= \brs{c_i}_{C}
\\&= e_i
\end{align*}
כלומר
\begin{align*}
\text{.} \prs{M^B_C c_1, \ldots, M^B_C c_n} = B
\end{align*}
לכן, יש הקוראים למטריצה
$M^B_C$
מטריצת מעבר מהבסיס
$C$
לבסיס
$B$.

כדי למנוע בלבול, במקום להתייחס למטריצה כזאת כמטריצת מעבר מבסיס זה או אחר, נכתוב מפורשות $M^B_C$.
\end{remark}

\begin{exercise}
יהי
$V = \Mat_2\prs{\mbb{R}}$
עם הבסיס
\[\text{.} \St_V \coloneqq \prs{\pmat{1 & 0 \\ 0 & 0}, \pmat{0 & 1 \\ 0 & 0}, \pmat{0 & 0 \\ 1 & 0}, \pmat{0 & 0 \\ 0 & 1}}\]
בסיס זה נקרא
\emph{הבסיס הסטנדרטי של $\Mat_2\prs{\mbb{R}}$}.

\begin{enumerate}
\item
יהי
\begin{align*}
T \colon V &\to V \\
\text{.} \hphantom{lala} A &\mapsto A^t
\end{align*}
חשבו את
$\brs{T}_{\St_V}$.

\item
יהי
\begin{align*}
T \colon V &\to V \\
\text{.} \hphantom{lala} A &\mapsto A^2
\end{align*}
חשבו את
$\brs{T}_{\St_V}$.

\item
תהי
\begin{align*}
\text{.} A &= \pmat{1 & 1 & 0 & 0 \\ 0 & 1 & 0 & 0 \\ 0 & 0 & 1 & 1 \\ 0 & 0 & 0 & 1}
\end{align*}
מצאו אופרטור
$T \in \End_{\mbb{R}}\prs{V}$
עבורו
$\brs{T}_{\St_2} = A$.
\end{enumerate}
\end{exercise}

\begin{solution}
\begin{enumerate}
\item
נחשב את
$\brs{T}_{\St_V}$
לפי ההגדרה ונקבל:
\begin{align*}
\brs{T}_{\St_V} &= \pmat{\vert & \vert & \vert & \vert \\
\brs{T\prs{\pmat{1 & 0 \\ 0 & 0}}}_{\St_V} & \brs{T\prs{\pmat{0 & 1 \\ 0 & 0}}}_{\St_V} & \brs{T\prs{\pmat{0 & 0 \\ 1 & 0}}}_{\St_V} & \brs{T\prs{\pmat{0 & 0 \\ 0 & 1}}}_{\St_V} \\
\vert & \vert & \vert & \vert}
\\&=
\pmat{\vert & \vert & \vert & \vert \\
\brs{\pmat{1 & 0 \\ 0 & 0}}_{\St_V} & \brs{\pmat{0 & 0 \\ 1 & 0}}_{\St_V} & \brs{\pmat{0 & 1 \\ 0 & 0}}_{\St_V} & \brs{\pmat{0 & 0 \\ 0 & 1}}_{\St_V} \\
\vert & \vert & \vert & \vert}
\\&=
\pmat{1 & 0 & 0 & 0 \\
0 & 0 & 1 & 0 \\
0 & 1 & 0 & 0 \\
0 & 0 & 0 & 1}
\end{align*}

\item
נחשב את
$\brs{T}_{\St_V}$
לפי ההגדרה ונקבל:
\begin{align*}
\brs{T}_{\St_V} &= \pmat{\vert & \vert & \vert & \vert \\
\brs{T\prs{\pmat{1 & 0 \\ 0 & 0}}}_{\St_V} & \brs{T\prs{\pmat{0 & 1 \\ 0 & 0}}}_{\St_V} & \brs{T\prs{\pmat{0 & 0 \\ 1 & 0}}}_{\St_V} & \brs{T\prs{\pmat{0 & 0 \\ 0 & 1}}}_{\St_V} \\
\vert & \vert & \vert & \vert}
\\&=
\pmat{\vert & \vert & \vert & \vert \\
\brs{\pmat{1 & 0 \\ 0 & 0}}_{\St_V} & \brs{\pmat{0 & 0 \\ 0 & 0}}_{\St_V} & \brs{\pmat{0 & 0 \\ 0 & 0}}_{\St_V} & \brs{\pmat{0 & 0 \\ 0 & 1}}_{\St_V} \\
\vert & \vert & \vert & \vert}
\\&=
\pmat{1 & 0 & 0 & 0 \\
0 & 0 & 0 & 0 \\
0 & 0 & 0 & 0 \\
0 & 0 & 0 & 1}
\end{align*}

\item
לפי הגדרת מטריצה מייצגת, עבור
$T \in \End_{\mbb{R}}\prs{V}$
מתקיים
\begin{align*}
\text{.} \brs{T}_{\St_V} = \pmat{\vert & \vert & \vert & \vert \\
\brs{T\prs{\pmat{1 & 0 \\ 0 & 0}}}_{\St_V} & \brs{T\prs{\pmat{0 & 1 \\ 0 & 0}}}_{\St_V} & \brs{T\prs{\pmat{0 & 0 \\ 1 & 0}}}_{\St_V} & \brs{T\prs{\pmat{0 & 0 \\ 0 & 1}}}_{\St_V} \\
\vert & \vert & \vert & \vert}
\end{align*}
זה שווה ל־%
$A$
אם ורק אם יש שוויון בין כל העמודות בין המטריצות:
\begin{align*}
\brs{T\prs{\pmat{1 & 0 \\ 0 & 0}}}_{\St_V} &= \pmat{1 \\ 0 \\ 0 \\ 0} \\
\brs{T\prs{\pmat{0 & 1 \\ 0 & 0}}}_{\St_V} &= \pmat{1 \\ 1 \\ 0 \\ 0} \\
\brs{T\prs{\pmat{0 & 0 \\ 1 & 0}}}_{\St_V} &= \pmat{0 \\ 0 \\ 1 \\ 0} \\
\brs{T\prs{\pmat{0 & 0 \\ 0 & 1}}}_{\St_V} &= \pmat{0 \\ 0 \\ 1 \\ 1}
\end{align*}
לפי הגדרת וקטורי קואורדינטות זה שקול לכך שמתקיימים השוויונות הבאים:
\begin{align}\label{eq:system-of-equation}
\begin{split}
T\prs{\pmat{1 & 0 \\ 0 & 0}} &= \pmat{1 & 0 \\ 0 & 0} \\
T\prs{\pmat{0 & 1 \\ 0 & 0}} &= \pmat{1 & 0 \\ 0 & 0} + \pmat{0 & 1 \\ 0 & 0} = \pmat{1 & 1 \\ 0 & 0} \\
T\prs{\pmat{0 & 0 \\ 1 & 0}} &= \pmat{0 & 0 \\ 1 & 0} \\
T\prs{\pmat{0 & 0 \\ 0 & 1}} &= \pmat{0 & 0 \\ 1 & 0} + \pmat{0 & 0 \\ 0 & 1} = \pmat{0 & 0 \\ 1 & 1}
\end{split}
\end{align}
מלינאריות של העתקות לינאריות נקבל כי במקרה זה
\begin{align*}
T\prs{\pmat{a & b \\ c & d}} &= a T\prs{\pmat{1 & 0 \\ 0 & 0}} + b T\prs{\pmat{0 & 1 \\ 0 & 0}} + c T\prs{\pmat{0 & 0 \\ 1 & 0}} + d T\prs{\pmat{0 & 0 \\ 0 & 1}}
\\&=
\pmat{a & 0 \\ 0 & 0} + \pmat{b & b \\ 0 & 0} + \pmat{0 & 0 \\ c & 0} + \pmat{0 & 0 \\ d & d}
\\&=
\pmat{a + b & b \\ c + d & d}
\end{align*}
לכל
$a,b,c,d \in \mbb{R}$.
להיפך, אופרטור
$T$
המקיים
\begin{align*}
\forall a,b,c,d \in \mbb{R} : T\prs{\pmat{a & b \\ c & d}} = \pmat{a + b & b \\ c + d & d}
\end{align*}
מקיים בפרט את השוויונים ב־%
\eqref{eq:system-of-equation}.
לכן, האופרטור
$T \in \End_{\mbb{R}}\prs{V}$
היחיד עבורו
$\brs{T}_{\St_V} = A$
הוא
\begin{align*}
T \colon V &\to V \\
\text{.} \pmat{a & b \\ c & d} &\mapsto \pmat{a + b & b \\ c + d & d}
\end{align*}
\end{enumerate}
\end{solution}

\begin{exercise}
יהי
$V = \mbb{C}_{\geq 2}\brs{x}$
מרחב הפולינומים ממעל לכל היותר
$2$
עם מקדמים מרוכבים, יהי
$T \in \End_{\mbb{C}}\prs{V}$,
ויהיו
\begin{align*}
B &\coloneqq \prs{1+x, x+x^2, 1+x^2} \\
C &\coloneqq \prs{x^2, x, 1}
\end{align*}
שני בסיסים של
$V$.

ידוע כי
\[\text{.}\brs{T}_B = \pmat{0 & 1 & 0 \\ 0 & 0 & 1 \\ 0 & 0 & 0}\]
חשבו את
$\brs{T}_C$.
\end{exercise}

\begin{solution}
ניעזר בטענה
\ref{proposition:matrix-of-composition}
כדי לקבל כי
\begin{align*}
\text{.} \brs{T}_C = M^B_C \brs{T}^C_B = M^B_C \brs{T}_B M^C_B
\end{align*}
כעת
\begin{align*}
\text{.} M^B_C = \pmat{\vert & \vert & \vert \\ \brs{1+x}_C & \brs{x + x^2}_C & \brs{1+x^2}_C \\ \vert & \vert & \vert} = \pmat{0 & 1 & 1 \\ 1 & 1 & 0 \\ 1 & 0 & 1}
\end{align*}
לפי טענה
\ref{proposition:change-of-basis-matrices}
מתקיים
$M^C_B = \prs{M^B_C}^{-1}$.
נהפוך את המטריצה
$M^B_C$.

\begin{align*}
\left( \begin{array}{@{}c|c@{}}
	\mat{0 & 1 & 1 \\ 1 & 1 & 0 \\ 1 & 0 & 1} & \mat{1 & 0 & 0 \\ 0 & 1 & 0 \\ 0 & 0 & 1}
\end{array} \right)
&\xrightarrow{R_1 \leftrightarrow R_2}
\left( \begin{array}{@{}c|c@{}}
	\mat{1 & 1 & 0 \\ 0 & 1 & 1 \\ 1 & 0 & 1} & \mat{0 & 1 & 0 \\ 1 & 0 & 0 \\ 0 & 0 & 1}
\end{array} \right)
\\&\xrightarrow{R_3 \rightarrow R_3 - R_1}
\left( \begin{array}{@{}c|c@{}}
	\mat{1 & 1 & 0 \\ 0 & 1 & 1 \\ 0 & -1 & 1} & \mat{0 & 1 & 0 \\ 1 & 0 & 0 \\ 0 & -1 & 1}
\end{array} \right)
\\&\xrightarrow{R_3 \rightarrow R_3 + R_2}
\left( \begin{array}{@{}c|c@{}}
	\mat{1 & 1 & 0 \\ 0 & 1 & 1 \\ 0 & 0 & 2} & \mat{0 & 1 & 0 \\ 1 & 0 & 0 \\ 1 & -1 & 1}
\end{array} \right)
\\&\xrightarrow{R_3 \rightarrow \frac{1}{2} R_3}
\left( \begin{array}{@{}c|c@{}}
	\mat{1 & 1 & 0 \\ 0 & 1 & 1 \\ 0 & 0 & 1} & \mat{0 & 1 & 0 \\ 1 & 0 & 0 \\ 1/2 & -1/2 & 1/2}
\end{array} \right)
\\&\xrightarrow{R_2 \rightarrow R_2 - R_3}
\left( \begin{array}{@{}c|c@{}}
	\mat{1 & 1 & 0 \\ 0 & 1 & 0 \\ 0 & 0 & 1} & \mat{0 & 1 & 0 \\ 1/2 & 1/2 & -1/2 \\ 1/2 & -1/2 & 1/2}
\end{array} \right)
\\&\xrightarrow{R_1 \rightarrow R_1 - R_2}
\left( \begin{array}{@{}c|c@{}}
	\mat{1 & 0 & 0 \\ 0 & 1 & 0 \\ 0 & 0 & 1} & \mat{-1/2 & 1/2 & 1/2 \\ 1/2 & 1/2 & -1/2 \\ 1/2 & -1/2 & 1/2}
\end{array} \right)
\end{align*}

לכן
\begin{align*}
\text{,} M^C_B = \prs{M^B_C}^{-1} = \pmat{-1/2 & 1/2 & 1/2 \\ 1/2 & 1/2 & -1/2 \\ 1/2 & -1/2 & 1/2}
\end{align*}
ונקבל כי
\begin{align*}
\brs{T}_C &= M^B_C \brs{T}_B M^C_B
\\&= \pmat{0 & 1 & 1 \\ 1 & 1 & 0 \\ 1 & 0 & 1} \pmat{0 & 1 & 0 \\ 0 & 0 & 1 \\ 0 & 0 & 0} \pmat{-1/2 & 1/2 & 1/2 \\ 1/2 & 1/2 & -1/2 \\ 1/2 & -1/2 & 1/2}
\\ \text{,} \hphantom{\brs{T}_C} &= \pmat{1/2 & -1/2 & 1/2 \\ 1 & 0 & 0 \\ 1/2 & 1/2 & -1/2}
\end{align*}
כנדרש.
\end{solution}

\newpage

\section{מרחבי העתקות לינאריות}

\begin{definition}[מרחב העתקות לינאריות]
יהיו
$V,W$
מרחבים וקטוריים מעל שדה
$\mbb{F}$.

נסמן את
\emph{מרחב ההעתקות הלינאריות מ־%
$V$
ל־%
$W$}
בתור
$\Hom_{\mbb{F}}\prs{V,W}$.
זהו מרחב וקטורי.

איברי
$\Hom_{\mbb{F}}\prs{V,W}$
נקראים גם
\emph{הומומורפיזמים לינאריים}.
\end{definition}

\begin{definition}[מרחב אופרטורים לינאריים]
יהי
$V$
מרחב וקטורי מעל שדה
$\mbb{F}$.

נגדיר את
\emph{מרחב האופרטורים הלינאריים על
$V$}
בתור
\begin{align*}
\text{.} \End_{\FF}\prs{V} \coloneqq \hom_{\FF}\prs{V,V}
\end{align*}
איבריו נקראים
\emph{אופרטורים לינאריים}
או
\emph{אנדומורפיזמים לינאריים}.
\end{definition}

\begin{exercise}
יהיו
$V,W$
מרחבים וקטוריים סוף־מימדיים מעל שדה
$\mbb{F}$.
יהיו
$B \coloneqq \prs{v_1, \ldots, v_n}$
ו־%
$C \coloneqq \prs{w_1, \ldots, w_m}$
בסיסים של
$V$
ו־%
$W$
בהתאמה, ולכל
$i \in \brs{n}$
ו־%
$j \in \brs{m}$
נגדיר
\begin{align*}
T_{i,j} \colon V &\to W
\end{align*}
על ידי
\begin{align*}
\text{.} T_{i,j}\prs{v_k} = \fcases{w_j & k = i \\ 0 & \text{otherwise}}
\end{align*}
יהי
\begin{align*}
\text{.} D \coloneqq \prs{T_{1,1}, \ldots, T_{1,m}, T_{2,1}, \ldots, T_{2,m}, \ldots, T_{n,1}, \ldots, T_{n,m}}
\end{align*}


\begin{enumerate}
\item
הראו ש־%
$D$
בסיס של
$\hom_{\FF}\prs{V,W}$
והסיקו כי
\[\text{.} \dim_{\mbb{F}}\prs{\hom_{\FF}\prs{V,W}} = \dim_{\mbb{F}}\prs{V} \cdot \dim_{\mbb{F}}\prs{W}\]

\item
נגדיר
\begin{align*}
\rho^B_C \colon \hom_\FF\prs{V,W} &\to \Mat_{m \times n}\prs{\FF} \\
\text{.} \hphantom{lalalalalalalalal} T &\mapsto \brs{T}^B_C
\end{align*}
הראו כי
$\rho^B_C\prs{T_{i,j}} = E_{j,i}$
לכל
$i \in \brs{n}$
ולכל
$j \in \brs{m}$,
כאשר
$E_{i,j} \in \Mat_{m \times n}$
מטריצה המקיימת
\[\text{.}\prs{E_{i,j}}_{k,\ell} = \fcases{1 & i = k \wedge j = \ell \\ 0 & \text{otherwise}}\]
הסיקו כי
$\rho^B_C$
הינו איזומורפיזם.
\end{enumerate}
\end{exercise}

\begin{solution}
\begin{enumerate}
\item אם
$D$
בסיס של
$\hom_\FF\prs{V,W}$,
נקבל כי
\begin{align*}
\text{,} \dim \hom_\FF\prs{V,W} = \abs{D} = n \cdot m = \dim_\FF \prs{V} \cdot \dim_\FF\prs{W}
\end{align*}
כנדרש. לכן, נותר להראות כי
$D$
אכן בסיס.

נרצה להראות כי
$D$
בסיס של
$\hom_{\FF}\prs{V,W}$,
על ידי כך שנראה שכל
$T \in \hom_{\FF}\prs{V,W}$
ניתן לכתיבה בצורה יחידה כצירוף לינארי של איברי
$D$.

נסמן ב־%
$\prs{v, v_i}_B$
את המקדם של
$v_i$
בכתיבה של
$v \in V$
כצירוף לינארי של איברי
$B$,
וב־%
$\prs{w, w_i}_C$
את המקדם של
$w_i$
בכתיבה של
$w \in W$
כצירוף לינארי של איברי
$C$.

כדי לכתוב את
$T$
כצירוף לינארי של איברי
$D$,
נרצה לדעת איך איברי
$D$
פועלים על וקטור כללי ב־%
$V$.
נקבל כי
\begin{align}\label{eq:T_ij}
\begin{split}
T_{i,j}\prs{v} &= T_{i,j}\prs{\sum_{k \in \brs{n}} \prs{v, v_k}_B v_k}
\\&= \sum_{k \in \brs{n}} \prs{v, v_k}_B T_{i,j}\prs{v_k}
\\&= \prs{v, v_i}_B w_j
\end{split}
\end{align}

יהי
$T \in \hom_{\FF}\prs{V,W}$.
אז, לכל
$v \in V$
מתקיים
\begin{align*}
T\prs{v} &= T\prs{\sum_{i \in \brs{n}} \prs{v, v_i}_B v_i}
\\&= \sum_{i \in \brs{n}} \prs{v, v_i}_B T\prs{v_i}
\end{align*}
ולאחר הצבה של
\begin{align*}
T\prs{v_i} = \sum_{j \in \brs{m}} \prs{T\prs{v_i}, w_j}_C w_j
\end{align*}
נקבל כי
\begin{align*}
T\prs{v} &= \sum_{i \in \brs{n}} \prs{v, v_i}_B \sum_{j \in \brs{m}} \prs{T\prs{v_i}, w_j}_C w_j
\\&= \sum_{\substack{i \in \brs{n} \\ j \in \brs{m}}} \prs{v, v_i}_B \prs{T\prs{v_i}, w_j}_C w_j
\\&= \sum_{\substack{i \in \brs{n} \\ j \in \brs{m}}} \prs{T\prs{v_i}, w_j}_C \prs{\prs{v, v_i}_B w_j}
\\ \text{,} \hphantom{T\prs{v}} &= \sum_{\substack{i \in \brs{n} \\ j \in \brs{m}}} \prs{T\prs{v_i}, w_j}_C T_{i,j}\prs{v}
\end{align*}
כאשר בשוויון האחרון השתמשנו ב־%
\eqref{eq:T_ij}.
בסך הכל, נקבל כי
\begin{align*}
\text{,} T &= \sum_{\substack{i \in \brs{n} \\ j \in \brs{m}}} \prs{T\prs{v_i}, w_j}_C T_{i,j}
\end{align*}
ולכן כל איבר ב־%
$\hom_{\FF}\prs{V,W}$
ניתן לכתיבה כצירוף של איברי
$D$.

נותר להראות כי כתיבה זאת יחידה.
נניח כי
$\prs{\alpha_{i,j}}_{\substack{i \in \brs{n} \\ j \in \brs{m}}} \subseteq \FF$
ו־%
$\prs{\beta_{i,j}}_{\substack{i \in \brs{n} \\ j \in \brs{m}}} \subseteq \FF$
מקיימים ש־%
\begin{align*}
\text{.} \sum_{\substack{i \in \brs{n} \\ j \in \brs{m}}} \alpha_{i,j} T_{i,j} = \sum_{\substack{i \in \brs{n} \\ j \in \brs{m}}} \beta_{i,j} T_{i,j}
\end{align*}
בפרט, מתקיים שוויון לאחר הצבה של
$v_k$
בשני האגפים, עבור
$k \in \brs{n}$.
נקבל מהגדרת
$T_{i,j}$
כי
\begin{align*}
\text{.} \forall k \in \brs{n} : \sum_{j \in \brs{m}} \alpha_{k,j} w_j = \sum_{j \in \brs{m}} \beta_{k,j} w_j
\end{align*}
אבל,
$\prs{w_1, \ldots, w_m}$
בסיס של
$W$,
ולכן
$\alpha_{k,j} = \beta_{k,j}$
לכל
$j \in \brs{m}$.
כיוון שזה נכון לכל
$k \in \brs{n}$,
נקבל כי
$\alpha_{i,j} = \beta_{i,j}$
לכל
$i \in \brs{n}$
ולכל
$k \in \brs{m}$,
כנדרש.

\item

יהיו
$i \in \brs{n}$
ו־%
$j \in \brs{m}$.
לפי הגדרת מטריצה מייצגת,
$\brs{T_{i,j}}^B_C$
מטריצה שלכל
$\ell \in \brs{n}$,
העמודה ה־%
$\ell$
שלה היא
$\brs{T_{i,j}\prs{v_\ell}}_C$.
לפי הגדרת
$T_{i,j}$
נקבל כי עמודה זאת שווה לאפס כאשר
$\ell \neq i$
ושווה ל־%
$\brs{w_j}_C$
אחרת.
אך
$\brs{w_j}_C$
וקטור שמקדמיו אפסים פרט ל־%
$1$
במקום ה־%
$j$.
נקבל כי
$\brs{T_{i,j}}^B_C$
מטריצה שמקדמיה שווים לאפס פרט ל־%
$1$
במיקום ה־%
$\prs{j,i}$,
ולכן זאת המטריצה
$E_{j,i}$.

כיוון ש־%
\[\prs{E_{1,1}, \ldots, E_{1,m}, E_{2,1}, \ldots, E_{2,m}, \ldots, E_{n,1}, \ldots, E_{n,m}}\]
בסיס של
$\Mat_{m \times n}\prs{\FF}$,
נקבל כי
$\rho^B_C$
שולחת בסיס של
$\hom_\FF\prs{V,W}$
לבסיס של
$\Mat_{m \times n}\prs{\FF}$,
ולכן הינה איזומורפיזם.
\end{enumerate}
\end{solution}

\printbibliography
\end{document}
