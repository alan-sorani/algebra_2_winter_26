\documentclass[a4paper,10pt,twoside,openany]{article}

\usepackage[lang=hebrew]{maths}
\usepackage{polynom}
\usepackage{hebrewdoc}
\usepackage{stylish}
\usepackage{lipsum}
\let\bs\blacksquare

\setlength{\parindent}{0pt}
\newcommand{\Std}{\mrm{Std}}

%%%%%%%%%%%%
% Styling %
%%%%%%%%%%%%

\usepackage{enumitem}

%%%%%%%%%%%%%
% Counters  %
%%%%%%%%%%%%%

\setcounter{section}{0}     
            
%BIBLIOGRAPHY
\usepackage[
backend=biber,
style=alphabetic,
]{biblatex}
\addbibresource{bibliography.bib} %Imports bibliography file

\title{
אלגברה ב' (01040168) - חורף 2026
\\
תרגול 6 - עוד על ההעתקה הצמודה, משפט ההצגה של ריס, ואיזומטריות
\\
אלן סורני
\\
הרשימות עודכנו לאחרונה בתאריך ה־%
\today
}
\date{}

\begin{document}
\maketitle

\section{ההעתקה הצמודה}

\begin{theorem}[ההעתקה הצמודה]
יהיו
$V,W$
מרחבי מכפלה פנימית סוף־מימדיים ותהי
$T \in \Hom_{\mbb{F}}\prs{V,W}$.
קיימת העתקה יחידה
$T^* \in \Hom_{\mbb{F}}\prs{W,V}$
עבורה
\begin{align*}
\trs{T\prs{v}, w}_W = \trs{v, T^*\prs{w}}_V
\end{align*}
לכל
$v \in V$
ולכל
$w \in W$.

היא נקראת
\emph{ההעתקה הצמודה}
של
$T$.
\end{theorem}

\begin{theorem}
יהיו
$\prs{V, \trs{\cdot, \cdot}_V}, \prs{W, \trs{\cdot, \cdot}_W}$
מרחבי מכפלה פנימית סוף מימדיים עם בסיסים אורתונורמליים
$B,C$
בהתאמה, ותהי
$T \in \Hom_{\mbb{F}}\prs{V,W}$.
אז
$\brs{T^*}^C_B = \prs{\brs{T}^B_C}^*$.
\end{theorem}

\begin{exercise}
יהי
$V = \Mat_n\prs{\mbb{C}}$
עם המכפלה הפנימית
$\trs{A,B} = \tr\prs{B^*, A}$,
ויהי
\begin{align*}
    \Phi \colon V &\to V \\
    \text{.} \hphantom{lala} A &\mapsto A^t
\end{align*}
חשבו את
$\Phi^*$.
\end{exercise}

\begin{solution}
נוכל לחשב ישירות,
\begin{align*}
    \trs{\Phi\prs{A},B} &= \trs{A^t, B} \\&= \tr\prs{B^* A^t} \\&= \overline{\tr\prs{\overline{B^* A^t}}} \\&= \overline{\tr\prs{B^t A^*}} \\&= \overline{\tr\prs{A^* B^t}} \\&= \overline{\trs{B^t, A}} \\&= \trs{A, B^t} \\&= \trs{A, \Phi\prs{B}}
\end{align*}
ולכן
$\Phi^* = \Phi$.

נוכל לחשב זאת גם בעזרת מטריצה מייצגת בבסיס אורתונורמלי.
נסמן ב־%
$E_{i,j}$
מטריצה עבורה
$\prs{E_{i,j}}_{k,\ell} = \delta_{i,k} \delta_{j,\ell}$.
כלומר, זאת מטריצה עם
$0$
בכל המקדמים חוץ מזה בשורה ה־%
$i$
והעמודה ה־%
$j$.
אז בבסיס
\[B \coloneqq \prs{E_{1,1}, \ldots, E_{n,n}, E_{1,2}, E_{2,1}, E_{1,3}, E_{3,1} \ldots, E_{n-1, n} E_{n,n-1}}\]
נקבל כי
\begin{align*}
    \text{.} \brs{\Phi}_B &= I_n \oplus \pmat{0 & 1 \\ 1 & 0} \oplus \ldots \oplus \pmat{0 & 1 \\ 1 & 0}
\end{align*}
כיוון ש־%
$B$
בסיס אורתונורמלי
נקבל כי
$\brs{\Phi^*}_B = \overline{\brs{\Phi}_B}^t$.
אך
$\brs{\Phi}_B$
מטריצה ממשית סימטרית, ולכן
$\brs{\Phi^*}_B = \brs{\Phi}_B$
ואז
$\Phi^* = \Phi$.
\end{solution}

\section{משפט ההצגה של ריס}

\begin{definition}[המרחב הדואלי]
יהי
$V$
מרחב מכפלה פנימית מעל שדה
$\mbb{F}$.
\emph{המרחב הדואלי}
של
$V$
הוא
\[\text{.} V^* \coloneqq \hom_{\mbb{F}}\prs{V,\mbb{F}}\]
איבריו נקראים
\emph{פונקציונלים לינאריים}.
\end{definition}

\begin{theorem}[משפט ההצגה של ריס]
יהי
$V$
מרחב מכפלה פנימית. לכל
$v \in V$
נסמן
\begin{align*}
h_v \colon V &\to V \\
\text{.} \hphantom{lalala} w &\mapsto \trs{w,v}
\end{align*}
אז ההעתקה
\begin{align*}
H^V \colon V &\to V^* \\
v &\mapsto h_v
\end{align*}
הינה איזומורפיזם לינארי.
\end{theorem}

\begin{corollary}[ניסוח קונרקטי למשפט ההצגה של ריס]
יהי
$V$
מרחב מכפלה פנימית סוף־מימדי מעל שדה
$\mbb{F}$,
ויהי
$\phi \in V^*$
פונקציונל לינארי על
$V$.
קיים וקטור
$w \in V$
יחיד עבורו
$\phi\prs{v} = \trs{v,w}$
לכל
$v \in V$.

בנוסף, אם
$B = \prs{v_1, \ldots, v_n}$
בסיס אורתונורמלי של
$V$,
מתקיים
\[\text{.} w = \sum_{i \in \brs{n}} \trs{w, v_i} v_i = \sum_{i \in \brs{n}} \overline{\phi\prs{v_i}} v_i\]
\end{corollary}

\begin{exercise}
יהי
$V = \Mat_n\prs{\mbb{C}}$
ותהי
$\tr \colon V \to \mbb{C}$
העתקת העקבה.
מיצאו מטריצה
$B$
עבורה
$\tr\prs{A} = \trs{A,B}$
לכל
$A \in V$.
\end{exercise}

\begin{solution}
נסתכל על הבסיס האורתונורמלי
$\prs{E_{i,j}}_{i,j \in \brs{n}}$
של
$V$
וניעזר בנוסחא. נקבל
\[\text{.} B = \sum_{i,j \in \brs{n}} \overline{\tr\prs{E_{i,j}}} E_{i,j} = \sum_{i \in \brs{n}} E_{i,i} = I_n\]
\end{solution}

\begin{exercise}
\begin{enumerate}
\item הוכיחו כי לכל
$n \in \mbb{N}$
קיים
$C > 0$
כך שלכל
$p \in \mbb{R}_n\brs{x}$
מתקיים
\[\text{.} \abs{p\prs{0}} \leq C \prs{\int_{-1}^1 p\prs{x}^2 \diff x}^{\frac{1}{2}}\]

\item חשבו את
$C$
המינימלי עבור
$n=2$.
\end{enumerate}
\end{exercise}

\begin{solution}
\begin{enumerate}
\item נשים לב כי
\[\trs{f,g} = \int_{-1}^1 f\prs{x} g\prs{x} \diff x\]
מכפלה פנימית.
אז
\begin{align*}
\text{.} \prs{\int_{-1}^1 p\prs{x}^2 \diff x}^{\frac{1}{2}} = \pmat{\trs{p,p}}^{\frac{1}{2}} = \norm{p}
\end{align*}
כלומר, עלינו להוכיח
\[\text{.} \abs{p\prs{0}} \leq C\norm{p}\]

ההצבה
\begin{align*}
\ev_0 \colon \mbb{R}_n\brs{x} &\to \mbb{R} \\
f &\mapsto f\prs{0}
\end{align*}
היא פונקציונל לינארי, ולכן ממשפט ריס יש
$g \in \mbb{R}_n\brs{x}$
עבורו
\[\text{.} p\prs{0} = \ev_0\prs{p} = \trs{p,g}\]
עכשיו, מקושי־שוורץ
\[\text{.} \abs{p\prs{0}} = \abs{\trs{p,g}} \leq \norm{p} \norm{g}\]
לכן ניקח
$C = \norm{g}$.

\item
נסמן
$g\prs{x} = ax^2 + bx + c$.
נשים לב כי
כאשר
$p = g$
יש שוויון בקושי־שוורץ ואז
\[\text{.} \abs{p\prs{0}} = \norm{p}\norm{g} \leq C\norm{p}\]
גורר
$C \geq \norm{g}$.
ראינו כי
$C = \norm{g}$
מקיים את הנדרש, ולכן נותר למצוא את
$\norm{g}$.

כדי לא להצטרך בסיס אורתונורמלי, שיהיה פחות יפה במקרה הזה, נשים לב שמספיק לחשב את המכפלות הפנימיות בין
$g$
לאיברי בסיס כלשהו.
לפי בסיס אורתונורמלי
$B = \prs{v_1, \ldots, v_n}$
מתקיים
\[\text{.} g = \sum_{i \in \brs{n}} \trs{g, v_i} v_i\]
אם
$C = \prs{u_1, \ldots, u_n}$
בסיס נוסף, נוכל לכתוב
\[v_i = \sum_{j \in \brs{n}} \alpha_{i,j} u_j\]
ואז
\[\text{,} g = \sum_{i \in \brs{n}} \trs{g, v_i} v_i = \sum_{i,j \in \brs{n}} \bar{\alpha}_{i,j} \trs{g, u_j}\]
ונקבל שמהכפלות הפנימיות
$\trs{g, u_i}$
נותנות מספיק אינפורמציה כדי למצוא את
$g$.

כעת, נחשב את המכפלות הפנימיות של
$g$
עם איברי הבסיס הסטנדרטי ונפתור מערכת משוואות שיהיה לה פתרון יחיד שהוא מקדמי
$g$.
\begin{align*}
1 &= 1\prs{0} = \trs{g\prs{x},1} = \int_{-1}^1 g\prs{x} \diff x = \left. \frac{ax^3}{3} + \frac{bx^2}{2} + cx \right|_{x=-1}^1 = \frac{2a}{3} + 2c \\
0 &= x\prs{0} = \trs{g\prs{x}, x} = \int_{-1}^1 g\prs{x} x \diff x = \left. \frac{ax^4}{4} + \frac{bx^3}{3} + \frac{cx^2}{2} \right|_{x=-1}^1 = \frac{2b}{3} \\
\text{.} 0 &= x^2\prs{0} = \trs{g\prs{x}, x^2} = \int_{-1}^1 g\prs{x} x^2 \diff x = \left. \frac{ax^5}{5} + \frac{bx^4}{4} + \frac{cx^3}{3} \right|_{x=-1}^1 = \frac{2a}{5} + \frac{2c}{3}
\end{align*}
מהמשוואה השנייה נקבל
$b = 0$.
מהמשוואה הראשונה פחות
$3$
פעמים השנייה נקבל
\[1 = \frac{2a}{3} - 3 \cdot \frac{2a}{5} + 0 = \frac{10a - 18a}{15}\]
ולכן
$a = -\frac{15}{8}$.
אז מהמשוואה השלישית נקבל
\[c = -\frac{3a}{5} = -\frac{9}{8}\]
ולכן
\[\text{.} g\prs{x} = -\frac{15}{8} x^2 - \frac{9}{8}\]

אז
\[\text{.} C = \norm{g} = \sqrt{\int_{-1}^1 \prs{-\frac{15}{8} x^2 - \frac{9}{8}}^2} \diff x = \frac{27}{4}\]
\end{enumerate}
\end{solution}

\section{איזומטריות ואופרטורים אוניטריים ואורתוגונליים}

\subsection{איזומטריות}

תכונה אינטואיטיבית גיאומטרית של אופרטורים על מרחבי מכפלה פנימית היא שמירה של המכפלה הפנימית.

\begin{definition}[איזומטריה בין מרחבי מכפלה פנימית]
יהיו
$V,W$
מרחבי מכפלה פנימית מעל שדה
$\mbb{F}$,
ותהי
$T \in \Hom_\FF\prs{V,W}$.
נקרא ל־%
$T$
\emph{איזומטריה}
אם
\begin{align*}
\trs{T\prs{v_1}, T\prs{v_2}}_W = \trs{v_1, v_2}_V
\end{align*}
לכל
$v_1, v_2$.
\end{definition}

\begin{proposition}[טענה מההרצאה]
$T$
הינה איזומטריה אם ורק אם
$T$
הפיכה ומתקיים
$T^* = T^{-1}$.
\end{proposition}

\begin{exercise}
יהיו
$V,W$
מרחבי מכפלה פנימית מעל שדה
$\mbb{F}$
ותהי
$T \in \Hom_\FF\prs{V,W}$
איזומטריה.
יהי
$B = \prs{v_1, \ldots, v_n}$
בסיס אורתונורמלי של
$V$.
אז
$T\prs{B} \coloneqq \trs{T\prs{v_1}, \ldots, T\prs{v_n}}$
הינו בסיס אורתונורמלי של
$W$.
\end{exercise}

\begin{solution}
לכל
$i,j \in \brs{n}$
מתקיים
\begin{align*}
\trs{T\prs{v_i}, T\prs{v_j}} &= \trs{v_i, v_j} = \delta_{i,j}
\end{align*}
ולכן הבסיס
$\prs{T\prs{v_1}, \ldots, T\prs{v_n}}$
גם הוא אורתונורמלי.
\end{solution}

\begin{definition}[מרחבי מכפלה פנימית איזומטריים]
יהיו
$V,W$
מרחבי מכפלה פנימית מעל שדה
$\mbb{F}$.
נגיד כי
$V,W$
\emph{איזומטריים}
אם קיימת איזומטריה
$T \in \Hom_\FF\prs{V,W}$.
\end{definition}

\begin{exercise}
יהיו
$V,W$
מרחבי מכפלה פנימית מעל שדה
$\FF$.

הראו כי
$V,W$
איזומטריים אם ורק אם
$\dim_\FF\prs{V} = \dim_\FF\prs{W}$.
\end{exercise}

\begin{solution}
נניח כי
$V,W$
איזומטריים, ולכן קיימת איזומטריה
$T \in \Hom_\FF\prs{V,W}$.
אך מהטענה, איזומטריה היא בפרט הפיכה, ולכן
$V,W$
איזומורפיים ולכן בעלי אותו מימד.

מצד שני, נניח כי
$\dim_\FF\prs{V} = \dim_\FF\prs{W}$.
לפי גרם־שמידט, קיימים בסיסים אורתונורמליים
$B,C$
של
$V,W$
בהתאמה. נסמן
\begin{align*}
B &= \prs{v_1, \ldots, v_n} \\
\text{,} C &= \prs{w_1, \ldots, w_n}
\end{align*}
ונגדיר
$T \in \Hom_\FF\prs{V,W}$
על ידי
\begin{align*}
\text{.} T\prs{v_i} = w_i
\end{align*}
אז
$T$
שולחת בסיס אורתונורמלי
$B$
לבסיס אורתונורמלי
$C$,
ולפי התרגיל הקודם נקבל כי היא איזומטריה.
\end{solution}

\subsection{אופרטורים אורתוגונליים ואוניטריים}

\begin{definition}[אופרטור אורתוגונלי (אוניטרי)]
אופרטור
$T \in \End_{\mbb{F}}\prs{V}$
מעל
$\mbb{R}$
(מעל
$\mbb{C}$
נקרא
\emph{אורתוגונלי (אוניטרי)}
אם
$T^* = T^{-1}$.
\end{definition}

\begin{definition}[מטריצה אורתונוגלית (אוניטרית)]
מטריצה
$A \in \Mat_n\prs{\mbb{F}}$
עבור
$\mbb{F} = \mbb{R}$
(עבור
$\mbb{F} = \mbb{C}$)
נקראת
\emph{אורתוגונלית (אוניטרית)}
אם
$A^* \coloneqq \overline{A^t} = A^{-1}$.
\end{definition}

\begin{exercise}
הראו כי
$T \in \End_{\mbb{F}}\brs{V}$
אורתוגונלי (אוניטרי) אם ורק אם
$\norm{Tv} = \norm{v}$
לכל
$v \in V$.

היעזרו בזהות הפולריזציה. מעל
$\mbb{R}$
מתקיים
\begin{align*}
\trs{x,y} = \frac{1}{4}\prs{\norm{x+y}^2 - \norm{x-y}^2}
\end{align*}
ומעל
$\mbb{C}$
מתקיים
\begin{align*}
\text{.} \trs{x,y} = \frac{1}{4}\prs{\norm{x+y}^2 - \norm{x-y}^2 + i \norm{x+iy}^2 -i \norm{x-iy}^2}
\end{align*}
\end{exercise}

\begin{solution}
אם
$T \in \End_{\mbb{F}}\prs{V}$
אופרטור אורתוגונלי (אוניטרי), מתקיים
\[\norm{Tv} = \sqrt{\trs{Tv, Tv}} = \sqrt{\trs{v,v}} = \norm{v}\]
לכל
$v \in V$.

להיפך, אם
$\norm{Tv} = \norm{v}$
לכל
$v \in V$,
נקבל מזהות הפולריזציה כי
$\trs{Tv, Tw} = \trs{v,w}$
לכל
$v,w \in W$.
\end{solution}

\begin{remark}
קיבלנו כי אופרטורים אורתוגונליים (או אוניטריים) הם בדיוק אלו השומרים על הנורמה, ולכן הם גם אלו השומרים על מרחקים.

העתקות השומרות על מרחקים נקראות
\emph{איזומטריות}
ולכן אופרטור אורתוגונלי נקרא גם
\emph{איזומטריה לינארית}.
כיוון שבקורס נדבר רק על העתקות לינאריות, נתייחס לפעמים לאופרטור לינארי בתור
\emph{איזומטריה}.
\end{remark}

\newpage

\begin{exercise}
יהי
$V$
מרחב מכפלה פנימית ממשי ממימד
$n$,
ויהי
$T \in \End_{\mbb{R}}\prs{V}$.
התנאים הבאים שקולים.

\begin{enumerate}
    \item $T$ אורתוגונלי.

    \item קיים בסיס אורתונורמלי
    $B = \prs{v_1, \ldots, v_n}$,
    כך שהבסיס
    $\prs{T v_1, \ldots, Tv_n}$
    אורתונורמלי.
    
    \item לכל בסיס אורתונורמלי
    $B = \prs{v_1, \ldots, v_n}$,
    הבסיס
    $\prs{T v_1, \ldots, Tv_n}$
    אורתונורמלי.
\end{enumerate}
\end{exercise}

\begin{solution}
\begin{description}
\item[$1 \implies 3$:]
ראינו בתרגיל קודם כי איזומטריה שולחת כל בסיס אורתונורמלי לבסיס אורתונורמלי.

\item[$3 \implies 2$:]
מיידי.

\item[$2 \implies 1$:]
יהי
$B = \prs{v_1, \ldots, v_n}$
בסיס אורתונורמלי של
$V$
עבורו
\[T\prs{B} \coloneqq \prs{T\prs{v_1}, \ldots, T\prs{v_n}}\]
אורתונורמלי.
יהיו
$u,v \in V$,
ונראה כי
$\trs{T\prs{u}, T\prs{v}} = \trs{u,v}$,
מה שיראה את הנדרש.
קיימים סקלרים
$\alpha_i, \beta_i \in \FF$
עבור
$i \in \brs{n}$
וכך שמתקיים
\begin{align*}
u &= \sum_{i \in \brs{n}} \alpha_i v_i \\
\text{.} v &= \sum_{i \in \brs{n}} \beta_i v_i
\end{align*}
אז
\begin{align*}
\trs{T\prs{u}, T\prs{v}} &= \trs{T\prs{\sum_{i \in \brs{n}} \alpha_i v_i}, T\prs{\sum_{i \in \brs{n}} \beta_i v_i}}
\\&=
\trs{\sum_{i \in \brs{n}} \alpha_i T\prs{v_i}, \sum_{i \in \brs{n}} \beta_i T\prs{v_i}}
\\&=
\sum_{i \in \brs{n}} \sum_{j \in \brs{n}} \alpha_i \overline{\beta_j} \trs{T\prs{v_i}, T\prs{v_j}}
\\&=
\sum_{i \in \brs{n}} \sum_{j \in \brs{n}} \alpha_i \overline{\beta_j} \delta_{i,j}
\\&=
\sum_{i \in \brs{n}} \sum_{j \in \brs{n}} \alpha_i \overline{\beta_j} \trs{v_i, v_j}
\\&=
\trs{\sum_{i \in \brs{n}} \alpha_i v_i, \sum_{i \in \brs{n}} \beta_i v_i}
\\ \text{,} \hphantom{\trs{T\prs{u}, T\prs{v}}} &=
\trs{u,v}
\end{align*}
כנדרש, וכאשר בשוויון השלישי והשישי השתמשנו בלינאריות של המכפלה הפנימית ברכיב הראשון, ובאנטי־לינאריות ברכיב השני.
\end{description}
\end{solution}

\begin{exercise}
יהי
$V = \mbb{R}^2$
ויהי
\begin{align*}
    R \colon V &\to V \\
    \pmat{x \\ y} &\mapsto \pmat{x \\ -y}
\end{align*}
שיקוף דרך ציר ה־%
$x$.

הראו כי
$R$
איזומטריה.
\end{exercise}

\begin{solution}
מתקיים
$\brs{R}_E = \pmat{1 & 0 \\ 0 & -1}$
וזאת מטריצה שעמודותיה מהוות בסיס אורתונורמלי של
$V$.
לכן
$R$
שולח בסיס אורתונורמלי (הסטנדרטי) לבסיס אורתונורמלי, ולכן הינו איזומטריה.
\end{solution}

\begin{exercise}
נזכיר שראינו בהרצאה כי
$\rho_\theta \coloneqq A_\theta$
עבור
\[A_\theta = \pmat{\cos \theta & - \sin \theta \\ \sin \theta & \cos\theta}\]
הינו איזומטריה.

הראו כי כל איזומטריה של
$\mbb{R}^2$
היא מהצורה
$\rho_\theta$
או
$\rho_\theta R$
עבור
$\theta \in \mbb{R}$.
\end{exercise}

\begin{solution}
תהי
$T \in \End_{\mbb{R}}\prs{\mbb{R}^2}$
איזומטריה.
מהתנאים השקולים, הבסיס
$\prs{T\prs{e_1}, T\prs{e_2}}$
הינו אורתונורמלי.
בפרט,
$v \coloneqq T\prs{e_1}$
הינו מנורמה
$1$.

נראה שניתן לכתוב
$v = \pmat{\cos\theta \\ \sin\theta}$.
נכתוב
$v = \pmat{v_1, v_2}$
ונקבל כי
$v_1^2 + v_2^2 = 1$.
בפרט,
$v_1 \in \brs{-1,1}$
ולכן יש
$\theta \in \mbb{R}$
עבורה
$v_1 = \cos\theta$.
נקבל כי
$v_2^2 = 1 - \prs{\cos \theta}^2 = \prs{\sin \theta}^2$
ולכן
$v_2 \in \set{\pm \sin \theta}$.
אם
$v_2 = \sin\theta$,
סיימנו.
אחרת, ניתן לכתוב
\begin{align*}
    v_1 &= \cos\prs{-\theta} \\
    v_2 &= \sin\prs{-\theta}
\end{align*}
ואז הזווית המתאימה היא
$-\theta$.

נקבל כי
$v = \rho_\theta\prs{e_1}$
ולכן
\[\text{.} \rho_{-\theta}\prs{v} = \rho_{-\theta} \circ T \prs{e_1} = e_1\]
הרכבת איזומטריות הינה איזומטריה, ולכן נקבל כי
$\rho_{-\theta} \circ T$
היא איזומטריה שמקבעת את
$e_1$.
מהתנאים השקולים, היא מעבירה בסיס אורתונורמלי לבסיס אורתונורמלי, לכן
$\prs{e_1, u} \coloneqq \prs{e_1, \rho_{-\theta} \circ T\prs{e_2}}$
אורתונורמלי.
אבל אז
$u \in \set{\pm e_2}$
כי
$u \in \set{e_1}^\perp = \Span\prs{e_2}$
מנורמה
$1$.
אם
$u = e_2$
נקבל כי
$\rho_{-\theta} \circ T = \id_{\mbb{R}^2}$
כלומר
$T = \rho_{\theta}$.
אחרת,
$\rho_{-\theta} \circ T = R$
ואז
$T = \rho_{\theta} \circ R$.
\end{solution}

\printbibliography
\end{document}
