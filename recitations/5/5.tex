\documentclass[a4paper,10pt,twoside,openany]{article}

\usepackage[lang=hebrew]{maths}
\usepackage{polynom}
\usepackage{hebrewdoc}
\usepackage{stylish}
\usepackage{lipsum}
\let\bs\blacksquare

\setlength{\parindent}{0pt}
\newcommand{\Std}{\mrm{Std}}

%%%%%%%%%%%%
% Styling %
%%%%%%%%%%%%

\usepackage{enumitem}

%%%%%%%%%%%%%
% Counters  %
%%%%%%%%%%%%%

\setcounter{section}{0}     
            
%BIBLIOGRAPHY
\usepackage[
backend=biber,
style=alphabetic,
]{biblatex}
\addbibresource{bibliography.bib} %Imports bibliography file

\title{
אלגברה ב' (01040168) - חורף 2026
\\
תרגול 5 - עוד הטלות, גרם שמידט, וההעתקה הצמודה
\\
אלן סורני
\\
הרשימות עודכנו לאחרונה בתאריך ה־%
\today
}
\date{}

\begin{document}
\maketitle

\section{הטלות אורתוגונליות וגרם שמידט}

\begin{definition}[הטלה אורתוגונלית]
יהי
$V$
מרחב מכפלה פנימית ויהי
$W \leq V$.
\emph{ההטלה האורתוגונלית על
$W$}
היא ההטלה על
$W$
ביחס לסכום הישר
$V = W \oplus W^\perp$.
\end{definition}

\begin{proposition}
יהי
$V$
מרחב מכפלה פנימית ויהי
$W \leq V$.
יהי
$B = \prs{w_1, \ldots, w_m}$
בסיס
\textbf{אורתונורמלי}
של
$W$
ויהי
$v \in V$.
תהי
$P_W$
ההטלה האורתוגונלית על
$W$.
אז
\[\text{.} P_W\prs{w} = \sum_{i \in [m]} \trs{v, w_i} w_i\]
\end{proposition}

\begin{exercise}
יהי
$V$
מרחב מכפלה פנימית ויהיו
$u,v \in V$.
הראו כי
$u \perp v$
אם ורק אם
\[\norm{u} \leq \norm{u+av}\]
לכל
$a \in \mbb{F}$.
\end{exercise}

\begin{solution}
אם
$u \perp v$
ו־%
$a \in \mbb{F}$,
נקבל מפיתגורס
\begin{align*}
\norm{u+av}^2 &= \norm{u}^2 + \abs{a}^2 \norm{v}^2 \geq \norm{u}^2
\end{align*}
ולכן
$\norm{u} \leq \norm{u+av}$.

נניח כי
$\trs{u,v} \neq 0$
ונניח תחילה כי
$\norm{v} = 1$.
אז
\[\trs{\trs{u,v}v, v} = \trs{u,v} \trs{v,v} = \trs{u,v} \norm{v}^2 = \trs{u,v} \neq 0\]
וגם
\[\text{.} \trs{u - \trs{u,v}v, v} = \trs{u,v} - \trs{\trs{u,v}v,v} = 0\]
אז, ממשפט פיתגורס
\begin{align*}
\norm{u}^2 &= \norm{u - \trs{u,v}v + \trs{u,v}v}^2
\\&=
\norm{u - \trs{u,v}v}^2 + \norm{\trs{u,v}v}^2
\\&>
\norm{u-\trs{u,v}v}^2
\end{align*}
כאשר האי־שוויון חזק כי
$\trs{u,v}v \neq 0$
מההנחה
$\trs{u,v} \neq 0$.
לכן, עבור
$a = \trs{u,v}$
לא מתקיים
$\norm{u} \leq \norm{u+av}$.

כעת, אם
$v$
כללי, הוקטור
$\frac{v}{\norm{v}}$
הינו מאורך
$1$
ולכן יש
$a' \in \mbb{F}$
עבורו
$\norm{u} > \norm{u+a' \frac{v}{\norm{v}}}$.
אז ניקח
$a = \frac{a'}{\norm{v}}$
ונקבל כי
$\norm{u} > \norm{u + av}$.
\end{solution}

כדי למצוא בסיסים אורתונורמליים ומשלימים ישרים, ניעזר בתהליך שלוקח בסיס כלשהו ומחזיר בסיס אורתונורמלי.

\begin{theorem}[גרם־שמידט]
יהי
$V$
מרחב מכפלה פנימית ויהי
$B = \prs{u_1, \ldots, u_n}$
בסיס של
$V$.
קיים בסיס אורתונורמלי
$C = \prs{v_1, \ldots, v_n}$
של
$V$
עבורו
\[\Span\prs{u_1, \ldots, u_i} = \Span\prs{v_1, \ldots, v_i}\]
לכל
$i \in \brs{n}$.
\end{theorem}

ניסוח זה של המשפט לא מתאר לנו איך למצוא את
$C$,
אבל ההוכחה שלו קונסטרוקטיבית ומתארת את האלגוריתם הבא.

\begin{enumerate}
\item
עבור
$i = 1$
ניקח
$v_i = \frac{u_i}{\norm{u_i}}$.

\item
עבור כל
$i$
לאחר מכן, לפי הסדר, ניקח
\[w_i = u_i - \sum_{j \in \brs{i-1}} \trs{u_i, v_j}\]
ואז
$v_i = \frac{w_i}{\norm{w_i}}$.
\end{enumerate}

\begin{corollary}
יהי
$W \leq V$
תת־מרחב במרחב מכפלה פנימית.
כדי למצוא אורתונורמלי של
$W$
ושל
$W^\perp$
ניקח בסיס
$B_W$
של
$W$
ונשלים אותו לבסיס
$B = B_W \cup B'_W$
של
$V$.
נבצע את תהליך גרם־שמידט על
$B$
ונקבל בסיס
$C = C_W \cup C_W^\perp$
כך ש־%
$\Span\prs{C_W} = \Span\prs{B_W} = W$.
כל הוקטורים ב־%
$C_W^\perp$
ניצבים ל־%
$W$
כי
$C$
אורתונורמלי, ולכן
$W' \coloneqq \Span\prs{C_W^\perp} \subseteq W^\perp$.
כמו כן,
\[\dim\prs{W'} = \dim\prs{V} - \dim\prs{W} = \dim\prs{W^\perp}\]
כי
$V = W \oplus W' = W \oplus W^\perp$,
ולכן יש שוויון
$W' = W^\perp$.
\end{corollary}

\begin{theorem}[מרחק של וקטור מתת־מרחב]
יהי
$V$
מרחב מכפלה פנימית, יהי
$W \leq V$,
תהי
$P_W$
ההטלה האורתוגונלית על
$W$
ויהי
$v \in V$.
מתקיים
\[\text{.} d\prs{v,W} = d\prs{v, p_W\prs{v}}\]
\end{theorem}

\begin{exercise}
יהי
$V = \Mat_2\prs{\mbb{R}}$
עם המכפלה הפנימית הסטנדרטית
\[\trs{A,B} = \tr\prs{B^t A}\]
ויהי
$W \leq V$
התת־מרחב של המטריצות הסימטריות.

\begin{enumerate}
\item מיצאו בסיס אורתונורמלי עבור
$W$
ועבור
$W^\perp$.

\item
הראו כי
$P_W\prs{A} = \frac{A + A^t}{2}$
בשתי דרכים שונות.

\item
חשבו את המרחק של
$A = \pmat{1 & 2 \\ 3 & 4}$
מ־%
$W$.
\end{enumerate}
\end{exercise}

\begin{solution}
\begin{enumerate}
\item
ניקח בסיס
$B_W = \prs{E_{1,1}, E_{1,2} + E_{2,1}, E_{2,2}}$
של
$W$
ונשלים אותו לבסיס
\[B = \prs{u_1, u_2, u_3, u_4} = \prs{E_{1,1}, E_{1,2} + E_{2,1}, E_{2,2}, E_{1,2}}\]
של
$V$.
נבצע את תהליך גרם־שמידט כדי לקבל בסיס אורתונורמלי
$\prs{v_1, v_2, v_3, v_4}$
עבורו
$W = \Span\prs{v_1, v_2, v_3}$
וגם
$W^\perp = \Span\prs{v_4}$.

נחשב
\begin{align*}
v_1 &= \frac{1}{\norm{E_{1,1}}} E_{1,1} = E_{1,1} \\
w_2 &= u_2 - \trs{u_2,v_1} v_1 = E_{1,2} + E_{2,1} - 0 = E_{1,2} + E_{2,1} \\
v_2 &= \frac{1}{\norm{w_2}} w_2 = \frac{1}{\sqrt{2}} \prs{E_{1,2} + E_{2,1}} \\
w_3 &= u_3 - \trs{u_3, v_2} v_2 - \trs{u_3, v_1} v_1 = u_3 = E_{2,2} \\
v_3 &= \frac{1}{\norm{w_3}} w_3 = E_{2,2} \\
w_4 &= u_4 - \trs{u_4, v_3}v_3 - \trs{u_4, v_2}v_2 - \trs{u_4, v_1}v_1 \\&= E_{1,2} - \trs{E_{1,2}, \frac{1}{\sqrt{2}} \prs{E_{1,2} + E_{2,1}}} \prs{\frac{1}{\sqrt{2}} \prs{E_{1,2} + E_{2,1}}}
\\&= E_{1,2} - \frac{1}{2} \prs{E_{1,2} + E_{2,1}}
\\&= \frac{1}{2} \prs{E_{1,2} - E_{2,1}} \\
v_4 &= \frac{w_4}{\norm{w_4}} = \frac{1}{\norm{E_{1,2} - E_{2,1}}} \prs{E_{1,2} - E_{2,1}} = \frac{1}{\sqrt{2}} \prs{E_{1,2} - E_{2,1}}
\end{align*}
ונקבל כי
\[\prs{v_1, v_2, v_3} = \prs{E_{1,1}, \frac{1}{\sqrt{2}} \prs{E_{1,2} + E_{2,1}}, E_{2,2}}\]
בסיס אורתונורמלי של
$W$
וכי
$\frac{1}{\sqrt{2}} \prs{E_{1,2} - E_{2,1}}$
בסיס אורתונורמלי של
$W^\perp$.
אז,
$W^\perp$
מרחב המטריצות האנטיסימטריות.

\item
בדרך אחת, זכור לנו מאלגברה א' כי
\[A = \frac{A + A^t}{2} + \frac{A - A^t}{2}\]
כאשר
$\frac{A + A^t}{2}$
סימטרית ו־%
$\frac{A - A^t}{2}$
אנטי־סימטרית.
אז
$\frac{A + A^t}{2} \in W$
וכיוון ש־%
$W^\perp$
מרחב המטריצות האנטיסימטריות, נקבל גם
$\frac{A - A^t}{2} \in W^\perp$.
לכן
$P_W\prs{A} = \frac{A + A^t}{2}$.

אבל, במקרה הכללי, לא נדע מראש איך לכתוב וקטור
$v \in V$
בתור סכום של וקטור ב־%
$W \leq V$
ווקטור ב־%
$W^\perp$.
כדי לחשב את ההטלה
$P_W\prs{A}$
נוכל לקחת בסיס אורתונורמלי של
$W$
ולהשתמש בנוסחא עבור ההטלה האורתוגונלית לפי בסיס כזה.

אכן,
\begin{align*}
P_W\prs{A} &= \sum_{i \in [3]} \trs{A, v_i} v_i \\&= \trs{A, E_{1,1}} E_{1,1} + \trs{A, \frac{1}{\sqrt{2}} \prs{E_{1,2} + E_{2,1}}} \cdot \frac{1}{\sqrt{2}} \prs{E_{1,2} + E_{2,1}} + \trs{A, E_{2,2}}, E_{2,2}
\\&=
a_{1,1} E_{1,1} + \frac{1}{2} \prs{a_{1,2} + a_{2,1}} \prs{E_{1,2} + E_{2,1}} + a_{2,2} E_{2,2}
\\&=
\pmat{a_{1,1} & \frac{a_{1,2} + a_{2,1}}{2} \\ \frac{a_{1,2} + a_{2,1}}{2} & a_{2,2}}
\\ \text{.} \hphantom{P_W\prs{A}} &=
\frac{A + A^t}{2}
\end{align*}

\item
נכתוב את
$A$
בתור סכום של מטריצה סימטרית ואנטי־סימטרית. ראינו כי
\[A = \frac{A + A^t}{2} + \frac{A - A^t}{2}\]
כאשר
$\frac{A + A^t}{2}$
סימטרית ו־%
$\frac{A - A^t}{2}$
אנטי־סימטרית.
אז
$d_W\prs{A} = \frac{A+A^t}{2}$
ונקבל כי
המרחק של
$A$
מ־%
$W$
הוא
\[\text{.} d\prs{A, \frac{A + A^t}{2}} = \norm{A - \frac{A + A^t}{2}}\]
מתקיים
\begin{align*}
\norm{A - \frac{A + A^t}{2}} &= \norm{\frac{A - A^t}{2}}
\\&= \frac{1}{2} \norm{\pmat{0 & -1 \\ & 1 & 0}}
\\&= \frac{1}{2} \cdot \sqrt{2}
\\&= \frac{1}{\sqrt{2}}
\end{align*}
ולכן
$d\prs{A, W} = \frac{1}{\sqrt{2}}$.
\end{enumerate}
\end{solution}

\section{עוד מכפלות פנימיות}

\begin{definition}[מכפלה פנימית לפי בסיס]
יהי
$V$
מרחב וקטורי ממימד
$n \in \NN_+$
מעל
$\FF \in \set{\RR, \CC}$
ויהי
$B$
בסיס של
$V$.

נגדיר מכפלה פנימית
$\trs{\cdot, \cdot}_B$
לפי
\begin{align*}
\text{.} \brs{u,v}_B \coloneqq \trs{\brs{u}_B, \brs{v}_B}_{\Std}
\end{align*}
כאשר
$\trs{\cdot, \cdot}_{\Std}$
המכפלה הפנימית הסטנדרטית על
$\FF^n$.
\end{definition}

\begin{exercise}
יהי
$V = M_2\prs{\mbb{R}}$
עם הבסיס
\[\text{.} B = \prs{\pmat{1 & 1 \\ 0 & 0}, \pmat{1 & 2 \\ 0 & 0}, \pmat{1 & 0 \\ 1 & 0}, \pmat{0 & 1 \\ 0 & 1}}\]
מיצאו מכפלה פנימית על
$V$
לפיה
$B$
בסיס אורתונורמלי.
\end{exercise}

\begin{solution}
ניזכר כי כל מכפלה פנימית על
$V$
היא מהצורה
$\trs{\cdot, \cdot}_C$
עבור בסיס
$C$
של
$V$.
לפי ההגדרה, נקבל כי אם
\begin{align*}
B = \prs{v_1, v_2, v_3, v_4}
\end{align*}
אז
\begin{align*}
\text{,} \trs{v_i, v_j}_B = \brs{\brs{v_i}_B, \brs{v_j}_B}_{\Std} = \trs{e_i, e_j}_{\Std} = \delta_{i,j}
\end{align*}
ולכן
$B$
אורתונורמלי לפי המכפלה הפנימית
$\trs{\cdot, \cdot}_B$.
נרצה אם כן לחשב את
$\trs{\cdot, \cdot}_B$.

יהי
$v = \pmat{a & b \\ c & d} \in V$.
נחפש
$\alpha_i \in \RR$
עבורם
$v = \sum_{i \in \brs{4}} \alpha_i$,
כלומר כאלו עבורם
\begin{align*}
\text{.} \pmat{a & b \\ c & d} &= \pmat{\alpha_1 + \alpha_2 + \alpha_3 & \alpha_1 + 2 \alpha_2 + \alpha_4 \\ \alpha_3 & \alpha_4}
\end{align*}
נקבל מערכת משוואות בה
$\alpha_3 = c, \alpha_4 = d$
וכן
\begin{align*}
a &= \alpha_1 + \alpha_2 + c \\
\text{.} b &= \alpha_1 + 2 \alpha_2 + d
\end{align*}
נקבל מהשורה הראשונה כי
\begin{align*}
\alpha_2 = a - \alpha_1 - c
\end{align*}
ואז מהשורה השנייה כי
\begin{align*}
b = 2a - 2c - \alpha_1 + d 
\end{align*}
ולכן
\begin{align*}
\text{.} \alpha_1 = 2a - b -2c + d
\end{align*}
נציב זאת במשוואה עבור
$\alpha_2$
ונקבל כי
\begin{align*}
\text{.} \alpha_2 = -a + b + c - d
\end{align*}
בסך הכל, נקבל כי
\begin{align*}
\text{,} \pmat{a & b \\ c & d} = \prs{2a - b -2c + d} v_1 + \prs{-a + b + c - d} v_2 + c v_3 + d v_4
\end{align*}
ובאותו האופן נקבל כי
\begin{align*}
\text{.} \pmat{\alpha & \beta \\ \gamma & \delta} &= \prs{2 \alpha - \beta - 2 \gamma + \delta} v_1 + \prs{- \alpha + \beta + \gamma - \delta} v_2 + \gamma v_3 + \delta v_4
\end{align*}
לכן
\begin{align*}
\trs{\pmat{a & b \\ c & d}, \pmat{\alpha & \beta \\ \gamma & \delta}}_B &=
\trs{\pmat{2a - b -2c + d \\ -a + b + c - d \\ c \\ d}, \pmat{2 \alpha - \beta - 2 \gamma + \delta \\ - \alpha + \beta + \gamma - \delta \\ \gamma \\ \delta}}_{\Std}
\\&=
\prs{2a - b -2c + d}\prs{2 \alpha - \beta -2 \gamma + \delta} + \prs{-a + b + c - d}\prs{-\alpha + \beta + \gamma - \delta} + c \gamma + d \delta
\end{align*}
כנדרש.
\end{solution}

\section{ההעתקה הצמודה}

\begin{theorem}[ההעתקה הצמודה]
יהיו
$V,W$
מרחבי מכפלה פנימית סוף־מימדיים ותהי
$T \in \Hom_{\mbb{F}}\prs{V,W}$.
קיימת העתקה יחידה
$T^* \in \Hom_{\mbb{F}}\prs{W,V}$
עבורה
\begin{align*}
\trs{T\prs{v}, w}_W = \trs{v, T^*\prs{w}}_V
\end{align*}
לכל
$v \in V$
ולכל
$w \in W$.

היא נקראת
\emph{ההעתקה הצמודה}
של
$T$.
\end{theorem}

\begin{theorem}
יהיו
$\prs{V, \trs{\cdot, \cdot}_V}, \prs{W, \trs{\cdot, \cdot}_W}$
מרחבי מכפלה פנימית סוף מימדיים עם בסיסים אורתונורמליים
$B,C$
בהתאמה, ותהי
$T \in \Hom_{\mbb{F}}\prs{V,W}$.
אז
$\brs{T^*}^C_B = \prs{\brs{T}^B_C}^*$.
\end{theorem}

כעת, אם
$T \in \End_{\mbb{F}}\prs{V}$,
יש לנו דרכים שונות לחשב את
$T^*$.
נוכל לנחש איזה אופרטור יקיים
$\trs{Tv, w} = \trs{v, T^* w}$
לכל
$v,w \in V$,
על ידי מניפולציות של הצגת המכפלה הפנימית.
נוכל גם לכתוב את
$T^*$
בצורה כללית ולפתור מערכת משוואות.
לבסוף, נוכל לקחת בסיס אורתונורמלי
$B$,
כדי שיתקיים
$\brs{T^*}_B = \overline{\brs{T}_B}^t$,
ואז לשחזר את
$T^*$
מהמטריצה המייצגת.
נראה דוגמאות לחישוב בכל אחת מהדרכים.

\begin{exercise}
יהי
$V = \mbb{R}_2\brs{x}$
עם המכפלה הפנימית
\[\text{.} \trs{f,g} = \int_{-1}^1 f\prs{x} \bar{g}\prs{x} \diff x\]
יהי
$D \in \End_{\mbb{R}}\prs{V}$
אופרטור הגזירה.
מיצאו את
$D^*$.
\end{exercise}

\begin{solution}
כדי להראות
$D^* = S$
די להראות כי
$\trs{Df,g} = \trs{f,Dg}$
עבור
$f,g$
בבסיס נתון, מלינאריות.
נחשב מה צריך להיות
$D^*$
לפי הבסיס
$\prs{1,x,x^2}$.
יהי
$g\prs{x} = ax^2 + bx + c$.
נחשב את מכפלת
$D\prs{g}$
עם
$1$
\begin{align*}
    0 &= \trs{0,g} \\&= \trs{D\prs{1}, g} \\&= \trs{1, D^*\prs{g}} \\&= \int_{-1}^1 D^*\prs{g}\prs{x} \diff x
\end{align*}
עם
$x$
\begin{align*}
        \frac{2a}{3} + 2c &= \left. \frac{ax^3}{3} + \frac{bx^2}{2} + cx \right|_{-1}^1 \\&= \int_{-1}^1 ax^2 + bx + c \diff x \\&= \trs{1,g} \\&= \trs{D\prs{x}, g} \\&= \trs{x, D^*\prs{g}} \\&= \int_{-1}^1 x D^*\prs{g}\prs{x} \diff x
\end{align*}
ועם
$x^2$
\begin{align*}
        \frac{4b}{3} &= \left. \frac{2ax^4}{4} + \frac{2bx^3}{3} + \frac{2cx^2}{2} \right|_{-1}^1 \\&= 2 \int_{-1}^1 ax^3 + bx^2 + cx \diff x \\&= \trs{2x,g} \\&= \trs{D\prs{x^2}, g} \\&= \trs{x^2, D^*\prs{g}} \\&= \int_{-1}^1 x^2 D^*\prs{g}\prs{x} \diff x
\end{align*}

נכתוב
$D^*\prs{g} = \alpha x^2 + \beta x + \gamma$
ונקבל
\begin{align*}
    0 &= \int_{-1}^1 \alpha x^2 + \beta x + \gamma \diff x = \frac{2 \alpha}{3} + 2 \gamma \\
    \frac{2a}{3} + c &= \int_{-1}^1 \alpha x^3 + \beta x^2 + \gamma x \diff x = \frac{2\beta}{3} \\
    \frac{4b}{3} &= \int_{-1}^1 \alpha x^4 + \beta x^3 + \gamma x^2 \diff x = \left. \frac{\alpha x^5}{5} + \frac{\beta x^4}{4} + \frac{\gamma x^3}{3} \right|_{-1}^1 = \frac{2 \alpha}{5} + \frac{2 \gamma}{3}
\end{align*}
מהמשוואה הראשונה,
$\gamma = -\frac{\alpha}{3}$.
אז מהמשוואה השלישית,
$\frac{4b}{3} = \frac{2 \alpha}{5} - \frac{2 \alpha}{9}$
כלומר
\[4b = \prs{\frac{6}{5} - \frac{2}{3}}\alpha = \prs{\frac{18}{15} - \frac{10}{15}}\alpha = \frac{8 \alpha}{15}\]
כלומר
$b = \frac{2 \alpha}{15}$
כלומר
$\alpha = \frac{15}{2} b$.
נקבל כי
$\gamma = -\frac{5}{2}b$
ומהמשוואה השנייה כי
$\beta = a + \frac{3c}{2}$.
לכן
\[\text{.} D^*\prs{ax^2 + bx + c} = \frac{15}{2} b x^2 + \prs{a + \frac{3c}{2}}x - \frac{5}{2}b\]
\end{solution}

\printbibliography
\end{document}
