\documentclass[a4paper,10pt,twoside,openany]{article}

\usepackage[lang=hebrew]{maths}
\usepackage{polynom}
\usepackage{hebrewdoc}
\usepackage{stylish}
\usepackage{lipsum}
\let\bs\blacksquare

\setlength{\parindent}{0pt}

%%%%%%%%%%%%
% Styling %
%%%%%%%%%%%%

\usepackage{enumitem}

%%%%%%%%%%%%%
% Counters  %
%%%%%%%%%%%%%

\setcounter{section}{0}     
            
%BIBLIOGRAPHY
\usepackage[
backend=biber,
style=alphabetic,
]{biblatex}
\addbibresource{bibliography.bib} %Imports bibliography file

\title{
אלגברה ב' (01040168) - חורף 2026
\\
תרגול 4 - מרחבי מכפלה פנימית, וניצבות
\\
אלן סורני
\\
הרשימות עודכנו לאחרונה בתאריך ה־%
\today
}
\date{}

\begin{document}
\maketitle

\section{מוטיבציה}

נסתכל תחילה על המרחב הוקטורי
$\mbb{R}^n$.
כיוון שנוכל לחשוב על וקטורים ב־%
$\mbb{R}^n$
בתור נקודות במרחב, נוכל לחשוב על המרחק בין שני וקטורים
$u,v \in \mbb{R}^n$,
שנסמנו
$d\prs{u,v}$.

כדי לחשב מרחק כזה, נסתכל על האורך של הקטע המחבר בין
$u,v$,
וזה אותו אורך כמו של הקטע המחבר בין
$0,u-v$.
לכן,
$d\prs{u,v} = d\prs{0,u-v}$.
נקרא למרחק
$d\prs{0,u-v}$
\emph{האורך}
של
$u-v$,
כיוון שזה האורך של הקו המחבר בין
$0,u-v$,
ונסמנו
$\norm{u-v}$
בדומה לסימון
$\abs{z}$
של ערך מוחלט ב־%
$\mbb{C}$.
נוכל להשתמש במשפט פיתגורס כדי לקבל כי
\[\norm{\pmat{v_1 \\ \vdots \\ v_n}} = \sqrt{v_1^2 + \ldots + v_n^2}\]
ולמשל מתקיים
$\norm{\pmat{a\\b}} = \sqrt{a^2 + b^2} = \abs{a + ib}$.

כשמדובר על גיאומטריה אוקלידית, מושג חשוב בנוסף למרחק ואורך הוא זה של זווית. נסתכל על שני וקטורים
$u,v$
באורך
$1$
ועל הישרים
$\ell_u, \ell_v$
שעוברים דרכם.
הקוסינוס
$\cos\prs{\alpha}$
של הזווית מ־%
$\ell_u$
ל־%
$\ell_v$
שווה לאורך של הוקטור שהקצה שלו הוא החיתוך בין
$\ell_u$
לאנך מ־%
$v$
ל־%
$\ell_u$.
נסמן וקטור זה
$\trs{v,u} u$,
כיוון שהוא אכן כפולה של
$u$
מהיותו על
$\ell_u$.
במקרה זה נקרא לו
\emph{ההטלה של
$v$
על
$u$}.
אז יתקיים
\[\cos\prs{\alpha} = \trs{v,u}\]
ולכן
\[\text{.} \alpha = \acos\prs{\trs{v,u}}\]

כדי להכליל את המושג של זווית, נכליל את הביטוי
$\trs{v,u}$
באמצעות ההגדרה הבאה, של מכפלה פנימית.

\section{מכפלות פנימיות}

\begin{definition}[מכפלה פנימית]
יהי
$V$
מרחב וקטורי מעל
$\mbb{F} \in \set{\mbb{R}, \mbb{C}}$.
\emph{מכפלה פנימית}
על
$V$
היא פונקציה
\begin{align*}
\trs{\cdot,\cdot} \colon V \times V &\to \mbb{F}
\end{align*}
המקיימת את התכונות הבאות.
\begin{description}
\item[חיוביות:]
לכל
$v \in V \setminus \set{0}$
מתקיים
$\trs{v,v} \geq 0$.

\item[סימטריות (הרמיטיות):]
לכל
$u,v \in V$
מתקיים
$\trs{u,v} = \overline{\trs{v,u}}$.

\item[לינאריות ברכיב הראשון:]
לכל
$u,v \in V$
ולכל
$\alpha \in \mbb{F}$
מתקיים
\[\text{.}\trs{\alpha u + v, w} = \alpha\trs{u,w} + \trs{v,w}\]
\end{description}

מרחב וקטורי
$V$
יחד עם מכפלה פנימית
$\trs{\cdot,\cdot}$
נקרא
\emph{מרחב מכפלה פנימית}.
\end{definition}

\begin{remark}
בודגמא מהמרחב האוקלידי, ניתן לראות שנובע מהדרישה הגיאומטרית שלנו כי
\[\trs{u,v} = \sum_{i \in [n]} u_i v_i\]
עבור
$u,v \in \mbb{R}^n$
מאורך
$1$.
נוכל בעצם להגדיר זאת עבור וקטור כללי, ונשים לב כי הדבר אכן מקיים את שלוש התכונות הדרושות, ולכן הינו מכפלה פנימית על
$\mbb{R}^n$.

מכפלה פנימית זאת נקראת
\emph{המכפלה הפנימית הסטנדרטית}
על
$\mbb{R}^n$
ולעתים נסמן אותה
$u \cdot v \coloneqq \trs{u,v}_{\mrm{std}}$.
\end{remark}

\begin{exercise}
קיבעו אלו מההעתקות הבאות הן מכפלות פנימיות.

\begin{enumerate}
\item
\begin{align*}
f_1 \colon \mbb{R}^3 \times \mbb{R}^3 \mbb{R} \\
\prs{\pmat{a\\b\\c}, \pmat{x\\y\\z}} &\mapsto ax + by + \prs{cz}^2
\end{align*}

\item
\begin{align*}
f_2 \colon \mbb{R}^4 \times \mbb{R}^4 &\to \mbb{R} \\
\prs{\pmat{a\\b\\c\\d}, \pmat{w\\x\\y\\z}} &\mapsto ax + by + cz 
\end{align*}

\item
\begin{align*}
f_3 \colon \mbb{C}^3 \times \mbb{C}^3 &\to \mbb{C} \\
\prs{\pmat{a\\b\\c}, \pmat{x\\y\\z}} &\mapsto ax + by + cz
\end{align*}

\item
\begin{align*}
f_4 \colon \Mat_n\prs{\mbb{C}} \times \Mat_n\prs{\mbb{C}} &\to \mbb{C} \\
\prs{A,B} &\mapsto \tr\prs{B^t A}
\end{align*}
\end{enumerate}
\end{exercise}

\begin{solution}
אף אחת מההעתקות אינה מכפלה פנימית.
\begin{enumerate}
\item
ההעתקה
$f_1$
אינה לינארית ברכיב הראשון, כי
\[f_1\prs{\pmat{0\\0\\1}, \pmat{0\\0\\1}} = 1\]
ואילו
\[\text{.} f_1\prs{\pmat{0\\0\\2}, \pmat{0\\0\\1}} = 2^2 = 4 \neq 2 = 2 f_1\prs{\pmat{0\\0\\1}, \pmat{0\\0\\1}}\]

\item
ההעתקה
$f_2$
אינה חיובית, כי
\[\text{.} f_2\prs{\pmat{0\\0\\0\\1}, \pmat{0\\0\\0\\1}} = 0 \leq 0\]

\item
ההעתקה
$f_3$
אינה הרמיטית, כי
\[f_3\prs{\pmat{0\\0\\1}, \pmat{0\\0\\i}} = i\]
ואילו
\[\text{.} f_3\prs{\pmat{0\\0\\i}, \pmat{0\\0\\1}} = i \neq -i = \bar{i} = \overline{f_3\prs{\pmat{0\\0\\1}, \pmat{0\\0\\i}}}\]
\end{enumerate}

\item
ההעתקה
$f_4$
אינה הרמיטית, כי
$f_4\prs{I_n, iI_n} = \tr\prs{iI_n} = in$
ואילו
\[\text{.} f_4\prs{i I_n, I_n} = \tr\prs{iI_n} =in \neq \overline{in} = \overline{f_4\prs{I_n, iI_n}}\]
\end{solution}

\section{תכונות של מכפלות פנימיות, ונורמות}

במרחב האוקלידי, כאשר
$\norm{u} = \norm{v} = 1$,
אמרנו שהערך
$\abs{\trs{u,v}}$
שווה לאורך של ההטלה של
$v$
על
$u$
(או להיפך, כי המכפלה הפנימית סימטרית).
בפרט, אורך זה יכול להיות לכל היותר
$1$,
והינו שווה
$1$
אם ורק אם
$u = v$,
כי אחרת ההטלה תהיה קצרה יותר.

ניעזר בלינאריות של המכפלה הפנימית ונקבל כי לוקטורים
$u,v \in \mbb{R} \setminus \set{0}$
כלליים מתקיים
\begin{align*}
\abs{\trs{u,v}} &= \trs{\norm{u} \frac{u}{\norm{u}}, \norm{v} \frac{v}{\norm{v}}}
\\&=
\norm{u} \norm{v} \abs{\trs{\frac{u}{\norm{u}}, \frac{v}{\norm{v}}}}
\\&\leq \norm{u} \norm{v}
\end{align*}
כאשר
$\abs{\trs{\frac{u}{\norm{u}}, \frac{v}{\norm{v}}}} \leq 1$
כי הוקטורים
$\frac{u}{\norm{u}}, \frac{v}{\norm{v}}$
הינם מאורך
$1$.

אי־שוויון כזה מתקבל גם במרחב מכפלה פנימית כללי, כפי שקובע אי־שוויון קושי־שוורץ, אך לשם כך עלינו להגדיר מושג של אורך בהגדרה הבאה.

\begin{definition}[נורמה]
יהי
$V$
מרחב מכפלה פנימית מעל
$\mbb{F} \in \set{\mbb{R}, \mbb{C}}$.
\emph{נורמה}
על
$V$
היא פונקציה
$\norm{\cdot} \colon V \to \mbb{R}$
המקיימת את התכונות הבאות.

\begin{description}
\item[חיוביות:]
לכל
$v \in V \setminus \set{0}$
מתקיים
$\norm{v} > 0$.

\item[הומוגניות:]
לכל
$v \in V$
ולכל
$\alpha \in \mbb{F}$
מתקיים
$\norm{\alpha v} = \abs{\alpha} \norm{v}$.

\item[אי־שוויון המשולש:]
לכל
$u,v \in V$
מתקיים
$\norm{u+v} \leq \norm{u} + \norm{v}$.
\end{description}

מרחב וקטורי עם נורמה נקרא
\emph{מרחב נורמי}.
\end{definition}

\begin{theorem}[אי־שוויון קושי־שוורץ]
יהי
$V$
מרחב מכפלה פנימית. אז לכל
$u,v \in V$
מתקיים
\[\abs{\trs{u,v}} \leq \norm{u} \cdot \norm{v}\]
ושוויון מתקיים אם ורק אם
$u,v$
תלויים לינארית.
\end{theorem}

\begin{exercise}
יהי
$V$
מרחב מכפלה פנימית, ויהיו
$u_1, \ldots, u_n, v_1, \ldots, v_n \in V$.
הראו שמתקיים
\begin{align*}
\text{.} \sum_{i = 1}^n \trs{u_i, v_i} \leq \sqrt{{\sum_{i = 1}^n \norm{u_i}^2}} \cdot \sqrt{\sum_{i=1}^n \norm{v_i}^2}
\end{align*}
\end{exercise}

\begin{solution}
נרצה לפרש את אגף שמאל בתור מכפלה פנימית.
כיוון שהוא מזכיר את המכפלה הפנימית על
$\mbb{R}^n$,
רק עם וקטורים מ־%
$V$
במקום מספרים ב־%
$\mbb{R}$,
נרצה להגדיר מכפלה זאת על המרחב
\[\text{.} V^n \coloneqq \set{\prs{v_1, \ldots, v_n}}{v_i \in V}\]
זה אכן מרחב וקטורי עם חיבור וכפל בסקלר לפי כל קואורדינטה בנפרד, ונגדיר עליו מכפלה פנימית לפי
\[\text{.} \trs{\prs{u_1, \ldots, u_n}, \prs{v_1, \ldots, v_n}} \coloneqq \sum_{i = 1}^n \trs{u_i, v_i}\]
אם
$v \coloneqq \prs{v_1, \ldots, v_n} \neq \prs{0, \ldots, 0}$,
יש
$v_j \neq 0$
ולכן
\[\text{,} \trs{v,v} \geq \trs{v_j, v_j} > 0\]
לכון מתקיימת חיוביות.
סימטריה ולינאריות ברכיב הראשון מתקיימות בכל רכיב בנפרד, ולכן גם בסך הכל.

כעת, אי־שוויון קושי־שוורץ על
$V^n$
אומר לנו כי
\begin{align*}
\sum_{i = 1}^n \trs{u_i, v_i}
&\leq
\abs{\sum_{i = 1}^n \trs{u_i, v_i}}
\\&=
\abs{\trs{u,v}}
\\&\leq
\norm{u} \norm{v}
\\&=
\sqrt{\trs{u,u}} \cdot \sqrt{v,v}
\\&=
\sqrt{{\sum_{i = 1}^n \trs{u_i, u_i}}} \cdot \sqrt{\sum_{i=1}^n \trs{v_i, v_i}}
\\\text{,}\hphantom{\sum_{i = 1}^n \trs{u_i, v_i}}&=
\sqrt{{\sum_{i = 1}^n \norm{u_i}^2}} \cdot \sqrt{\sum_{i=1}^n \norm{v_i}^2}
\end{align*}
כנדרש.
\end{solution}

\section{בסיסים אורתונורמליים והמרחב הניצב}

\begin{definition}[מטריקה המושרית מנורמה]
יהי
$\prs{V,\norm{\cdot}}$
מרחב נורמי.
\emph{המטריקה המושרית}
על
$V$
היא
$d\prs{x,y} = \norm{x-y}$.
\end{definition}

כעת, את המרחק בין שני שני וקטורים קל בדרך כלל לחשב כאשר יש לנו נורמה נתונה. מעט מסובך יותר לחשב את המרחב מוקטור לקבוצה.

\begin{definition}[מרחק מקבוצה]
יהי
$\prs{X,d}$
מרחב מטרי, יהי
$x \in X$
ותהי
$S \subseteq X$.
נגדיר את
\emph{המרחק של
$x$
מ־%
$S$}
להיות
\begin{align*}
\text{.} d\prs{x,S} \coloneqq \inf\set{d\prs{x,s}}{s \in S}
\end{align*}
\end{definition}

תת־קבוצות של מרחב וקטורי שמעניינות אותנו הן בדרך כלל תת־מרחבים. במקרה האוקלידי, אנו יודעות כיצד לחשב את המרחק מתת־מרחב. כדי לחשב את
$d\prs{x,W}$
עבור
$W \leq \mbb{R}^n$,
נעביר אנך מ־%
$x$
ל־%
$W$
ונחשב את המרחק בין
$x$
ונקודת החיתוך בין
$W$
לאנך.

כדי לעשות דבר דומה במרחבי מכפלה פנימית כלליים, נצטרך לדבר קודם כל על ניצבות.

\begin{definition}[זווית במרחב מכפלה פנימית]
יהי
$V$
מרחב מכפלה פנימית ויהיו
$u,v \in V$.
נגדיר את
\emph{הזווית}
בין
$u,v$
בתור
\[\text{.} \angle\prs{u,v} = \acos\prs{{\frac{\Re\prs{\trs{u,v}}}{\norm{u} \norm{v}}}}\]
\end{definition}

\begin{definition}[וקטורים ניצבים]
יהי
$V$
מרחב מכפלה פנימית.
וקטורים
$u,v \in V$
נקראים
\emph{ניצבים}
אם
$\trs{u,v} = 0$.
במקרה זה נסמן
$u \perp v$.
\end{definition}

\begin{remark}
מעל
$\mbb{R}$,
וקטורים ניצבים בדיוק כאשר הזווית בינם היא
$\frac{\pi}{2}$.
\end{remark}

\begin{definition}
קבוצה
$S \subseteq V$
במרחב מכפלה פנימית נקראת
\emph{אורתוגונלית}
אם
$s_1 \perp s_2$
לכל
$s_1, s_2 \in S$
שונים.
\end{definition}

\begin{theorem}[פיתגורס]
תהי
$\prs{v_1, \ldots, v_n}$
סדרה של וקטורים אורתוגונליים במרחב מכפלה פנימית
$V$.
אז
\begin{align*}
\text{.} \norm{v_1 + \ldots + v_n}^2 = \norm{v_1}^2 + \ldots + \norm{v_n}^2
\end{align*}
\end{theorem}

כדי לדבר על המרחק של וקטור
$v \in V$
מתת־מרחב
$W \leq V$,
נרצה לכתוב את
$v$
בתור סכום של שני וקטורים, כאשר אחד מהם מ־%
$W$
והשני ניצב ל־%
$W$.
ראינו כי
$V = W \oplus W^{\perp}$
עבור
$W^\perp$
תת־המרחב של הוקטורים הניצבים ל־%
$W$
ולכן זה אפשרי.
ככה, ושהמרחק יצא, כמו במקרה האוקלידי, האורך של החלק הניצב ל־%
$W$.

\begin{definition}[מרחב ניצב]
יהי
$V$
מרחב מכפלה פנימית, ותהי
$S \subseteq V$.
\emph{המרחב הניצב ל־%
$S$}
הוא
\[\text{.} S^\perp = \set{v \in V}{\forall s \in S \colon v \perp s}\]
\end{definition}

\begin{proposition}
יהי
$V$
מרחב מכפלה פנימית ויהי
$W \leq V$.
מתקיים
$V = W \oplus W^\perp$.
\end{proposition}

\begin{exercise}
יהי
$V$
מרחב מכפלה פנימית סוף־מימדי ותהיינה
$S,T \subseteq V$.

\begin{enumerate}
\item 
נניח כי
$S \subseteq T$.
הראו כי
$T^\perp \subseteq S^\perp$.

\emph{התאמה כזאת, כמו $S \mapsto S^\perp$, שהופכת יחס הכלה, נקראת
\emph{התאמת גלואה}.}

\item
נסמן
$W = \Span\prs{S}$.
הראו כי
$S^\perp = W^\perp$.

\item
הראו כי
$\prs{S^{\perp}}^\perp = \Span\prs{S}$.
\end{enumerate}
\end{exercise}

\begin{solution}
\begin{enumerate}
\item
יהי
$v \in T^\perp$
ויהי
$s \in S$.
אז
$s \in T$
כי
$S \subseteq T$,
ולכן
$v \perp s$,
כי
$v$
ניצב לכל וקטור ב־%
$T$.
לכן
$v \in S^\perp$.

\item
מתקיים
$S \subseteq W$
ולכן
$W^\perp \subseteq S^\perp$.
יהי
$v \in S^\perp$
ויהי
$w \in W$.
כיוון ש־%
$W = \Span\prs{S}$,
יש איברים
$s_1, \ldots, s_k$
וסקלרים
$\alpha_1, \ldots, \alpha_k$
עבורם
$w = \sum_{i \in [k]} \alpha_i s_i$.
אז
\[\trs{v,w} = \trs{v, \sum_{i \in [k]} \alpha_i s_i} = \sum_{i \in [k]} \alpha_i \cancelto{0}{\trs{v_i, s_i}} = 0\]
ולכן
$v \in W^\perp$.

\item
ראינו כי
$\prs{W^\perp}^\perp = W$
עבור
$W \leq V$.
ניקח
$W = \Span\prs{S}$.
מתקיים
$S \subseteq W$
ולכן
$W^\perp \subseteq S^\perp$
ואז
$\prs{S^\perp}^\perp \subseteq \prs{W^\perp}^\perp = W$.
כמו כן,
$S^\perp = W^\perp$
תת־מרחב וקטורי ומכיוון ש־%
$V = S^\perp \oplus \prs{S^\perp}^\perp$,
נקבל כי
\[\dim\prs{\prs{S^\perp}^\perp} = \dim\prs{V} - \dim\prs{S^\perp} = \dim\prs{V} - \dim\prs{W^\perp} = \dim\prs{W}\]
ולכן יש שוויון
$\prs{S^\perp}^\perp = W = \Span\prs{S}$.
\end{enumerate}
\end{solution}

\begin{remark}
לא יכולנו בתרגיל האחרון להניח כי
$S$
אינסופית. באופן כללי,
\[\Span\prs{S} = \set{\sum_{i \in [k]} \alpha_i s_i}{\substack{\alpha_i \in \mbb{F} \\ s_i \in S}}\]
גם כאשר
$S$
אינסופית.
\end{remark}

\begin{exercise}
מצאו את
$W^\perp$
עבור
\[\text{.} W \ceq \spn\prs{e_1 + e_2} \leq \mbb{R}^2\]
\end{exercise}

\begin{solution}
ראינו כי מתקיים
$v \in \Span\prs{e_1 + e_2}^\perp$
אם ורק אם
$v \perp e_1 + e_2$.
זה מתקיים
אם ורק אם
$v_1 + v_2 = 0$
אם ורק אם
$v_2 = - v_1$.
לכן
\[\text{.} W^\perp = \spn\prs{e_1 - e_2}\]
\end{solution}

\begin{definition}[בסיס אורתונורמלי]
יהי
$V$
מרחב מכפלה פנימית.
בסיס
$B = \prs{v_1, \ldots, v_n}$
של
$V$
נקרא
\emph{אורתוגונלי}
אם
$\trs{v_i, v_j} = 0$
לכל
$i \neq j$,
ו%
\emph{אורתונורמלי}
אם
$\trs{v_i, v_j} = \delta_{i,j} \coloneqq \fcases{0 & i \neq j \\ 1 & i=j}$.
\end{definition}

\begin{definition}[הטלה אורתוגונלית]
יהי
$V$
מרחב מכפלה פנימית ויהי
$W \leq V$.
\emph{ההטלה האורתוגונלית על
$W$}
היא ההטלה על
$W$
ביחס לסכום הישר
$V = W \oplus W^\perp$.
\end{definition}

\begin{proposition}
יהי
$V$
מרחב מכפלה פנימית ויהי
$W \leq V$.
יהי
$B = \prs{w_1, \ldots, w_m}$
בסיס
\textbf{אורתונורמלי}
של
$W$
ויהי
$v \in V$.
תהי
$P_W$
ההטלה האורתוגונלית על
$W$.
אז
\[\text{.} P_W\prs{w} = \sum_{i \in [m]} \trs{v, w_i} w_i\]
\end{proposition}

\begin{exercise}
יהי
$V$
מרחב מכפלה פנימית ויהיו
$u,v \in V$.
הראו כי
$u \perp v$
אם ורק אם
\[\norm{u} \leq \norm{u+av}\]
לכל
$a \in \mbb{F}$.
\end{exercise}

\begin{solution}
אם
$u \perp v$
ו־%
$a \in \mbb{F}$,
נקבל מפיתגורס
\begin{align*}
\norm{u+av}^2 &= \norm{u}^2 + \abs{a}^2 \norm{v}^2 \geq \norm{u}^2
\end{align*}
ולכן
$\norm{u} \leq \norm{u+av}$.

נניח כי
$\trs{u,v} \neq 0$
ונניח תחילה כי
$\norm{v} = 1$.
אז
\[\trs{\trs{u,v}v, v} = \trs{u,v} \trs{v,v} = \trs{u,v} \norm{v}^2 = \trs{u,v} \neq 0\]
וגם
\[\text{.} \trs{u - \trs{u,v}v, v} = \trs{u,v} - \trs{\trs{u,v}v,v} = 0\]
אז, ממשפט פיתגורס
\begin{align*}
\norm{u}^2 &= \norm{u - \trs{u,v}v + \trs{u,v}v}^2
\\&=
\norm{u - \trs{u,v}v}^2 + \norm{\trs{u,v}v}^2
\\&>
\norm{u-\trs{u,v}v}^2
\end{align*}
כאשר האי־שוויון חזק כי
$\trs{u,v}v \neq 0$
מההנחה
$\trs{u,v} \neq 0$.
לכן, עבור
$a = \trs{u,v}$
לא מתקיים
$\norm{u} \leq \norm{u+av}$.

כעת, אם
$v$
כללי, הוקטור
$\frac{v}{\norm{v}}$
הינו מאורך
$1$
ולכן יש
$a' \in \mbb{F}$
עבורו
$\norm{u} > \norm{u+a' \frac{v}{\norm{v}}}$.
אז ניקח
$a = \frac{a'}{\norm{v}}$
ונקבל כי
$\norm{u} > \norm{u + av}$.
\end{solution}

כדי למצוא בסיסים אורתונורמליים ומשלימים ישרים, ניעזר בתהליך שלוקח בסיס כלשהו ומחזיר בסיס אורתונורמלי.

\begin{theorem}[גרם־שמידט]
יהי
$V$
מרחב מכפלה פנימית ויהי
$B = \prs{u_1, \ldots, u_n}$
בסיס של
$V$.
קיים בסיס אורתונורמלי
$C = \prs{v_1, \ldots, v_n}$
של
$V$
עבורו
\[\Span\prs{u_1, \ldots, u_i} = \Span\prs{v_1, \ldots, v_i}\]
לכל
$i \in \brs{n}$.
\end{theorem}

ניסוח זה של המשפט לא מתאר לנו איך למצוא את
$C$,
אבל ההוכחה שלו קונסטרוקטיבית ומתארת את האלגוריתם הבא.

\begin{enumerate}
\item
עבור
$i = 1$
ניקח
$v_i = \frac{u_i}{\norm{u_i}}$.

\item
עבור כל
$i$
לאחר מכן, לפי הסדר, ניקח
\[w_i = u_i - \sum_{j \in \brs{i-1}} \trs{u_i, v_j}\]
ואז
$v_i = \frac{w_i}{\norm{w_i}}$.
\end{enumerate}

\begin{corollary}
יהי
$W \leq V$
תת־מרחב במרחב מכפלה פנימית.
כדי למצוא אורתונורמלי של
$W$
ושל
$W^\perp$
ניקח בסיס
$B_W$
של
$W$
ונשלים אותו לבסיס
$B = B_W \cup B'_W$
של
$V$.
נבצע את תהליך גרם־שמידט על
$B$
ונקבל בסיס
$C = C_W \cup C_W^\perp$
כך ש־%
$\Span\prs{C_W} = \Span\prs{B_W} = W$.
כל הוקטורים ב־%
$C_W^\perp$
ניצבים ל־%
$W$
כי
$C$
אורתונורמלי, ולכן
$W' \coloneqq \Span\prs{C_W^\perp} \subseteq W^\perp$.
כמו כן,
\[\dim\prs{W'} = \dim\prs{V} - \dim\prs{W} = \dim\prs{W^\perp}\]
כי
$V = W \oplus W' = W \oplus W^\perp$,
ולכן יש שוויון
$W' = W^\perp$.
\end{corollary}

\begin{theorem}[מרחק של וקטור מתת־מרחב]
יהי
$V$
מרחב מכפלה פנימית, יהי
$W \leq V$,
תהי
$P_W$
ההטלה האורתוגונלית על
$W$
ויהי
$v \in V$.
מתקיים
\[\text{.} d\prs{v,W} = d\prs{v, p_W\prs{v}}\]
\end{theorem}

\begin{exercise}
יהי
$V = \Mat_2\prs{\mbb{R}}$
עם המכפלה הפנימית הסטנדרטית
\[\trs{A,B} = \tr\prs{B^t A}\]
ויהי
$W \leq V$
התת־מרחב של המטריצות הסימטריות.

\begin{enumerate}
\item מיצאו בסיס אורתונורמלי עבור
$W$
ועבור
$W^\perp$.

\item
הראו כי
$P_W\prs{A} = \frac{A + A^t}{2}$
בשתי דרכים שונות.

\item
חשבו את המרחק של
$A = \pmat{1 & 2 \\ 3 & 4}$
מ־%
$W$.
\end{enumerate}
\end{exercise}

\begin{solution}
\begin{enumerate}
\item
ניקח בסיס
$B_W = \prs{E_{1,1}, E_{1,2} + E_{2,1}, E_{2,2}}$
של
$W$
ונשלים אותו לבסיס
\[B = \prs{u_1, u_2, u_3, u_4} = \prs{E_{1,1}, E_{1,2} + E_{2,1}, E_{2,2}, E_{1,2}}\]
של
$V$.
נבצע את תהליך גרם־שמידט כדי לקבל בסיס אורתונורמלי
$\prs{v_1, v_2, v_3, v_4}$
עבורו
$W = \Span\prs{v_1, v_2, v_3}$
וגם
$W^\perp = \Span\prs{v_4}$.

נחשב
\begin{align*}
v_1 &= \frac{1}{\norm{E_{1,1}}} E_{1,1} = E_{1,1} \\
w_2 &= u_2 - \trs{u_2,v_1} v_1 = E_{1,2} + E_{2,1} - 0 = E_{1,2} + E_{2,1} \\
v_2 &= \frac{1}{\norm{w_2}} w_2 = \frac{1}{\sqrt{2}} \prs{E_{1,2} + E_{2,1}} \\
w_3 &= u_3 - \trs{u_3, v_2} v_2 - \trs{u_3, v_1} v_1 = u_3 = E_{2,2} \\
v_3 &= \frac{1}{\norm{w_3}} w_3 = E_{2,2} \\
w_4 &= u_4 - \trs{u_4, v_3}v_3 - \trs{u_4, v_2}v_2 - \trs{u_4, v_1}v_1 \\&= E_{1,2} - \trs{E_{1,2}, \frac{1}{\sqrt{2}} \prs{E_{1,2} + E_{2,1}}} \prs{\frac{1}{\sqrt{2}} \prs{E_{1,2} + E_{2,1}}}
\\&= E_{1,2} - \frac{1}{2} \prs{E_{1,2} + E_{2,1}}
\\&= \frac{1}{2} \prs{E_{1,2} - E_{2,1}} \\
v_4 &= \frac{w_4}{\norm{w_4}} = \frac{1}{\norm{E_{1,2} - E_{2,1}}} \prs{E_{1,2} - E_{2,1}} = \frac{1}{\sqrt{2}} \prs{E_{1,2} - E_{2,1}}
\end{align*}
ונקבל כי
\[\prs{v_1, v_2, v_3} = \prs{E_{1,1}, \frac{1}{\sqrt{2}} \prs{E_{1,2} + E_{2,1}}, E_{2,2}}\]
בסיס אורתונורמלי של
$W$
וכי
$\frac{1}{\sqrt{2}} \prs{E_{1,2} - E_{2,1}}$
בסיס אורתונורמלי של
$W^\perp$.
אז,
$W^\perp$
מרחב המטריצות האנטיסימטריות.

\item
בדרך אחת, זכור לנו מאלגברה א' כי
\[A = \frac{A + A^t}{2} + \frac{A - A^t}{2}\]
כאשר
$\frac{A + A^t}{2}$
סימטרית ו־%
$\frac{A - A^t}{2}$
אנטי־סימטרית.
אז
$\frac{A + A^t}{2} \in W$
וכיוון ש־%
$W^\perp$
מרחב המטריצות האנטיסימטריות, נקבל גם
$\frac{A - A^t}{2} \in W^\perp$.
לכן
$P_W\prs{A} = \frac{A + A^t}{2}$.

אבל, במקרה הכללי, לא נדע מראש איך לכתוב וקטור
$v \in V$
בתור סכום של וקטור ב־%
$W \leq V$
ווקטור ב־%
$W^\perp$.
כדי לחשב את ההטלה
$P_W\prs{A}$
נוכל לקחת בסיס אורתונורמלי של
$W$
ולהשתמש בנוסחא עבור ההטלה האורתוגונלית לפי בסיס כזה.

אכן,
\begin{align*}
P_W\prs{A} &= \sum_{i \in [3]} \trs{A, v_i} v_i \\&= \trs{A, E_{1,1}} E_{1,1} + \trs{A, \frac{1}{\sqrt{2}} \prs{E_{1,2} + E_{2,1}}} \cdot \frac{1}{\sqrt{2}} \prs{E_{1,2} + E_{2,1}} + \trs{A, E_{2,2}}, E_{2,2}
\\&=
a_{1,1} E_{1,1} + \frac{1}{2} \prs{a_{1,2} + a_{2,1}} \prs{E_{1,2} + E_{2,1}} + a_{2,2} E_{2,2}
\\&=
\pmat{a_{1,1} & \frac{a_{1,2} + a_{2,1}}{2} \\ \frac{a_{1,2} + a_{2,1}}{2} & a_{2,2}}
\\ \text{.} \hphantom{P_W\prs{A}} &=
\frac{A + A^t}{2}
\end{align*}

\item
נכתוב את
$A$
בתור סכום של מטריצה סימטרית ואנטי־סימטרית. ראינו כי
\[A = \frac{A + A^t}{2} + \frac{A - A^t}{2}\]
כאשר
$\frac{A + A^t}{2}$
סימטרית ו־%
$\frac{A - A^t}{2}$
אנטי־סימטרית.
אז
$d_W\prs{A} = \frac{A+A^t}{2}$
ונקבל כי
המרחק של
$A$
מ־%
$W$
הוא
\[\text{.} d\prs{A, \frac{A + A^t}{2}} = \norm{A - \frac{A + A^t}{2}}\]
מתקיים
\begin{align*}
\norm{A - \frac{A + A^t}{2}} &= \norm{\frac{A - A^t}{2}}
\\&= \frac{1}{2} \norm{\pmat{0 & -1 \\ & 1 & 0}}
\\&= \frac{1}{2} \cdot \sqrt{2}
\\&= \frac{1}{\sqrt{2}}
\end{align*}
ולכן
$d\prs{A, W} = \frac{1}{\sqrt{2}}$.
\end{enumerate}
\end{solution}

\printbibliography
\end{document}
